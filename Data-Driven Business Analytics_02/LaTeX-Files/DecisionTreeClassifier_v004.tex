\documentclass[paper=landscape]{scrartcl}
	\usepackage[paperwidth=40cm,paperheight=320cm,margin=1in]{geometry}

    \usepackage[breakable]{tcolorbox}
    \usepackage{parskip} % Stop auto-indenting (to mimic markdown behaviour)
    
    \usepackage{iftex}
    \ifPDFTeX
    	\usepackage[T1]{fontenc}
    	\usepackage{mathpazo}
    \else
    	\usepackage{fontspec}
    \fi

    % Basic figure setup, for now with no caption control since it's done
    % automatically by Pandoc (which extracts ![](path) syntax from Markdown).
    \usepackage{graphicx}
    % Maintain compatibility with old templates. Remove in nbconvert 6.0
    \let\Oldincludegraphics\includegraphics
    % Ensure that by default, figures have no caption (until we provide a
    % proper Figure object with a Caption API and a way to capture that
    % in the conversion process - todo).
    \usepackage{caption}
    \DeclareCaptionFormat{nocaption}{}
    \captionsetup{format=nocaption,aboveskip=0pt,belowskip=0pt}

    \usepackage[Export]{adjustbox} % Used to constrain images to a maximum size
    \adjustboxset{max size={0.9\linewidth}{0.9\paperheight}}
    \usepackage{float}
    \floatplacement{figure}{H} % forces figures to be placed at the correct location
    \usepackage{xcolor} % Allow colors to be defined
    \usepackage{enumerate} % Needed for markdown enumerations to work
    \usepackage{geometry} % Used to adjust the document margins
    \usepackage{amsmath} % Equations
    \usepackage{amssymb} % Equations
    \usepackage{textcomp} % defines textquotesingle
    % Hack from http://tex.stackexchange.com/a/47451/13684:
    \AtBeginDocument{%
        \def\PYZsq{\textquotesingle}% Upright quotes in Pygmentized code
    }
    \usepackage{upquote} % Upright quotes for verbatim code
    \usepackage{eurosym} % defines \euro
    \usepackage[mathletters]{ucs} % Extended unicode (utf-8) support
    \usepackage{fancyvrb} % verbatim replacement that allows latex
    \usepackage{grffile} % extends the file name processing of package graphics 
                         % to support a larger range
    \makeatletter % fix for grffile with XeLaTeX
    \def\Gread@@xetex#1{%
      \IfFileExists{"\Gin@base".bb}%
      {\Gread@eps{\Gin@base.bb}}%
      {\Gread@@xetex@aux#1}%
    }
    \makeatother

    % The hyperref package gives us a pdf with properly built
    % internal navigation ('pdf bookmarks' for the table of contents,
    % internal cross-reference links, web links for URLs, etc.)
    \usepackage{hyperref}
    % The default LaTeX title has an obnoxious amount of whitespace. By default,
    % titling removes some of it. It also provides customization options.
    \usepackage{titling}
    \usepackage{longtable} % longtable support required by pandoc >1.10
    \usepackage{booktabs}  % table support for pandoc > 1.12.2
    \usepackage[inline]{enumitem} % IRkernel/repr support (it uses the enumerate* environment)
    \usepackage[normalem]{ulem} % ulem is needed to support strikethroughs (\sout)
                                % normalem makes italics be italics, not underlines
    \usepackage{mathrsfs}
    

    
    % Colors for the hyperref package
    \definecolor{urlcolor}{rgb}{0,.145,.698}
    \definecolor{linkcolor}{rgb}{.71,0.21,0.01}
    \definecolor{citecolor}{rgb}{.12,.54,.11}

    % ANSI colors
    \definecolor{ansi-black}{HTML}{3E424D}
    \definecolor{ansi-black-intense}{HTML}{282C36}
    \definecolor{ansi-red}{HTML}{E75C58}
    \definecolor{ansi-red-intense}{HTML}{B22B31}
    \definecolor{ansi-green}{HTML}{00A250}
    \definecolor{ansi-green-intense}{HTML}{007427}
    \definecolor{ansi-yellow}{HTML}{DDB62B}
    \definecolor{ansi-yellow-intense}{HTML}{B27D12}
    \definecolor{ansi-blue}{HTML}{208FFB}
    \definecolor{ansi-blue-intense}{HTML}{0065CA}
    \definecolor{ansi-magenta}{HTML}{D160C4}
    \definecolor{ansi-magenta-intense}{HTML}{A03196}
    \definecolor{ansi-cyan}{HTML}{60C6C8}
    \definecolor{ansi-cyan-intense}{HTML}{258F8F}
    \definecolor{ansi-white}{HTML}{C5C1B4}
    \definecolor{ansi-white-intense}{HTML}{A1A6B2}
    \definecolor{ansi-default-inverse-fg}{HTML}{FFFFFF}
    \definecolor{ansi-default-inverse-bg}{HTML}{000000}

    % commands and environments needed by pandoc snippets
    % extracted from the output of `pandoc -s`
    \providecommand{\tightlist}{%
      \setlength{\itemsep}{0pt}\setlength{\parskip}{0pt}}
    \DefineVerbatimEnvironment{Highlighting}{Verbatim}{commandchars=\\\{\}}
    % Add ',fontsize=\small' for more characters per line
    \newenvironment{Shaded}{}{}
    \newcommand{\KeywordTok}[1]{\textcolor[rgb]{0.00,0.44,0.13}{\textbf{{#1}}}}
    \newcommand{\DataTypeTok}[1]{\textcolor[rgb]{0.56,0.13,0.00}{{#1}}}
    \newcommand{\DecValTok}[1]{\textcolor[rgb]{0.25,0.63,0.44}{{#1}}}
    \newcommand{\BaseNTok}[1]{\textcolor[rgb]{0.25,0.63,0.44}{{#1}}}
    \newcommand{\FloatTok}[1]{\textcolor[rgb]{0.25,0.63,0.44}{{#1}}}
    \newcommand{\CharTok}[1]{\textcolor[rgb]{0.25,0.44,0.63}{{#1}}}
    \newcommand{\StringTok}[1]{\textcolor[rgb]{0.25,0.44,0.63}{{#1}}}
    \newcommand{\CommentTok}[1]{\textcolor[rgb]{0.38,0.63,0.69}{\textit{{#1}}}}
    \newcommand{\OtherTok}[1]{\textcolor[rgb]{0.00,0.44,0.13}{{#1}}}
    \newcommand{\AlertTok}[1]{\textcolor[rgb]{1.00,0.00,0.00}{\textbf{{#1}}}}
    \newcommand{\FunctionTok}[1]{\textcolor[rgb]{0.02,0.16,0.49}{{#1}}}
    \newcommand{\RegionMarkerTok}[1]{{#1}}
    \newcommand{\ErrorTok}[1]{\textcolor[rgb]{1.00,0.00,0.00}{\textbf{{#1}}}}
    \newcommand{\NormalTok}[1]{{#1}}
    
    % Additional commands for more recent versions of Pandoc
    \newcommand{\ConstantTok}[1]{\textcolor[rgb]{0.53,0.00,0.00}{{#1}}}
    \newcommand{\SpecialCharTok}[1]{\textcolor[rgb]{0.25,0.44,0.63}{{#1}}}
    \newcommand{\VerbatimStringTok}[1]{\textcolor[rgb]{0.25,0.44,0.63}{{#1}}}
    \newcommand{\SpecialStringTok}[1]{\textcolor[rgb]{0.73,0.40,0.53}{{#1}}}
    \newcommand{\ImportTok}[1]{{#1}}
    \newcommand{\DocumentationTok}[1]{\textcolor[rgb]{0.73,0.13,0.13}{\textit{{#1}}}}
    \newcommand{\AnnotationTok}[1]{\textcolor[rgb]{0.38,0.63,0.69}{\textbf{\textit{{#1}}}}}
    \newcommand{\CommentVarTok}[1]{\textcolor[rgb]{0.38,0.63,0.69}{\textbf{\textit{{#1}}}}}
    \newcommand{\VariableTok}[1]{\textcolor[rgb]{0.10,0.09,0.49}{{#1}}}
    \newcommand{\ControlFlowTok}[1]{\textcolor[rgb]{0.00,0.44,0.13}{\textbf{{#1}}}}
    \newcommand{\OperatorTok}[1]{\textcolor[rgb]{0.40,0.40,0.40}{{#1}}}
    \newcommand{\BuiltInTok}[1]{{#1}}
    \newcommand{\ExtensionTok}[1]{{#1}}
    \newcommand{\PreprocessorTok}[1]{\textcolor[rgb]{0.74,0.48,0.00}{{#1}}}
    \newcommand{\AttributeTok}[1]{\textcolor[rgb]{0.49,0.56,0.16}{{#1}}}
    \newcommand{\InformationTok}[1]{\textcolor[rgb]{0.38,0.63,0.69}{\textbf{\textit{{#1}}}}}
    \newcommand{\WarningTok}[1]{\textcolor[rgb]{0.38,0.63,0.69}{\textbf{\textit{{#1}}}}}
    
    
    % Define a nice break command that doesn't care if a line doesn't already
    % exist.
    \def\br{\hspace*{\fill} \\* }
    % Math Jax compatibility definitions
    \def\gt{>}
    \def\lt{<}
    \let\Oldtex\TeX
    \let\Oldlatex\LaTeX
    \renewcommand{\TeX}{\textrm{\Oldtex}}
    \renewcommand{\LaTeX}{\textrm{\Oldlatex}}
    % Document parameters
    % Document title
    \title{DecisionTreeClassifier\_v004}
    
    
    
    
    
% Pygments definitions
\makeatletter
\def\PY@reset{\let\PY@it=\relax \let\PY@bf=\relax%
    \let\PY@ul=\relax \let\PY@tc=\relax%
    \let\PY@bc=\relax \let\PY@ff=\relax}
\def\PY@tok#1{\csname PY@tok@#1\endcsname}
\def\PY@toks#1+{\ifx\relax#1\empty\else%
    \PY@tok{#1}\expandafter\PY@toks\fi}
\def\PY@do#1{\PY@bc{\PY@tc{\PY@ul{%
    \PY@it{\PY@bf{\PY@ff{#1}}}}}}}
\def\PY#1#2{\PY@reset\PY@toks#1+\relax+\PY@do{#2}}

\expandafter\def\csname PY@tok@w\endcsname{\def\PY@tc##1{\textcolor[rgb]{0.73,0.73,0.73}{##1}}}
\expandafter\def\csname PY@tok@c\endcsname{\let\PY@it=\textit\def\PY@tc##1{\textcolor[rgb]{0.25,0.50,0.50}{##1}}}
\expandafter\def\csname PY@tok@cp\endcsname{\def\PY@tc##1{\textcolor[rgb]{0.74,0.48,0.00}{##1}}}
\expandafter\def\csname PY@tok@k\endcsname{\let\PY@bf=\textbf\def\PY@tc##1{\textcolor[rgb]{0.00,0.50,0.00}{##1}}}
\expandafter\def\csname PY@tok@kp\endcsname{\def\PY@tc##1{\textcolor[rgb]{0.00,0.50,0.00}{##1}}}
\expandafter\def\csname PY@tok@kt\endcsname{\def\PY@tc##1{\textcolor[rgb]{0.69,0.00,0.25}{##1}}}
\expandafter\def\csname PY@tok@o\endcsname{\def\PY@tc##1{\textcolor[rgb]{0.40,0.40,0.40}{##1}}}
\expandafter\def\csname PY@tok@ow\endcsname{\let\PY@bf=\textbf\def\PY@tc##1{\textcolor[rgb]{0.67,0.13,1.00}{##1}}}
\expandafter\def\csname PY@tok@nb\endcsname{\def\PY@tc##1{\textcolor[rgb]{0.00,0.50,0.00}{##1}}}
\expandafter\def\csname PY@tok@nf\endcsname{\def\PY@tc##1{\textcolor[rgb]{0.00,0.00,1.00}{##1}}}
\expandafter\def\csname PY@tok@nc\endcsname{\let\PY@bf=\textbf\def\PY@tc##1{\textcolor[rgb]{0.00,0.00,1.00}{##1}}}
\expandafter\def\csname PY@tok@nn\endcsname{\let\PY@bf=\textbf\def\PY@tc##1{\textcolor[rgb]{0.00,0.00,1.00}{##1}}}
\expandafter\def\csname PY@tok@ne\endcsname{\let\PY@bf=\textbf\def\PY@tc##1{\textcolor[rgb]{0.82,0.25,0.23}{##1}}}
\expandafter\def\csname PY@tok@nv\endcsname{\def\PY@tc##1{\textcolor[rgb]{0.10,0.09,0.49}{##1}}}
\expandafter\def\csname PY@tok@no\endcsname{\def\PY@tc##1{\textcolor[rgb]{0.53,0.00,0.00}{##1}}}
\expandafter\def\csname PY@tok@nl\endcsname{\def\PY@tc##1{\textcolor[rgb]{0.63,0.63,0.00}{##1}}}
\expandafter\def\csname PY@tok@ni\endcsname{\let\PY@bf=\textbf\def\PY@tc##1{\textcolor[rgb]{0.60,0.60,0.60}{##1}}}
\expandafter\def\csname PY@tok@na\endcsname{\def\PY@tc##1{\textcolor[rgb]{0.49,0.56,0.16}{##1}}}
\expandafter\def\csname PY@tok@nt\endcsname{\let\PY@bf=\textbf\def\PY@tc##1{\textcolor[rgb]{0.00,0.50,0.00}{##1}}}
\expandafter\def\csname PY@tok@nd\endcsname{\def\PY@tc##1{\textcolor[rgb]{0.67,0.13,1.00}{##1}}}
\expandafter\def\csname PY@tok@s\endcsname{\def\PY@tc##1{\textcolor[rgb]{0.73,0.13,0.13}{##1}}}
\expandafter\def\csname PY@tok@sd\endcsname{\let\PY@it=\textit\def\PY@tc##1{\textcolor[rgb]{0.73,0.13,0.13}{##1}}}
\expandafter\def\csname PY@tok@si\endcsname{\let\PY@bf=\textbf\def\PY@tc##1{\textcolor[rgb]{0.73,0.40,0.53}{##1}}}
\expandafter\def\csname PY@tok@se\endcsname{\let\PY@bf=\textbf\def\PY@tc##1{\textcolor[rgb]{0.73,0.40,0.13}{##1}}}
\expandafter\def\csname PY@tok@sr\endcsname{\def\PY@tc##1{\textcolor[rgb]{0.73,0.40,0.53}{##1}}}
\expandafter\def\csname PY@tok@ss\endcsname{\def\PY@tc##1{\textcolor[rgb]{0.10,0.09,0.49}{##1}}}
\expandafter\def\csname PY@tok@sx\endcsname{\def\PY@tc##1{\textcolor[rgb]{0.00,0.50,0.00}{##1}}}
\expandafter\def\csname PY@tok@m\endcsname{\def\PY@tc##1{\textcolor[rgb]{0.40,0.40,0.40}{##1}}}
\expandafter\def\csname PY@tok@gh\endcsname{\let\PY@bf=\textbf\def\PY@tc##1{\textcolor[rgb]{0.00,0.00,0.50}{##1}}}
\expandafter\def\csname PY@tok@gu\endcsname{\let\PY@bf=\textbf\def\PY@tc##1{\textcolor[rgb]{0.50,0.00,0.50}{##1}}}
\expandafter\def\csname PY@tok@gd\endcsname{\def\PY@tc##1{\textcolor[rgb]{0.63,0.00,0.00}{##1}}}
\expandafter\def\csname PY@tok@gi\endcsname{\def\PY@tc##1{\textcolor[rgb]{0.00,0.63,0.00}{##1}}}
\expandafter\def\csname PY@tok@gr\endcsname{\def\PY@tc##1{\textcolor[rgb]{1.00,0.00,0.00}{##1}}}
\expandafter\def\csname PY@tok@ge\endcsname{\let\PY@it=\textit}
\expandafter\def\csname PY@tok@gs\endcsname{\let\PY@bf=\textbf}
\expandafter\def\csname PY@tok@gp\endcsname{\let\PY@bf=\textbf\def\PY@tc##1{\textcolor[rgb]{0.00,0.00,0.50}{##1}}}
\expandafter\def\csname PY@tok@go\endcsname{\def\PY@tc##1{\textcolor[rgb]{0.53,0.53,0.53}{##1}}}
\expandafter\def\csname PY@tok@gt\endcsname{\def\PY@tc##1{\textcolor[rgb]{0.00,0.27,0.87}{##1}}}
\expandafter\def\csname PY@tok@err\endcsname{\def\PY@bc##1{\setlength{\fboxsep}{0pt}\fcolorbox[rgb]{1.00,0.00,0.00}{1,1,1}{\strut ##1}}}
\expandafter\def\csname PY@tok@kc\endcsname{\let\PY@bf=\textbf\def\PY@tc##1{\textcolor[rgb]{0.00,0.50,0.00}{##1}}}
\expandafter\def\csname PY@tok@kd\endcsname{\let\PY@bf=\textbf\def\PY@tc##1{\textcolor[rgb]{0.00,0.50,0.00}{##1}}}
\expandafter\def\csname PY@tok@kn\endcsname{\let\PY@bf=\textbf\def\PY@tc##1{\textcolor[rgb]{0.00,0.50,0.00}{##1}}}
\expandafter\def\csname PY@tok@kr\endcsname{\let\PY@bf=\textbf\def\PY@tc##1{\textcolor[rgb]{0.00,0.50,0.00}{##1}}}
\expandafter\def\csname PY@tok@bp\endcsname{\def\PY@tc##1{\textcolor[rgb]{0.00,0.50,0.00}{##1}}}
\expandafter\def\csname PY@tok@fm\endcsname{\def\PY@tc##1{\textcolor[rgb]{0.00,0.00,1.00}{##1}}}
\expandafter\def\csname PY@tok@vc\endcsname{\def\PY@tc##1{\textcolor[rgb]{0.10,0.09,0.49}{##1}}}
\expandafter\def\csname PY@tok@vg\endcsname{\def\PY@tc##1{\textcolor[rgb]{0.10,0.09,0.49}{##1}}}
\expandafter\def\csname PY@tok@vi\endcsname{\def\PY@tc##1{\textcolor[rgb]{0.10,0.09,0.49}{##1}}}
\expandafter\def\csname PY@tok@vm\endcsname{\def\PY@tc##1{\textcolor[rgb]{0.10,0.09,0.49}{##1}}}
\expandafter\def\csname PY@tok@sa\endcsname{\def\PY@tc##1{\textcolor[rgb]{0.73,0.13,0.13}{##1}}}
\expandafter\def\csname PY@tok@sb\endcsname{\def\PY@tc##1{\textcolor[rgb]{0.73,0.13,0.13}{##1}}}
\expandafter\def\csname PY@tok@sc\endcsname{\def\PY@tc##1{\textcolor[rgb]{0.73,0.13,0.13}{##1}}}
\expandafter\def\csname PY@tok@dl\endcsname{\def\PY@tc##1{\textcolor[rgb]{0.73,0.13,0.13}{##1}}}
\expandafter\def\csname PY@tok@s2\endcsname{\def\PY@tc##1{\textcolor[rgb]{0.73,0.13,0.13}{##1}}}
\expandafter\def\csname PY@tok@sh\endcsname{\def\PY@tc##1{\textcolor[rgb]{0.73,0.13,0.13}{##1}}}
\expandafter\def\csname PY@tok@s1\endcsname{\def\PY@tc##1{\textcolor[rgb]{0.73,0.13,0.13}{##1}}}
\expandafter\def\csname PY@tok@mb\endcsname{\def\PY@tc##1{\textcolor[rgb]{0.40,0.40,0.40}{##1}}}
\expandafter\def\csname PY@tok@mf\endcsname{\def\PY@tc##1{\textcolor[rgb]{0.40,0.40,0.40}{##1}}}
\expandafter\def\csname PY@tok@mh\endcsname{\def\PY@tc##1{\textcolor[rgb]{0.40,0.40,0.40}{##1}}}
\expandafter\def\csname PY@tok@mi\endcsname{\def\PY@tc##1{\textcolor[rgb]{0.40,0.40,0.40}{##1}}}
\expandafter\def\csname PY@tok@il\endcsname{\def\PY@tc##1{\textcolor[rgb]{0.40,0.40,0.40}{##1}}}
\expandafter\def\csname PY@tok@mo\endcsname{\def\PY@tc##1{\textcolor[rgb]{0.40,0.40,0.40}{##1}}}
\expandafter\def\csname PY@tok@ch\endcsname{\let\PY@it=\textit\def\PY@tc##1{\textcolor[rgb]{0.25,0.50,0.50}{##1}}}
\expandafter\def\csname PY@tok@cm\endcsname{\let\PY@it=\textit\def\PY@tc##1{\textcolor[rgb]{0.25,0.50,0.50}{##1}}}
\expandafter\def\csname PY@tok@cpf\endcsname{\let\PY@it=\textit\def\PY@tc##1{\textcolor[rgb]{0.25,0.50,0.50}{##1}}}
\expandafter\def\csname PY@tok@c1\endcsname{\let\PY@it=\textit\def\PY@tc##1{\textcolor[rgb]{0.25,0.50,0.50}{##1}}}
\expandafter\def\csname PY@tok@cs\endcsname{\let\PY@it=\textit\def\PY@tc##1{\textcolor[rgb]{0.25,0.50,0.50}{##1}}}

\def\PYZbs{\char`\\}
\def\PYZus{\char`\_}
\def\PYZob{\char`\{}
\def\PYZcb{\char`\}}
\def\PYZca{\char`\^}
\def\PYZam{\char`\&}
\def\PYZlt{\char`\<}
\def\PYZgt{\char`\>}
\def\PYZsh{\char`\#}
\def\PYZpc{\char`\%}
\def\PYZdl{\char`\$}
\def\PYZhy{\char`\-}
\def\PYZsq{\char`\'}
\def\PYZdq{\char`\"}
\def\PYZti{\char`\~}
% for compatibility with earlier versions
\def\PYZat{@}
\def\PYZlb{[}
\def\PYZrb{]}
\makeatother


    % For linebreaks inside Verbatim environment from package fancyvrb. 
    \makeatletter
        \newbox\Wrappedcontinuationbox 
        \newbox\Wrappedvisiblespacebox 
        \newcommand*\Wrappedvisiblespace {\textcolor{red}{\textvisiblespace}} 
        \newcommand*\Wrappedcontinuationsymbol {\textcolor{red}{\llap{\tiny$\m@th\hookrightarrow$}}} 
        \newcommand*\Wrappedcontinuationindent {3ex } 
        \newcommand*\Wrappedafterbreak {\kern\Wrappedcontinuationindent\copy\Wrappedcontinuationbox} 
        % Take advantage of the already applied Pygments mark-up to insert 
        % potential linebreaks for TeX processing. 
        %        {, <, #, %, $, ' and ": go to next line. 
        %        _, }, ^, &, >, - and ~: stay at end of broken line. 
        % Use of \textquotesingle for straight quote. 
        \newcommand*\Wrappedbreaksatspecials {% 
            \def\PYGZus{\discretionary{\char`\_}{\Wrappedafterbreak}{\char`\_}}% 
            \def\PYGZob{\discretionary{}{\Wrappedafterbreak\char`\{}{\char`\{}}% 
            \def\PYGZcb{\discretionary{\char`\}}{\Wrappedafterbreak}{\char`\}}}% 
            \def\PYGZca{\discretionary{\char`\^}{\Wrappedafterbreak}{\char`\^}}% 
            \def\PYGZam{\discretionary{\char`\&}{\Wrappedafterbreak}{\char`\&}}% 
            \def\PYGZlt{\discretionary{}{\Wrappedafterbreak\char`\<}{\char`\<}}% 
            \def\PYGZgt{\discretionary{\char`\>}{\Wrappedafterbreak}{\char`\>}}% 
            \def\PYGZsh{\discretionary{}{\Wrappedafterbreak\char`\#}{\char`\#}}% 
            \def\PYGZpc{\discretionary{}{\Wrappedafterbreak\char`\%}{\char`\%}}% 
            \def\PYGZdl{\discretionary{}{\Wrappedafterbreak\char`\$}{\char`\$}}% 
            \def\PYGZhy{\discretionary{\char`\-}{\Wrappedafterbreak}{\char`\-}}% 
            \def\PYGZsq{\discretionary{}{\Wrappedafterbreak\textquotesingle}{\textquotesingle}}% 
            \def\PYGZdq{\discretionary{}{\Wrappedafterbreak\char`\"}{\char`\"}}% 
            \def\PYGZti{\discretionary{\char`\~}{\Wrappedafterbreak}{\char`\~}}% 
        } 
        % Some characters . , ; ? ! / are not pygmentized. 
        % This macro makes them "active" and they will insert potential linebreaks 
        \newcommand*\Wrappedbreaksatpunct {% 
            \lccode`\~`\.\lowercase{\def~}{\discretionary{\hbox{\char`\.}}{\Wrappedafterbreak}{\hbox{\char`\.}}}% 
            \lccode`\~`\,\lowercase{\def~}{\discretionary{\hbox{\char`\,}}{\Wrappedafterbreak}{\hbox{\char`\,}}}% 
            \lccode`\~`\;\lowercase{\def~}{\discretionary{\hbox{\char`\;}}{\Wrappedafterbreak}{\hbox{\char`\;}}}% 
            \lccode`\~`\:\lowercase{\def~}{\discretionary{\hbox{\char`\:}}{\Wrappedafterbreak}{\hbox{\char`\:}}}% 
            \lccode`\~`\?\lowercase{\def~}{\discretionary{\hbox{\char`\?}}{\Wrappedafterbreak}{\hbox{\char`\?}}}% 
            \lccode`\~`\!\lowercase{\def~}{\discretionary{\hbox{\char`\!}}{\Wrappedafterbreak}{\hbox{\char`\!}}}% 
            \lccode`\~`\/\lowercase{\def~}{\discretionary{\hbox{\char`\/}}{\Wrappedafterbreak}{\hbox{\char`\/}}}% 
            \catcode`\.\active
            \catcode`\,\active 
            \catcode`\;\active
            \catcode`\:\active
            \catcode`\?\active
            \catcode`\!\active
            \catcode`\/\active 
            \lccode`\~`\~ 	
        }
    \makeatother

    \let\OriginalVerbatim=\Verbatim
    \makeatletter
    \renewcommand{\Verbatim}[1][1]{%
        %\parskip\z@skip
        \sbox\Wrappedcontinuationbox {\Wrappedcontinuationsymbol}%
        \sbox\Wrappedvisiblespacebox {\FV@SetupFont\Wrappedvisiblespace}%
        \def\FancyVerbFormatLine ##1{\hsize\linewidth
            \vtop{\raggedright\hyphenpenalty\z@\exhyphenpenalty\z@
                \doublehyphendemerits\z@\finalhyphendemerits\z@
                \strut ##1\strut}%
        }%
        % If the linebreak is at a space, the latter will be displayed as visible
        % space at end of first line, and a continuation symbol starts next line.
        % Stretch/shrink are however usually zero for typewriter font.
        \def\FV@Space {%
            \nobreak\hskip\z@ plus\fontdimen3\font minus\fontdimen4\font
            \discretionary{\copy\Wrappedvisiblespacebox}{\Wrappedafterbreak}
            {\kern\fontdimen2\font}%
        }%
        
        % Allow breaks at special characters using \PYG... macros.
        \Wrappedbreaksatspecials
        % Breaks at punctuation characters . , ; ? ! and / need catcode=\active 	
        \OriginalVerbatim[#1,codes*=\Wrappedbreaksatpunct]%
    }
    \makeatother

    % Exact colors from NB
    \definecolor{incolor}{HTML}{303F9F}
    \definecolor{outcolor}{HTML}{D84315}
    \definecolor{cellborder}{HTML}{CFCFCF}
    \definecolor{cellbackground}{HTML}{F7F7F7}
    
    % prompt
    \makeatletter
    \newcommand{\boxspacing}{\kern\kvtcb@left@rule\kern\kvtcb@boxsep}
    \makeatother
    \newcommand{\prompt}[4]{
        \ttfamily\llap{{\color{#2}[#3]:\hspace{3pt}#4}}\vspace{-\baselineskip}
    }
    

    
    % Prevent overflowing lines due to hard-to-break entities
    \sloppy 
    % Setup hyperref package
    \hypersetup{
      breaklinks=true,  % so long urls are correctly broken across lines
      colorlinks=true,
      urlcolor=urlcolor,
      linkcolor=linkcolor,
      citecolor=citecolor,
      }
    % Slightly bigger margins than the latex defaults
    
    \geometry{verbose,tmargin=1in,bmargin=1in,lmargin=1in,rmargin=1in}
    
    

\begin{document}
    
    \maketitle
    
    

    
    \hypertarget{decisiontreeclassifier}{%
\section{DecisionTreeClassifier}\label{decisiontreeclassifier}}

© Thomas Robert Holy 2019 Version 1.0 Visit me on GitHub:
https://github.com/trh0ly Kaggle Link:
https://www.kaggle.com/c/dda-p2/leaderboard

\hypertarget{package-import}{%
\subsection{Package Import}\label{package-import}}

    \begin{tcolorbox}[breakable, size=fbox, boxrule=1pt, pad at break*=1mm,colback=cellbackground, colframe=cellborder]
\prompt{In}{incolor}{1}{\boxspacing}
\begin{Verbatim}[commandchars=\\\{\}]
\PY{k+kn}{import} \PY{n+nn}{numpy} \PY{k}{as} \PY{n+nn}{np}
\PY{k+kn}{import} \PY{n+nn}{pandas} \PY{k}{as} \PY{n+nn}{pd}
\PY{k+kn}{from} \PY{n+nn}{sklearn}\PY{n+nn}{.}\PY{n+nn}{pipeline} \PY{k+kn}{import} \PY{n}{Pipeline}
\PY{k+kn}{from} \PY{n+nn}{sklearn}\PY{n+nn}{.}\PY{n+nn}{linear\PYZus{}model} \PY{k+kn}{import} \PY{n}{LogisticRegression}
\PY{k+kn}{from} \PY{n+nn}{sklearn}\PY{n+nn}{.}\PY{n+nn}{preprocessing} \PY{k+kn}{import} \PY{n}{StandardScaler}
\PY{k+kn}{from} \PY{n+nn}{sklearn}\PY{n+nn}{.}\PY{n+nn}{preprocessing} \PY{k+kn}{import} \PY{n}{RobustScaler}
\PY{k+kn}{from} \PY{n+nn}{sklearn}\PY{n+nn}{.}\PY{n+nn}{preprocessing} \PY{k+kn}{import} \PY{n}{MinMaxScaler}
\PY{k+kn}{from} \PY{n+nn}{sklearn}\PY{n+nn}{.}\PY{n+nn}{metrics} \PY{k+kn}{import} \PY{n}{confusion\PYZus{}matrix}
\PY{k+kn}{from} \PY{n+nn}{sklearn}\PY{n+nn}{.}\PY{n+nn}{metrics} \PY{k+kn}{import} \PY{n}{classification\PYZus{}report}
\PY{k+kn}{from} \PY{n+nn}{sklearn}\PY{n+nn}{.}\PY{n+nn}{metrics} \PY{k+kn}{import} \PY{n}{roc\PYZus{}curve}\PY{p}{,} \PY{n}{roc\PYZus{}auc\PYZus{}score}
\PY{k+kn}{from} \PY{n+nn}{sklearn}\PY{n+nn}{.}\PY{n+nn}{metrics} \PY{k+kn}{import} \PY{n}{auc}
\PY{k+kn}{from} \PY{n+nn}{sklearn}\PY{n+nn}{.}\PY{n+nn}{ensemble} \PY{k+kn}{import} \PY{n}{RandomForestClassifier}
\PY{k+kn}{from} \PY{n+nn}{sklearn}\PY{n+nn}{.}\PY{n+nn}{neighbors} \PY{k+kn}{import} \PY{n}{KNeighborsClassifier}
\PY{k+kn}{from} \PY{n+nn}{sklearn}\PY{n+nn}{.}\PY{n+nn}{tree} \PY{k+kn}{import} \PY{n}{DecisionTreeClassifier}
\PY{k+kn}{from} \PY{n+nn}{sklearn}\PY{n+nn}{.}\PY{n+nn}{preprocessing} \PY{k+kn}{import} \PY{n}{PolynomialFeatures}
\PY{k+kn}{from} \PY{n+nn}{sklearn}\PY{n+nn}{.}\PY{n+nn}{model\PYZus{}selection} \PY{k+kn}{import} \PY{n}{GridSearchCV}
\PY{k+kn}{from} \PY{n+nn}{sklearn}\PY{n+nn}{.}\PY{n+nn}{model\PYZus{}selection} \PY{k+kn}{import} \PY{n}{StratifiedKFold}
\PY{k+kn}{from} \PY{n+nn}{sklearn}\PY{n+nn}{.}\PY{n+nn}{model\PYZus{}selection} \PY{k+kn}{import} \PY{n}{cross\PYZus{}val\PYZus{}score}
\PY{k+kn}{from} \PY{n+nn}{sklearn}\PY{n+nn}{.}\PY{n+nn}{model\PYZus{}selection} \PY{k+kn}{import} \PY{n}{train\PYZus{}test\PYZus{}split}
\PY{k+kn}{import} \PY{n+nn}{matplotlib}\PY{n+nn}{.}\PY{n+nn}{pyplot} \PY{k}{as} \PY{n+nn}{plt}
\PY{k+kn}{import} \PY{n+nn}{datetime} \PY{k}{as} \PY{n+nn}{dt}
\PY{k+kn}{from} \PY{n+nn}{IPython}\PY{n+nn}{.}\PY{n+nn}{core}\PY{n+nn}{.}\PY{n+nn}{display} \PY{k+kn}{import} \PY{n}{display}\PY{p}{,} \PY{n}{HTML}
\PY{k+kn}{from} \PY{n+nn}{scipy}\PY{n+nn}{.}\PY{n+nn}{spatial}\PY{n+nn}{.}\PY{n+nn}{distance} \PY{k+kn}{import} \PY{n}{euclidean}
\PY{k+kn}{from} \PY{n+nn}{sklearn}\PY{n+nn}{.}\PY{n+nn}{metrics}\PY{n+nn}{.}\PY{n+nn}{pairwise} \PY{k+kn}{import} \PY{n}{manhattan\PYZus{}distances}
\PY{k+kn}{from} \PY{n+nn}{sklearn}\PY{n+nn}{.}\PY{n+nn}{svm} \PY{k+kn}{import} \PY{n}{SVC} 
\end{Verbatim}
\end{tcolorbox}

    \hypertarget{hilfsfunktionen}{%
\subsection{Hilfsfunktionen}\label{hilfsfunktionen}}

\hypertarget{funktion-zur-betrachtung-der-konfusinsmatrix}{%
\subsubsection{Funktion zur Betrachtung der
Konfusinsmatrix}\label{funktion-zur-betrachtung-der-konfusinsmatrix}}

    \begin{tcolorbox}[breakable, size=fbox, boxrule=1pt, pad at break*=1mm,colback=cellbackground, colframe=cellborder]
\prompt{In}{incolor}{2}{\boxspacing}
\begin{Verbatim}[commandchars=\\\{\}]
\PY{c+c1}{\PYZsh{} Definition einer Funktion, welche eine Konfusionsmatrix und einen Klassifikationsreport}
\PY{c+c1}{\PYZsh{} zurückgibt. Die Konfusionsmatrix kann, wenn ein Wert für c gegeben ist, für beliebige }
\PY{c+c1}{\PYZsh{} Werte von c betrachtet werden.}
\PY{c+c1}{\PYZsh{}\PYZhy{}\PYZhy{}\PYZhy{}\PYZhy{}\PYZhy{}\PYZhy{}\PYZhy{}\PYZhy{}\PYZhy{}\PYZhy{}\PYZhy{}\PYZhy{}}
\PY{c+c1}{\PYZsh{} Argumente:}
\PY{c+c1}{\PYZsh{} \PYZhy{} X: DataFrame auf welchem die Prognose durchgefürt werden soll (ohne die Zielgröße)}
\PY{c+c1}{\PYZsh{} \PYZhy{} y\PYZus{}true: Zum DataFrame X gehörige Werte der Zielgröße}
\PY{c+c1}{\PYZsh{} \PYZhy{} model: Modell auf Basis dessen die Konfusionsmatrix berechnet werden soll}
\PY{c+c1}{\PYZsh{} \PYZhy{} class\PYZus{}names: Bezeichnung für die Spalten des Dataframes (default=[\PYZsq{}0\PYZsq{}, \PYZsq{}1\PYZsq{}], mit 0 = negativ und 1 = positiv)}
\PY{c+c1}{\PYZsh{} \PYZhy{} c:}
\PY{c+c1}{\PYZsh{} \PYZhy{}\PYZhy{}\PYZhy{}\PYZgt{} Wenn None, dann wird die Konfusionsmatrix ohne die Einbeziehung von c bestimmt}
\PY{c+c1}{\PYZsh{} \PYZhy{}\PYZhy{}\PYZhy{}\PYZgt{} Wenn != None, dann wird die Konfusionsmatrix in Abhängigkeit von c bestimmt}
\PY{c+c1}{\PYZsh{}\PYZhy{}\PYZhy{}\PYZhy{}\PYZhy{}\PYZhy{}\PYZhy{}\PYZhy{}\PYZhy{}\PYZhy{}\PYZhy{}\PYZhy{}\PYZhy{}}

\PY{k}{def} \PY{n+nf}{get\PYZus{}confusion\PYZus{}matrix}\PY{p}{(}\PY{n}{X}\PY{p}{,} \PY{n}{y\PYZus{}true}\PY{p}{,} \PY{n}{model}\PY{p}{,} \PY{n}{class\PYZus{}names}\PY{o}{=}\PY{p}{[}\PY{l+s+s1}{\PYZsq{}}\PY{l+s+s1}{0}\PY{l+s+s1}{\PYZsq{}}\PY{p}{,} \PY{l+s+s1}{\PYZsq{}}\PY{l+s+s1}{1}\PY{l+s+s1}{\PYZsq{}}\PY{p}{]}\PY{p}{,} \PY{n}{c}\PY{o}{=}\PY{k+kc}{None}\PY{p}{)}\PY{p}{:}
    
    \PY{c+c1}{\PYZsh{}\PYZhy{}\PYZhy{}\PYZhy{}\PYZhy{}\PYZhy{}\PYZhy{}\PYZhy{}\PYZhy{}\PYZhy{}\PYZhy{}\PYZhy{}\PYZhy{}\PYZhy{}\PYZhy{}\PYZhy{}\PYZhy{}\PYZhy{}\PYZhy{}\PYZhy{}\PYZhy{}\PYZhy{}\PYZhy{}\PYZhy{}\PYZhy{}\PYZhy{}\PYZhy{}\PYZhy{}\PYZhy{}}
    \PY{c+c1}{\PYZsh{} Vorgelagerte Berechnung falls ein Wert für c gegeben ist}
    \PY{c+c1}{\PYZsh{} und die Konfusionsmatrix für ein gegebenes c anpasst}
    \PY{k}{if} \PY{n}{c} \PY{o}{!=} \PY{k+kc}{None}\PY{p}{:}
        \PY{n}{pred\PYZus{}probability} \PY{o}{=} \PY{n}{model}\PY{o}{.}\PY{n}{predict\PYZus{}proba}\PY{p}{(}\PY{n}{X}\PY{p}{)}
        \PY{n}{pred\PYZus{}probability} \PY{o}{=} \PY{n}{pred\PYZus{}probability} \PY{o}{\PYZgt{}}\PY{o}{=} \PY{n}{c}
        \PY{n}{y\PYZus{}pred} \PY{o}{=} \PY{n}{pred\PYZus{}probability}\PY{p}{[}\PY{p}{:}\PY{p}{,} \PY{l+m+mi}{1}\PY{p}{]}\PY{o}{.}\PY{n}{astype}\PY{p}{(}\PY{n+nb}{int}\PY{p}{)}
    
    \PY{c+c1}{\PYZsh{}\PYZhy{}\PYZhy{}\PYZhy{}\PYZhy{}\PYZhy{}\PYZhy{}\PYZhy{}\PYZhy{}\PYZhy{}\PYZhy{}\PYZhy{}\PYZhy{}\PYZhy{}\PYZhy{}\PYZhy{}\PYZhy{}\PYZhy{}\PYZhy{}\PYZhy{}\PYZhy{}\PYZhy{}\PYZhy{}\PYZhy{}\PYZhy{}\PYZhy{}\PYZhy{}\PYZhy{}\PYZhy{}}
    \PY{c+c1}{\PYZsh{} Wenn kein Wert für c gegeben, dann führe Prognose }
    \PY{c+c1}{\PYZsh{} lediglich auf Basis des Modells durch}
    \PY{k}{if} \PY{n}{c} \PY{o}{==} \PY{k+kc}{None}\PY{p}{:}
        \PY{n}{y\PYZus{}pred} \PY{o}{=} \PY{n}{model}\PY{o}{.}\PY{n}{predict}\PY{p}{(}\PY{n}{X}\PY{p}{)}
    
    \PY{c+c1}{\PYZsh{}\PYZhy{}\PYZhy{}\PYZhy{}\PYZhy{}\PYZhy{}\PYZhy{}\PYZhy{}\PYZhy{}\PYZhy{}\PYZhy{}\PYZhy{}\PYZhy{}\PYZhy{}\PYZhy{}\PYZhy{}\PYZhy{}\PYZhy{}\PYZhy{}\PYZhy{}\PYZhy{}\PYZhy{}\PYZhy{}\PYZhy{}\PYZhy{}\PYZhy{}\PYZhy{}\PYZhy{}\PYZhy{}}
    \PY{c+c1}{\PYZsh{} Berechnet die Konfusionsmatrix}
    \PY{n}{conf\PYZus{}mat} \PY{o}{=} \PY{n}{confusion\PYZus{}matrix}\PY{p}{(}\PY{n}{y\PYZus{}true}\PY{p}{,} \PY{n}{y\PYZus{}pred}\PY{p}{)}
    
    \PY{c+c1}{\PYZsh{}\PYZhy{}\PYZhy{}\PYZhy{}\PYZhy{}\PYZhy{}\PYZhy{}\PYZhy{}\PYZhy{}\PYZhy{}\PYZhy{}\PYZhy{}\PYZhy{}\PYZhy{}\PYZhy{}\PYZhy{}\PYZhy{}\PYZhy{}\PYZhy{}\PYZhy{}\PYZhy{}\PYZhy{}\PYZhy{}\PYZhy{}\PYZhy{}\PYZhy{}\PYZhy{}\PYZhy{}\PYZhy{}}
    \PY{c+c1}{\PYZsh{} Überführung in einen DataFrame für eine bessere Übersichtlichkeit}
    \PY{n}{df\PYZus{}index} \PY{o}{=} \PY{n}{pd}\PY{o}{.}\PY{n}{MultiIndex}\PY{o}{.}\PY{n}{from\PYZus{}tuples}\PY{p}{(}\PY{p}{[}\PY{p}{(}\PY{l+s+s1}{\PYZsq{}}\PY{l+s+s1}{Wahrer Wert}\PY{l+s+s1}{\PYZsq{}}\PY{p}{,} \PY{n}{cn}\PY{p}{)} \PY{k}{for} \PY{n}{cn} \PY{o+ow}{in} \PY{n}{class\PYZus{}names}\PY{p}{]}\PY{p}{)}
    \PY{n}{df\PYZus{}cols} \PY{o}{=} \PY{n}{pd}\PY{o}{.}\PY{n}{MultiIndex}\PY{o}{.}\PY{n}{from\PYZus{}tuples}\PY{p}{(}\PY{p}{[}\PY{p}{(}\PY{l+s+s1}{\PYZsq{}}\PY{l+s+s1}{Prognose des Modells}\PY{l+s+s1}{\PYZsq{}}\PY{p}{,} \PY{n}{cn}\PY{p}{)} \PY{k}{for} \PY{n}{cn} \PY{o+ow}{in} \PY{n}{class\PYZus{}names}\PY{p}{]}\PY{p}{)}
    \PY{n}{df\PYZus{}conf\PYZus{}mat} \PY{o}{=} \PY{n}{pd}\PY{o}{.}\PY{n}{DataFrame}\PY{p}{(}\PY{n}{conf\PYZus{}mat}\PY{p}{,} \PY{n}{index}\PY{o}{=}\PY{n}{df\PYZus{}index}\PY{p}{,} \PY{n}{columns}\PY{o}{=}\PY{n}{df\PYZus{}cols}\PY{p}{)}

    \PY{k}{return} \PY{n}{df\PYZus{}conf\PYZus{}mat}\PY{p}{,} \PY{n}{classification\PYZus{}report}\PY{p}{(}\PY{n}{y\PYZus{}true}\PY{p}{,} \PY{n}{y\PYZus{}pred}\PY{p}{)}
\end{Verbatim}
\end{tcolorbox}

    \hypertarget{funktion-zur-betrachtung-der-roc-kurve}{%
\subsubsection{Funktion zur Betrachtung der
ROC-Kurve}\label{funktion-zur-betrachtung-der-roc-kurve}}

    \begin{tcolorbox}[breakable, size=fbox, boxrule=1pt, pad at break*=1mm,colback=cellbackground, colframe=cellborder]
\prompt{In}{incolor}{3}{\boxspacing}
\begin{Verbatim}[commandchars=\\\{\}]
\PY{c+c1}{\PYZsh{} Definition einer Funktion, welche auf Basis eines gegeben Modells und zweier zusammengehöriger}
\PY{c+c1}{\PYZsh{} DataFrames die receiver operating characteristic curve (ROC\PYZhy{}Curve) visualisiert}
\PY{c+c1}{\PYZsh{}\PYZhy{}\PYZhy{}\PYZhy{}\PYZhy{}\PYZhy{}\PYZhy{}\PYZhy{}\PYZhy{}\PYZhy{}\PYZhy{}\PYZhy{}\PYZhy{}}
\PY{c+c1}{\PYZsh{} Argumente:}
\PY{c+c1}{\PYZsh{} \PYZhy{} X: DataFrame auf welchem die Prognose durchgefürt werden soll (ohne die Zielgröße)}
\PY{c+c1}{\PYZsh{} \PYZhy{} y\PYZus{}true: Zum DataFrame X gehörige Werte der Zielgröße}
\PY{c+c1}{\PYZsh{} \PYZhy{} model: Modell auf Basis dessen die ROC\PYZhy{}Curve berechnet werden soll}
\PY{c+c1}{\PYZsh{}\PYZhy{}\PYZhy{}\PYZhy{}\PYZhy{}\PYZhy{}\PYZhy{}\PYZhy{}\PYZhy{}\PYZhy{}\PYZhy{}\PYZhy{}\PYZhy{}}

\PY{k}{def} \PY{n+nf}{roc\PYZus{}curve\PYZus{}func}\PY{p}{(}\PY{n}{X}\PY{p}{,} \PY{n}{y\PYZus{}true}\PY{p}{,} \PY{n}{model}\PY{p}{)}\PY{p}{:}    
        
        \PY{c+c1}{\PYZsh{}\PYZhy{}\PYZhy{}\PYZhy{}\PYZhy{}\PYZhy{}\PYZhy{}\PYZhy{}\PYZhy{}\PYZhy{}\PYZhy{}\PYZhy{}\PYZhy{}\PYZhy{}\PYZhy{}\PYZhy{}\PYZhy{}\PYZhy{}\PYZhy{}\PYZhy{}\PYZhy{}\PYZhy{}\PYZhy{}\PYZhy{}\PYZhy{}\PYZhy{}\PYZhy{}\PYZhy{}\PYZhy{}}
        \PY{c+c1}{\PYZsh{} Berechnung FPR, TPR und AUC auf Basis des Modells  }
        \PY{n}{y\PYZus{}score} \PY{o}{=} \PY{n}{model}\PY{o}{.}\PY{n}{predict\PYZus{}proba}\PY{p}{(}\PY{n}{X}\PY{p}{)}\PY{p}{[}\PY{p}{:}\PY{p}{,}\PY{l+m+mi}{1}\PY{p}{]}            
        \PY{n}{FPR}\PY{p}{,} \PY{n}{TPR}\PY{p}{,} \PY{n}{\PYZus{}} \PY{o}{=} \PY{n}{roc\PYZus{}curve}\PY{p}{(}\PY{n}{y\PYZus{}true}\PY{p}{,} \PY{n}{y\PYZus{}score}\PY{p}{)}
        \PY{n}{AUC} \PY{o}{=} \PY{n}{auc}\PY{p}{(}\PY{n}{FPR}\PY{p}{,} \PY{n}{TPR}\PY{p}{)}            
            
        \PY{c+c1}{\PYZsh{}\PYZhy{}\PYZhy{}\PYZhy{}\PYZhy{}\PYZhy{}\PYZhy{}\PYZhy{}\PYZhy{}\PYZhy{}\PYZhy{}\PYZhy{}\PYZhy{}\PYZhy{}\PYZhy{}\PYZhy{}\PYZhy{}\PYZhy{}\PYZhy{}\PYZhy{}\PYZhy{}\PYZhy{}\PYZhy{}\PYZhy{}\PYZhy{}\PYZhy{}\PYZhy{}\PYZhy{}\PYZhy{}}
        \PY{c+c1}{\PYZsh{} Darstellung als Grafik}
        \PY{n}{plt}\PY{o}{.}\PY{n}{figure}\PY{p}{(}\PY{p}{)}
        \PY{n}{plt}\PY{o}{.}\PY{n}{plot}\PY{p}{(}\PY{n}{FPR}\PY{p}{,} \PY{n}{TPR}\PY{p}{,} \PY{n}{color}\PY{o}{=}\PY{l+s+s1}{\PYZsq{}}\PY{l+s+s1}{red}\PY{l+s+s1}{\PYZsq{}}\PY{p}{,} \PY{n}{lw}\PY{o}{=}\PY{l+m+mi}{2}\PY{p}{,} \PY{n}{label}\PY{o}{=}\PY{l+s+s1}{\PYZsq{}}\PY{l+s+s1}{ROC\PYZhy{}Kurve (AUC = }\PY{l+s+si}{\PYZpc{}0.5f}\PY{l+s+s1}{)}\PY{l+s+s1}{\PYZsq{}} \PY{o}{\PYZpc{}} \PY{n}{AUC}\PY{p}{)}
        \PY{n}{plt}\PY{o}{.}\PY{n}{plot}\PY{p}{(}\PY{p}{[}\PY{l+m+mi}{0}\PY{p}{,} \PY{l+m+mi}{1}\PY{p}{]}\PY{p}{,} \PY{p}{[}\PY{l+m+mi}{0}\PY{p}{,} \PY{l+m+mi}{1}\PY{p}{]}\PY{p}{,} \PY{n}{color}\PY{o}{=}\PY{l+s+s1}{\PYZsq{}}\PY{l+s+s1}{black}\PY{l+s+s1}{\PYZsq{}}\PY{p}{,} \PY{n}{lw}\PY{o}{=}\PY{l+m+mi}{2}\PY{p}{,} \PY{n}{linestyle}\PY{o}{=}\PY{l+s+s1}{\PYZsq{}}\PY{l+s+s1}{\PYZhy{}\PYZhy{}}\PY{l+s+s1}{\PYZsq{}}\PY{p}{)}
        \PY{n}{plt}\PY{o}{.}\PY{n}{xlim}\PY{p}{(}\PY{p}{[}\PY{o}{\PYZhy{}}\PY{l+m+mf}{0.005}\PY{p}{,} \PY{l+m+mf}{1.0}\PY{p}{]}\PY{p}{)}
        \PY{n}{plt}\PY{o}{.}\PY{n}{ylim}\PY{p}{(}\PY{p}{[}\PY{l+m+mf}{0.0}\PY{p}{,} \PY{l+m+mf}{1.05}\PY{p}{]}\PY{p}{)}
        \PY{n}{plt}\PY{o}{.}\PY{n}{xlabel}\PY{p}{(}\PY{l+s+s1}{\PYZsq{}}\PY{l+s+s1}{False Positive Rate}\PY{l+s+s1}{\PYZsq{}}\PY{p}{)}
        \PY{n}{plt}\PY{o}{.}\PY{n}{ylabel}\PY{p}{(}\PY{l+s+s1}{\PYZsq{}}\PY{l+s+s1}{True Positive Rate}\PY{l+s+s1}{\PYZsq{}}\PY{p}{)}
        \PY{n}{plt}\PY{o}{.}\PY{n}{title}\PY{p}{(}\PY{l+s+s1}{\PYZsq{}}\PY{l+s+s1}{ROC\PYZhy{}Kurve}\PY{l+s+s1}{\PYZsq{}}\PY{p}{)}
        \PY{n}{plt}\PY{o}{.}\PY{n}{legend}\PY{p}{(}\PY{n}{loc}\PY{o}{=}\PY{l+s+s2}{\PYZdq{}}\PY{l+s+s2}{lower right}\PY{l+s+s2}{\PYZdq{}}\PY{p}{)}
        \PY{n}{plt}\PY{o}{.}\PY{n}{show}\PY{p}{(}\PY{p}{)}
\end{Verbatim}
\end{tcolorbox}

    \hypertarget{funktion-fuxfcr-das-submitten}{%
\subsubsection{Funktion für das
Submitten}\label{funktion-fuxfcr-das-submitten}}

    \begin{tcolorbox}[breakable, size=fbox, boxrule=1pt, pad at break*=1mm,colback=cellbackground, colframe=cellborder]
\prompt{In}{incolor}{4}{\boxspacing}
\begin{Verbatim}[commandchars=\\\{\}]
\PY{c+c1}{\PYZsh{} Definition einer Funktion, welche das Submitten der Prognose auf dem Testdatensatz erleichert}
\PY{c+c1}{\PYZsh{}\PYZhy{}\PYZhy{}\PYZhy{}\PYZhy{}\PYZhy{}\PYZhy{}\PYZhy{}\PYZhy{}\PYZhy{}\PYZhy{}\PYZhy{}\PYZhy{}}
\PY{c+c1}{\PYZsh{} Argumente:}
\PY{c+c1}{\PYZsh{} \PYZhy{} model: Modell auf Basis dessen die Prognose erfolgt}
\PY{c+c1}{\PYZsh{} \PYZhy{} c:}
\PY{c+c1}{\PYZsh{} \PYZhy{}\PYZhy{}\PYZhy{}\PYZgt{} Wenn None, dann wird die Prognose ohne die Berücksichtung von c vorgenommen}
\PY{c+c1}{\PYZsh{} \PYZhy{}\PYZhy{}\PYZhy{}\PYZgt{} Wenn != None, dann wird Prognose mit der Berücksichtung von c vorgenommen}
\PY{c+c1}{\PYZsh{} \PYZhy{} save:}
\PY{c+c1}{\PYZsh{} \PYZhy{}\PYZhy{}\PYZhy{}\PYZgt{} Wenn False, dann werden die prognostizierten Daten nicht gespeichert}
\PY{c+c1}{\PYZsh{} \PYZhy{}\PYZhy{}\PYZhy{}\PYZgt{} Wenn True, dann werden die prognostizierten Daten als .csv gespeichert}
\PY{c+c1}{\PYZsh{} \PYZhy{} manu\PYZus{}name:}
\PY{c+c1}{\PYZsh{} \PYZhy{}\PYZhy{}\PYZhy{}\PYZgt{} Wenn None, dann wird ein nicht eindeutiger Standardname als Bezeichnung der .csv gewählt}
\PY{c+c1}{\PYZsh{} \PYZhy{}\PYZhy{}\PYZhy{}\PYZgt{} Wenn != None, dann wird die zu speichernde .csv mit einem timestamp versehen}
\PY{c+c1}{\PYZsh{}\PYZhy{}\PYZhy{}\PYZhy{}\PYZhy{}\PYZhy{}\PYZhy{}\PYZhy{}\PYZhy{}\PYZhy{}\PYZhy{}\PYZhy{}\PYZhy{}}

\PY{k}{def} \PY{n+nf}{submit}\PY{p}{(}\PY{n}{model}\PY{p}{,} \PY{n}{c}\PY{o}{=}\PY{k+kc}{None}\PY{p}{,} \PY{n}{save}\PY{o}{=}\PY{k+kc}{False}\PY{p}{,} \PY{n}{manu\PYZus{}name}\PY{o}{=}\PY{k+kc}{False}\PY{p}{)}\PY{p}{:}
    
    \PY{c+c1}{\PYZsh{}\PYZhy{}\PYZhy{}\PYZhy{}\PYZhy{}\PYZhy{}\PYZhy{}\PYZhy{}\PYZhy{}\PYZhy{}\PYZhy{}\PYZhy{}\PYZhy{}\PYZhy{}\PYZhy{}\PYZhy{}\PYZhy{}\PYZhy{}\PYZhy{}\PYZhy{}\PYZhy{}\PYZhy{}\PYZhy{}\PYZhy{}\PYZhy{}\PYZhy{}\PYZhy{}\PYZhy{}\PYZhy{}\PYZhy{}\PYZhy{}\PYZhy{}\PYZhy{}}
    \PY{c+c1}{\PYZsh{} Testdatensatz einlesen}
    \PY{n}{X\PYZus{}test} \PY{o}{=} \PY{n}{pd}\PY{o}{.}\PY{n}{read\PYZus{}csv}\PY{p}{(}\PY{l+s+s1}{\PYZsq{}}\PY{l+s+s1}{test.csv}\PY{l+s+s1}{\PYZsq{}}\PY{p}{,} \PY{n}{index\PYZus{}col}\PY{o}{=}\PY{l+m+mi}{0}\PY{p}{)}    
    
    \PY{c+c1}{\PYZsh{}\PYZhy{}\PYZhy{}\PYZhy{}\PYZhy{}\PYZhy{}\PYZhy{}\PYZhy{}\PYZhy{}\PYZhy{}\PYZhy{}\PYZhy{}\PYZhy{}\PYZhy{}\PYZhy{}\PYZhy{}\PYZhy{}\PYZhy{}\PYZhy{}\PYZhy{}\PYZhy{}\PYZhy{}\PYZhy{}\PYZhy{}\PYZhy{}\PYZhy{}\PYZhy{}\PYZhy{}\PYZhy{}\PYZhy{}\PYZhy{}\PYZhy{}\PYZhy{}}
    \PY{c+c1}{\PYZsh{} Prognosewerte auf Sensordaten des Testdatensatzes und unter}
    \PY{c+c1}{\PYZsh{} Berücksichtung von c erzeugen}
    \PY{k}{if} \PY{n}{c} \PY{o}{!=} \PY{k+kc}{None}\PY{p}{:}
        \PY{n}{predicted\PYZus{}test} \PY{o}{=} \PY{p}{(}\PY{n}{model}\PY{o}{.}\PY{n}{predict\PYZus{}proba}\PY{p}{(}\PY{n}{X\PYZus{}test}\PY{p}{)} \PY{o}{\PYZgt{}}\PY{o}{=} \PY{n}{c}\PY{p}{)}\PY{p}{[}\PY{p}{:}\PY{p}{,}\PY{l+m+mi}{1}\PY{p}{]}\PY{o}{.}\PY{n}{astype}\PY{p}{(}\PY{n+nb}{int}\PY{p}{)}
    
    \PY{c+c1}{\PYZsh{}\PYZhy{}\PYZhy{}\PYZhy{}\PYZhy{}\PYZhy{}\PYZhy{}\PYZhy{}\PYZhy{}\PYZhy{}\PYZhy{}\PYZhy{}\PYZhy{}\PYZhy{}\PYZhy{}\PYZhy{}\PYZhy{}\PYZhy{}\PYZhy{}\PYZhy{}\PYZhy{}\PYZhy{}\PYZhy{}\PYZhy{}\PYZhy{}\PYZhy{}\PYZhy{}\PYZhy{}\PYZhy{}\PYZhy{}\PYZhy{}\PYZhy{}\PYZhy{}}
    \PY{c+c1}{\PYZsh{} Prognosewerte auf Sensordaten des Testdatensatzes erzeugen}
    \PY{c+c1}{\PYZsh{} ohne c zu Berücksichtigen}
    \PY{k}{if} \PY{n}{c} \PY{o}{==} \PY{k+kc}{None}\PY{p}{:}
        \PY{n}{predicted\PYZus{}test} \PY{o}{=} \PY{n}{model}\PY{o}{.}\PY{n}{predict}\PY{p}{(}\PY{n}{X\PYZus{}test}\PY{p}{)}
        
    \PY{c+c1}{\PYZsh{}\PYZhy{}\PYZhy{}\PYZhy{}\PYZhy{}\PYZhy{}\PYZhy{}\PYZhy{}\PYZhy{}\PYZhy{}\PYZhy{}\PYZhy{}\PYZhy{}\PYZhy{}\PYZhy{}\PYZhy{}\PYZhy{}\PYZhy{}\PYZhy{}\PYZhy{}\PYZhy{}\PYZhy{}\PYZhy{}\PYZhy{}\PYZhy{}\PYZhy{}\PYZhy{}\PYZhy{}\PYZhy{}\PYZhy{}\PYZhy{}\PYZhy{}\PYZhy{}}
    \PY{c+c1}{\PYZsh{} Submissiondatensatz einlesen und prognostizierte Werte hineinschreiben}
    \PY{n}{submission} \PY{o}{=} \PY{n}{pd}\PY{o}{.}\PY{n}{read\PYZus{}csv}\PY{p}{(}\PY{l+s+s1}{\PYZsq{}}\PY{l+s+s1}{sample\PYZus{}submission.csv}\PY{l+s+s1}{\PYZsq{}}\PY{p}{)}
    \PY{n}{submission}\PY{p}{[}\PY{l+s+s1}{\PYZsq{}}\PY{l+s+s1}{Fehlerhaft}\PY{l+s+s1}{\PYZsq{}}\PY{p}{]} \PY{o}{=} \PY{n}{predicted\PYZus{}test}
    
    \PY{c+c1}{\PYZsh{}\PYZhy{}\PYZhy{}\PYZhy{}\PYZhy{}\PYZhy{}\PYZhy{}\PYZhy{}\PYZhy{}\PYZhy{}\PYZhy{}\PYZhy{}\PYZhy{}\PYZhy{}\PYZhy{}\PYZhy{}\PYZhy{}\PYZhy{}\PYZhy{}\PYZhy{}\PYZhy{}\PYZhy{}\PYZhy{}\PYZhy{}\PYZhy{}\PYZhy{}\PYZhy{}\PYZhy{}\PYZhy{}\PYZhy{}\PYZhy{}\PYZhy{}\PYZhy{}}
    \PY{c+c1}{\PYZsh{} In .csv speichern, wenn save=True}
    \PY{k}{if} \PY{n}{save} \PY{o}{==} \PY{k+kc}{True}\PY{p}{:}
        
        \PY{c+c1}{\PYZsh{}\PYZhy{}\PYZhy{}\PYZhy{}\PYZhy{}\PYZhy{}\PYZhy{}\PYZhy{}\PYZhy{}\PYZhy{}\PYZhy{}\PYZhy{}\PYZhy{}\PYZhy{}\PYZhy{}\PYZhy{}\PYZhy{}\PYZhy{}\PYZhy{}\PYZhy{}\PYZhy{}\PYZhy{}\PYZhy{}\PYZhy{}\PYZhy{}\PYZhy{}\PYZhy{}\PYZhy{}\PYZhy{}\PYZhy{}\PYZhy{}\PYZhy{}\PYZhy{}}
        \PY{c+c1}{\PYZsh{} Standardnamen wählen, wenn manu\PYZus{}name == False}
        \PY{k}{if} \PY{n}{manu\PYZus{}name} \PY{o}{==} \PY{k+kc}{False}\PY{p}{:}
            \PY{n}{submission}\PY{o}{.}\PY{n}{to\PYZus{}csv}\PY{p}{(}\PY{l+s+s1}{\PYZsq{}}\PY{l+s+s1}{./predicted\PYZus{}values.csv}\PY{l+s+s1}{\PYZsq{}}\PY{p}{,} \PY{n}{index}\PY{o}{=}\PY{k+kc}{False}\PY{p}{)}
            
        \PY{c+c1}{\PYZsh{}\PYZhy{}\PYZhy{}\PYZhy{}\PYZhy{}\PYZhy{}\PYZhy{}\PYZhy{}\PYZhy{}\PYZhy{}\PYZhy{}\PYZhy{}\PYZhy{}\PYZhy{}\PYZhy{}\PYZhy{}\PYZhy{}\PYZhy{}\PYZhy{}\PYZhy{}\PYZhy{}\PYZhy{}\PYZhy{}\PYZhy{}\PYZhy{}\PYZhy{}\PYZhy{}\PYZhy{}\PYZhy{}\PYZhy{}\PYZhy{}\PYZhy{}\PYZhy{}}
        \PY{c+c1}{\PYZsh{} Standardnamen mit timestamp kombinieren, wenn  manu\PYZus{}name == True}
        \PY{k}{if} \PY{n}{manu\PYZus{}name} \PY{o}{==} \PY{k+kc}{True}\PY{p}{:}            
            \PY{k+kn}{import} \PY{n+nn}{datetime} 
            \PY{n}{now} \PY{o}{=} \PY{n}{datetime}\PY{o}{.}\PY{n}{datetime}\PY{o}{.}\PY{n}{now}\PY{p}{(}\PY{p}{)}
            \PY{n}{name} \PY{o}{=} \PY{n}{now}\PY{o}{.}\PY{n}{strftime}\PY{p}{(}\PY{l+s+s1}{\PYZsq{}}\PY{l+s+s1}{\PYZpc{}}\PY{l+s+s1}{Y\PYZhy{}}\PY{l+s+s1}{\PYZpc{}}\PY{l+s+s1}{m\PYZhy{}}\PY{l+s+si}{\PYZpc{}d}\PY{l+s+s1}{T}\PY{l+s+s1}{\PYZpc{}}\PY{l+s+s1}{H}\PY{l+s+s1}{\PYZpc{}}\PY{l+s+s1}{M}\PY{l+s+s1}{\PYZpc{}}\PY{l+s+s1}{S}\PY{l+s+s1}{\PYZsq{}}\PY{p}{)} \PY{o}{+} \PY{p}{(}\PY{l+s+s1}{\PYZsq{}}\PY{l+s+s1}{\PYZhy{}}\PY{l+s+si}{\PYZpc{}02d}\PY{l+s+s1}{\PYZsq{}} \PY{o}{\PYZpc{}} \PY{p}{(}\PY{n}{now}\PY{o}{.}\PY{n}{microsecond} \PY{o}{/} \PY{l+m+mi}{10000}\PY{p}{)}\PY{p}{)}            
            \PY{n}{submission}\PY{o}{.}\PY{n}{to\PYZus{}csv}\PY{p}{(}\PY{l+s+s1}{\PYZsq{}}\PY{l+s+s1}{./predicted\PYZus{}values\PYZus{}}\PY{l+s+s1}{\PYZsq{}} \PY{o}{+} \PY{n+nb}{str}\PY{p}{(}\PY{n}{name}\PY{p}{)} \PY{o}{+} \PY{l+s+s1}{\PYZsq{}}\PY{l+s+s1}{.csv}\PY{l+s+s1}{\PYZsq{}}\PY{p}{,} \PY{n}{index}\PY{o}{=}\PY{k+kc}{False}\PY{p}{)}            
        
    \PY{k}{return} \PY{n}{submission}\PY{o}{.}\PY{n}{head}\PY{p}{(}\PY{p}{)}\PY{p}{,} \PY{n}{submission}\PY{o}{.}\PY{n}{loc}\PY{p}{[}\PY{n}{submission}\PY{p}{[}\PY{l+s+s1}{\PYZsq{}}\PY{l+s+s1}{Fehlerhaft}\PY{l+s+s1}{\PYZsq{}}\PY{p}{]} \PY{o}{==} \PY{l+m+mi}{1}\PY{p}{]}
\end{Verbatim}
\end{tcolorbox}

    \hypertarget{funktion-zum-filtern-von-quantilen}{%
\subsubsection{Funktion zum Filtern von
Quantilen}\label{funktion-zum-filtern-von-quantilen}}

    \begin{tcolorbox}[breakable, size=fbox, boxrule=1pt, pad at break*=1mm,colback=cellbackground, colframe=cellborder]
\prompt{In}{incolor}{5}{\boxspacing}
\begin{Verbatim}[commandchars=\\\{\}]
\PY{c+c1}{\PYZsh{} Definition einer Funktion, welche einen gegeben DataFrame }
\PY{c+c1}{\PYZsh{} um untere und obere Quantile beschneiden kann}
\PY{c+c1}{\PYZsh{}\PYZhy{}\PYZhy{}\PYZhy{}\PYZhy{}\PYZhy{}\PYZhy{}\PYZhy{}\PYZhy{}\PYZhy{}\PYZhy{}\PYZhy{}\PYZhy{}}
\PY{c+c1}{\PYZsh{} Argumente:}
\PY{c+c1}{\PYZsh{} \PYZhy{} orignal\PYZus{}df: DataFrame welcher bearbeitet werden soll}
\PY{c+c1}{\PYZsh{} \PYZhy{} quantile\PYZus{}low: Unteres Quantil bis zu welchem orignal\PYZus{}df beschnitten werden soll}
\PY{c+c1}{\PYZsh{} \PYZhy{} quantile\PYZus{}high: Oberes Quantil welchem orignal\PYZus{}df beschnitten werden soll}
\PY{c+c1}{\PYZsh{} \PYZhy{} colum\PYZus{}to\PYZus{}drop: Spalte des orignal\PYZus{}df, welche während des Vorgangs gedroppt werden soll}
\PY{c+c1}{\PYZsh{}\PYZhy{}\PYZhy{}\PYZhy{}\PYZhy{}\PYZhy{}\PYZhy{}\PYZhy{}\PYZhy{}\PYZhy{}\PYZhy{}\PYZhy{}\PYZhy{}}

\PY{k}{def} \PY{n+nf}{filter\PYZus{}my\PYZus{}df}\PY{p}{(}\PY{n}{orignal\PYZus{}df}\PY{p}{,} \PY{n}{quantile\PYZus{}low}\PY{p}{,} \PY{n}{quantile\PYZus{}high}\PY{p}{,} \PY{n}{colum\PYZus{}to\PYZus{}drop}\PY{p}{)}\PY{p}{:}

    \PY{c+c1}{\PYZsh{}\PYZhy{}\PYZhy{}\PYZhy{}\PYZhy{}\PYZhy{}\PYZhy{}\PYZhy{}\PYZhy{}\PYZhy{}\PYZhy{}\PYZhy{}\PYZhy{}\PYZhy{}\PYZhy{}\PYZhy{}\PYZhy{}\PYZhy{}\PYZhy{}\PYZhy{}\PYZhy{}\PYZhy{}\PYZhy{}\PYZhy{}\PYZhy{}\PYZhy{}\PYZhy{}\PYZhy{}\PYZhy{}}
    \PY{c+c1}{\PYZsh{} Spalte \PYZdq{}colum\PYZus{}to\PYZus{}drop\PYZdq{} aus dem Datensatz entfernen}
    \PY{n}{df\PYZus{}filtered} \PY{o}{=} \PY{n}{orignal\PYZus{}df}\PY{o}{.}\PY{n}{loc}\PY{p}{[}\PY{p}{:}\PY{p}{,} \PY{n}{orignal\PYZus{}df}\PY{o}{.}\PY{n}{columns} \PY{o}{!=} \PY{n}{colum\PYZus{}to\PYZus{}drop}\PY{p}{]} 
    \PY{c+c1}{\PYZsh{} Quantil\PYZhy{}DataFrame erzeugen}
    \PY{n}{quant\PYZus{}df} \PY{o}{=} \PY{n}{df\PYZus{}filtered}\PY{o}{.}\PY{n}{quantile}\PY{p}{(}\PY{p}{[}\PY{n}{quantile\PYZus{}low}\PY{p}{,} \PY{n}{quantile\PYZus{}high}\PY{p}{]}\PY{p}{)} 
    \PY{c+c1}{\PYZsh{} Quantil\PYZhy{}DataFrame auf orignal\PYZus{}df anweden}
    \PY{n}{df\PYZus{}filtered} \PY{o}{=} \PY{n}{df\PYZus{}filtered}\PY{o}{.}\PY{n}{apply}\PY{p}{(}\PY{k}{lambda} \PY{n}{x}\PY{p}{:} \PY{n}{x}\PY{p}{[}\PY{p}{(}\PY{n}{x}\PY{o}{\PYZgt{}}\PY{n}{quant\PYZus{}df}\PY{o}{.}\PY{n}{loc}\PY{p}{[}\PY{n}{quantile\PYZus{}low}\PY{p}{,}\PY{n}{x}\PY{o}{.}\PY{n}{name}\PY{p}{]}\PY{p}{)} \PY{o}{\PYZam{}} 
                                                \PY{p}{(}\PY{n}{x} \PY{o}{\PYZlt{}} \PY{n}{quant\PYZus{}df}\PY{o}{.}\PY{n}{loc}\PY{p}{[}\PY{n}{quantile\PYZus{}high}\PY{p}{,}\PY{n}{x}\PY{o}{.}\PY{n}{name}\PY{p}{]}\PY{p}{)}\PY{p}{]}\PY{p}{,}
                                                \PY{n}{axis}\PY{o}{=}\PY{l+m+mi}{0}\PY{p}{)}
    \PY{c+c1}{\PYZsh{}\PYZhy{}\PYZhy{}\PYZhy{}\PYZhy{}\PYZhy{}\PYZhy{}\PYZhy{}\PYZhy{}\PYZhy{}\PYZhy{}\PYZhy{}\PYZhy{}\PYZhy{}\PYZhy{}\PYZhy{}\PYZhy{}\PYZhy{}\PYZhy{}\PYZhy{}\PYZhy{}\PYZhy{}\PYZhy{}\PYZhy{}\PYZhy{}\PYZhy{}\PYZhy{}\PYZhy{}\PYZhy{}}
    \PY{c+c1}{\PYZsh{} Spalte \PYZdq{}Fehlerhaft\PYZdq{} dem gefiltertem DataFrame wieder anfügen}
    \PY{n}{df\PYZus{}filtered} \PY{o}{=} \PY{n}{pd}\PY{o}{.}\PY{n}{concat}\PY{p}{(}\PY{p}{[}\PY{n}{orignal\PYZus{}df}\PY{o}{.}\PY{n}{loc}\PY{p}{[}\PY{p}{:}\PY{p}{,}\PY{n}{colum\PYZus{}to\PYZus{}drop}\PY{p}{]}\PY{p}{,} \PY{n}{df\PYZus{}filtered}\PY{p}{]}\PY{p}{,} \PY{n}{axis}\PY{o}{=}\PY{l+m+mi}{1}\PY{p}{)}
    \PY{c+c1}{\PYZsh{} Aus Beschneidung resultierende NaN\PYZhy{}Werte bereinigen}
    \PY{n}{df\PYZus{}filtered}\PY{o}{.}\PY{n}{dropna}\PY{p}{(}\PY{n}{inplace}\PY{o}{=}\PY{k+kc}{True}\PY{p}{)}
    
    \PY{k}{return} \PY{n}{df\PYZus{}filtered}
\end{Verbatim}
\end{tcolorbox}

    \hypertarget{datensatz-einlesen-bereinigigen-und-betrachten}{%
\subsection{Datensatz einlesen (bereinigigen) und
betrachten}\label{datensatz-einlesen-bereinigigen-und-betrachten}}

\hypertarget{datensatz-einlesen}{%
\subsubsection{Datensatz einlesen}\label{datensatz-einlesen}}

    \begin{tcolorbox}[breakable, size=fbox, boxrule=1pt, pad at break*=1mm,colback=cellbackground, colframe=cellborder]
\prompt{In}{incolor}{6}{\boxspacing}
\begin{Verbatim}[commandchars=\\\{\}]
\PY{c+c1}{\PYZsh{}\PYZhy{}\PYZhy{}\PYZhy{}\PYZhy{}\PYZhy{}\PYZhy{}\PYZhy{}\PYZhy{}\PYZhy{}\PYZhy{}\PYZhy{}\PYZhy{}\PYZhy{}\PYZhy{}\PYZhy{}\PYZhy{}\PYZhy{}\PYZhy{}\PYZhy{}\PYZhy{}\PYZhy{}\PYZhy{}\PYZhy{}\PYZhy{}\PYZhy{}\PYZhy{}\PYZhy{}\PYZhy{}}
\PY{c+c1}{\PYZsh{} Datensatz einlesen}
\PY{n}{data} \PY{o}{=} \PY{n}{pd}\PY{o}{.}\PY{n}{read\PYZus{}csv}\PY{p}{(}\PY{l+s+s1}{\PYZsq{}}\PY{l+s+s1}{train.csv}\PY{l+s+s1}{\PYZsq{}}\PY{p}{,} \PY{n}{index\PYZus{}col}\PY{o}{=}\PY{l+m+mi}{0}\PY{p}{)}
\end{Verbatim}
\end{tcolorbox}

    \hypertarget{optionale-datensatzbereinigung}{%
\subsubsection{Optionale
Datensatzbereinigung}\label{optionale-datensatzbereinigung}}

    \begin{tcolorbox}[breakable, size=fbox, boxrule=1pt, pad at break*=1mm,colback=cellbackground, colframe=cellborder]
\prompt{In}{incolor}{7}{\boxspacing}
\begin{Verbatim}[commandchars=\\\{\}]
\PY{l+s+sd}{\PYZdq{}\PYZdq{}\PYZdq{}}
\PY{l+s+sd}{\PYZsh{}\PYZhy{}\PYZhy{}\PYZhy{}\PYZhy{}\PYZhy{}\PYZhy{}\PYZhy{}\PYZhy{}\PYZhy{}\PYZhy{}\PYZhy{}\PYZhy{}\PYZhy{}\PYZhy{}\PYZhy{}\PYZhy{}\PYZhy{}\PYZhy{}\PYZhy{}\PYZhy{}\PYZhy{}\PYZhy{}\PYZhy{}\PYZhy{}\PYZhy{}\PYZhy{}\PYZhy{}\PYZhy{}}
\PY{l+s+sd}{\PYZsh{} Datensatz unterteilen}

\PY{l+s+sd}{df\PYZus{}fehlerfrei = data.loc[data[\PYZsq{}Fehlerhaft\PYZsq{}] == 0]}
\PY{l+s+sd}{df\PYZus{}fehlerhaft = data.loc[data[\PYZsq{}Fehlerhaft\PYZsq{}] == 1]}
\PY{l+s+sd}{\PYZdq{}\PYZdq{}\PYZdq{}}

\PY{l+s+sd}{\PYZdq{}\PYZdq{}\PYZdq{}}
\PY{l+s+sd}{\PYZsh{}\PYZhy{}\PYZhy{}\PYZhy{}\PYZhy{}\PYZhy{}\PYZhy{}\PYZhy{}\PYZhy{}\PYZhy{}\PYZhy{}\PYZhy{}\PYZhy{}\PYZhy{}\PYZhy{}\PYZhy{}\PYZhy{}\PYZhy{}\PYZhy{}\PYZhy{}\PYZhy{}\PYZhy{}\PYZhy{}\PYZhy{}\PYZhy{}\PYZhy{}\PYZhy{}\PYZhy{}\PYZhy{}}
\PY{l+s+sd}{\PYZsh{} Fehlerfreie Stückgüter}
\PY{l+s+sd}{colum\PYZus{}to\PYZus{}drop = \PYZsq{}Fehlerhaft\PYZsq{}}
\PY{l+s+sd}{orignal\PYZus{}df = df\PYZus{}fehlerfrei}
\PY{l+s+sd}{low = .0 \PYZsh{} Unteres Quantil }
\PY{l+s+sd}{high = .99 \PYZsh{} Oberes Quantil}
\PY{l+s+sd}{df\PYZus{}fehlerfrei\PYZus{}filtered = filter\PYZus{}my\PYZus{}df(df\PYZus{}fehlerfrei, low, high, colum\PYZus{}to\PYZus{}drop)}

\PY{l+s+sd}{\PYZsh{}\PYZhy{}\PYZhy{}\PYZhy{}\PYZhy{}\PYZhy{}\PYZhy{}\PYZhy{}\PYZhy{}\PYZhy{}\PYZhy{}\PYZhy{}\PYZhy{}\PYZhy{}\PYZhy{}\PYZhy{}\PYZhy{}\PYZhy{}\PYZhy{}\PYZhy{}\PYZhy{}\PYZhy{}\PYZhy{}\PYZhy{}\PYZhy{}\PYZhy{}\PYZhy{}\PYZhy{}\PYZhy{}}
\PY{l+s+sd}{\PYZsh{} Fehlerhafte Stückgüter}
\PY{l+s+sd}{colum\PYZus{}to\PYZus{}drop = \PYZsq{}Fehlerhaft\PYZsq{}}
\PY{l+s+sd}{orignal\PYZus{}df = df\PYZus{}fehlerhaft}
\PY{l+s+sd}{low = .018333 \PYZsh{} Unteres Quantil }
\PY{l+s+sd}{high = 1. \PYZsh{} Oberes Quantil}
\PY{l+s+sd}{df\PYZus{}fehlerhaft\PYZus{}filtered = filter\PYZus{}my\PYZus{}df(df\PYZus{}fehlerhaft, low, high, colum\PYZus{}to\PYZus{}drop)}

\PY{l+s+sd}{\PYZsh{}\PYZhy{}\PYZhy{}\PYZhy{}\PYZhy{}\PYZhy{}\PYZhy{}\PYZhy{}\PYZhy{}\PYZhy{}\PYZhy{}\PYZhy{}\PYZhy{}\PYZhy{}\PYZhy{}\PYZhy{}\PYZhy{}\PYZhy{}\PYZhy{}\PYZhy{}\PYZhy{}\PYZhy{}\PYZhy{}\PYZhy{}\PYZhy{}\PYZhy{}\PYZhy{}\PYZhy{}\PYZhy{}}
\PY{l+s+sd}{\PYZsh{} Teil\PYZhy{}DataFrames zusammenführen}
\PY{l+s+sd}{data\PYZus{}filtered = pd.concat([df\PYZus{}fehlerhaft\PYZus{}filtered, df\PYZus{}fehlerfrei\PYZus{}filtered], sort=False)}
\PY{l+s+sd}{\PYZdq{}\PYZdq{}\PYZdq{}}
\end{Verbatim}
\end{tcolorbox}

            \begin{tcolorbox}[breakable, size=fbox, boxrule=.5pt, pad at break*=1mm, opacityfill=0]
\prompt{Out}{outcolor}{7}{\boxspacing}
\begin{Verbatim}[commandchars=\\\{\}]
"\textbackslash{}n\#----------------------------\textbackslash{}n\# Fehlerfreie Stückgüter\textbackslash{}ncolum\_to\_drop =
'Fehlerhaft'\textbackslash{}norignal\_df = df\_fehlerfrei\textbackslash{}nlow = .0 \# Unteres Quantil \textbackslash{}nhigh =
.99 \# Oberes Quantil\textbackslash{}ndf\_fehlerfrei\_filtered = filter\_my\_df(df\_fehlerfrei, low,
high, colum\_to\_drop)\textbackslash{}n\textbackslash{}n\#----------------------------\textbackslash{}n\# Fehlerhafte
Stückgüter\textbackslash{}ncolum\_to\_drop = 'Fehlerhaft'\textbackslash{}norignal\_df = df\_fehlerhaft\textbackslash{}nlow =
.018333 \# Unteres Quantil \textbackslash{}nhigh = 1. \# Oberes Quantil\textbackslash{}ndf\_fehlerhaft\_filtered =
filter\_my\_df(df\_fehlerhaft, low, high,
colum\_to\_drop)\textbackslash{}n\textbackslash{}n\#----------------------------\textbackslash{}n\# Teil-DataFrames
zusammenführen\textbackslash{}ndata\_filtered = pd.concat([df\_fehlerhaft\_filtered,
df\_fehlerfrei\_filtered], sort=False)\textbackslash{}n"
\end{Verbatim}
\end{tcolorbox}
        
    \hypertarget{beschreibung-der-separierten-datensuxe4tze-betrachtung-min--maximum-und-qunatile}{%
\subsubsection{Beschreibung der separierten Datensätze (Betrachtung
Min-/ Maximum und
Qunatile)}\label{beschreibung-der-separierten-datensuxe4tze-betrachtung-min--maximum-und-qunatile}}

    \begin{tcolorbox}[breakable, size=fbox, boxrule=1pt, pad at break*=1mm,colback=cellbackground, colframe=cellborder]
\prompt{In}{incolor}{8}{\boxspacing}
\begin{Verbatim}[commandchars=\\\{\}]
\PY{l+s+sd}{\PYZdq{}\PYZdq{}\PYZdq{}}
\PY{l+s+sd}{df\PYZus{}fehlerfrei.describe()}
\PY{l+s+sd}{\PYZdq{}\PYZdq{}\PYZdq{}}
\end{Verbatim}
\end{tcolorbox}

            \begin{tcolorbox}[breakable, size=fbox, boxrule=.5pt, pad at break*=1mm, opacityfill=0]
\prompt{Out}{outcolor}{8}{\boxspacing}
\begin{Verbatim}[commandchars=\\\{\}]
'\textbackslash{}ndf\_fehlerfrei.describe()\textbackslash{}n'
\end{Verbatim}
\end{tcolorbox}
        
    \begin{tcolorbox}[breakable, size=fbox, boxrule=1pt, pad at break*=1mm,colback=cellbackground, colframe=cellborder]
\prompt{In}{incolor}{9}{\boxspacing}
\begin{Verbatim}[commandchars=\\\{\}]
\PY{l+s+sd}{\PYZdq{}\PYZdq{}\PYZdq{}}
\PY{l+s+sd}{df\PYZus{}fehlerhaft.describe()}
\PY{l+s+sd}{\PYZdq{}\PYZdq{}\PYZdq{}}
\end{Verbatim}
\end{tcolorbox}

            \begin{tcolorbox}[breakable, size=fbox, boxrule=.5pt, pad at break*=1mm, opacityfill=0]
\prompt{Out}{outcolor}{9}{\boxspacing}
\begin{Verbatim}[commandchars=\\\{\}]
'\textbackslash{}ndf\_fehlerhaft.describe()\textbackslash{}n'
\end{Verbatim}
\end{tcolorbox}
        
    \begin{tcolorbox}[breakable, size=fbox, boxrule=1pt, pad at break*=1mm,colback=cellbackground, colframe=cellborder]
\prompt{In}{incolor}{10}{\boxspacing}
\begin{Verbatim}[commandchars=\\\{\}]
\PY{n}{data\PYZus{}new} \PY{o}{=} \PY{n}{data} \PY{c+c1}{\PYZsh{}\PYZus{}filtered}
\PY{n}{data\PYZus{}new}\PY{p}{[}\PY{l+s+s1}{\PYZsq{}}\PY{l+s+s1}{Fehlerhaft}\PY{l+s+s1}{\PYZsq{}}\PY{p}{]}\PY{o}{.}\PY{n}{value\PYZus{}counts}\PY{p}{(}\PY{p}{)}
\end{Verbatim}
\end{tcolorbox}

            \begin{tcolorbox}[breakable, size=fbox, boxrule=.5pt, pad at break*=1mm, opacityfill=0]
\prompt{Out}{outcolor}{10}{\boxspacing}
\begin{Verbatim}[commandchars=\\\{\}]
0    20208
1      284
Name: Fehlerhaft, dtype: int64
\end{Verbatim}
\end{tcolorbox}
        
    \hypertarget{betrachtung-korrelationsmatrix}{%
\subsubsection{Betrachtung
Korrelationsmatrix}\label{betrachtung-korrelationsmatrix}}

    \begin{tcolorbox}[breakable, size=fbox, boxrule=1pt, pad at break*=1mm,colback=cellbackground, colframe=cellborder]
\prompt{In}{incolor}{11}{\boxspacing}
\begin{Verbatim}[commandchars=\\\{\}]
\PY{n}{data\PYZus{}new} \PY{o}{=} \PY{n}{data} \PY{c+c1}{\PYZsh{}\PYZus{}filtered}

\PY{c+c1}{\PYZsh{}\PYZhy{}\PYZhy{}\PYZhy{}\PYZhy{}\PYZhy{}\PYZhy{}\PYZhy{}\PYZhy{}\PYZhy{}\PYZhy{}\PYZhy{}\PYZhy{}\PYZhy{}\PYZhy{}\PYZhy{}\PYZhy{}\PYZhy{}\PYZhy{}\PYZhy{}\PYZhy{}\PYZhy{}\PYZhy{}\PYZhy{}\PYZhy{}\PYZhy{}\PYZhy{}\PYZhy{}\PYZhy{}}
\PY{c+c1}{\PYZsh{} Für schnellere Laufzeit und mehr Übersicht in den Plots: Stichprobe der Daten abbilden}
\PY{n}{data\PYZus{}sample} \PY{o}{=} \PY{n}{data\PYZus{}new}\PY{o}{.}\PY{n}{sample}\PY{p}{(}\PY{l+m+mi}{2000}\PY{p}{,} \PY{n}{random\PYZus{}state}\PY{o}{=}\PY{l+m+mi}{28}\PY{p}{)}  \PY{c+c1}{\PYZsh{} random\PYZus{}state sorgt für reproduzierbare Stichprobe, sodass die Stichprobe für uns alle identisch ist}
\PY{n}{\PYZus{}} \PY{o}{=} \PY{n}{pd}\PY{o}{.}\PY{n}{plotting}\PY{o}{.}\PY{n}{scatter\PYZus{}matrix}\PY{p}{(}\PY{n}{data\PYZus{}sample}\PY{p}{,} \PY{n}{c}\PY{o}{=}\PY{n}{data\PYZus{}sample}\PY{p}{[}\PY{l+s+s1}{\PYZsq{}}\PY{l+s+s1}{Fehlerhaft}\PY{l+s+s1}{\PYZsq{}}\PY{p}{]}\PY{p}{,} \PY{n}{cmap}\PY{o}{=}\PY{l+s+s1}{\PYZsq{}}\PY{l+s+s1}{seismic}\PY{l+s+s1}{\PYZsq{}}\PY{p}{,} \PY{n}{figsize}\PY{o}{=}\PY{p}{(}\PY{l+m+mi}{16}\PY{p}{,} \PY{l+m+mi}{20}\PY{p}{)}\PY{p}{)}
\end{Verbatim}
\end{tcolorbox}

    \begin{center}
    \adjustimage{max size={0.9\linewidth}{0.9\paperheight}}{output_19_0.png}
    \end{center}
    { \hspace*{\fill} \\}
    
    \hypertarget{dateinsatz-in-traings--und-validierungsteil-splitten}{%
\subsubsection{Dateinsatz in Traings- und Validierungsteil
splitten}\label{dateinsatz-in-traings--und-validierungsteil-splitten}}

    \begin{tcolorbox}[breakable, size=fbox, boxrule=1pt, pad at break*=1mm,colback=cellbackground, colframe=cellborder]
\prompt{In}{incolor}{12}{\boxspacing}
\begin{Verbatim}[commandchars=\\\{\}]
\PY{n}{X} \PY{o}{=} \PY{n}{data\PYZus{}new}\PY{o}{.}\PY{n}{drop}\PY{p}{(}\PY{l+s+s1}{\PYZsq{}}\PY{l+s+s1}{Fehlerhaft}\PY{l+s+s1}{\PYZsq{}}\PY{p}{,} \PY{n}{axis}\PY{o}{=}\PY{l+m+mi}{1}\PY{p}{)}
\PY{n}{y} \PY{o}{=} \PY{n}{data\PYZus{}new}\PY{p}{[}\PY{l+s+s1}{\PYZsq{}}\PY{l+s+s1}{Fehlerhaft}\PY{l+s+s1}{\PYZsq{}}\PY{p}{]}
\PY{n}{X\PYZus{}train}\PY{p}{,} \PY{n}{X\PYZus{}validierung}\PY{p}{,} \PY{n}{y\PYZus{}train}\PY{p}{,} \PY{n}{y\PYZus{}validierung} \PY{o}{=} \PY{n}{train\PYZus{}test\PYZus{}split}\PY{p}{(}\PY{n}{X}\PY{p}{,} \PY{n}{y}\PY{p}{,} \PY{n}{test\PYZus{}size}\PY{o}{=}\PY{l+m+mf}{0.2}\PY{p}{,} \PY{n}{random\PYZus{}state}\PY{o}{=}\PY{l+m+mi}{2121}\PY{p}{)}
\end{Verbatim}
\end{tcolorbox}

    \hypertarget{modell-aufstellen}{%
\subsection{Modell aufstellen}\label{modell-aufstellen}}

    \begin{tcolorbox}[breakable, size=fbox, boxrule=1pt, pad at break*=1mm,colback=cellbackground, colframe=cellborder]
\prompt{In}{incolor}{13}{\boxspacing}
\begin{Verbatim}[commandchars=\\\{\}]
\PY{c+c1}{\PYZsh{} Definition einer Funktion, welche eine Gittersuche mit einem DecisionTreeClassifier durchführt}
\PY{c+c1}{\PYZsh{} und nach einer 5\PYZhy{}fach Kreuzvalidierung das beste Modell zurückgibt}
\PY{c+c1}{\PYZsh{}\PYZhy{}\PYZhy{}\PYZhy{}\PYZhy{}\PYZhy{}\PYZhy{}\PYZhy{}\PYZhy{}\PYZhy{}\PYZhy{}\PYZhy{}\PYZhy{}}
\PY{c+c1}{\PYZsh{} Argumente:}
\PY{c+c1}{\PYZsh{} \PYZhy{} i: Fügt X\PYZca{}i der Featurematrix hinzu}
\PY{c+c1}{\PYZsh{} \PYZhy{} X: DataFrame auf welchem die Prognose durchgefürt werden soll (ohne die Zielgröße)}
\PY{c+c1}{\PYZsh{} \PYZhy{} y\PYZus{}true: Zum DataFrame X gehörige Werte der Zielgröße}
\PY{c+c1}{\PYZsh{} \PYZhy{} my\PYZus{}scaler: Zu verwendender Scaler; per default MinMaxScaler; weitere Scaler: RobustScaler, Standardscaler}
\PY{c+c1}{\PYZsh{} \PYZhy{} C: Parameter für die Regularisierung; je kleiner, desto stärker die Regularisierung}
\PY{c+c1}{\PYZsh{} \PYZhy{} max\PYZus{}features: Anzahl der Features die einbezogen werden sollen (default=[\PYZsq{}auto\PYZsq{}]=sqrt(n\PYZus{}features))}
\PY{c+c1}{\PYZsh{} \PYZhy{} jobs: Anzahl der Threads die für den Durchlauf zur Verfügung stehen}
\PY{c+c1}{\PYZsh{} \PYZhy{} gs\PYZus{}scoring: Scoring Verfahren im Rahmen der GridSearch }
\PY{c+c1}{\PYZsh{} \PYZhy{} folts: Komplexität der Kreuzvalidierung}
\PY{c+c1}{\PYZsh{}\PYZhy{}\PYZhy{}\PYZhy{}\PYZhy{}\PYZhy{}\PYZhy{}\PYZhy{}\PYZhy{}\PYZhy{}\PYZhy{}\PYZhy{}\PYZhy{}}

\PY{k}{def} \PY{n+nf}{dt\PYZus{}1}\PY{p}{(}\PY{n}{i}\PY{p}{,} \PY{n}{X}\PY{p}{,} \PY{n}{y\PYZus{}true}\PY{p}{,} \PY{n}{my\PYZus{}scaler}\PY{o}{=}\PY{n}{MinMaxScaler}\PY{p}{,} \PY{n}{jobs}\PY{o}{=}\PY{o}{\PYZhy{}}\PY{l+m+mi}{3}\PY{p}{,} \PY{n}{gs\PYZus{}scoring}\PY{o}{=}\PY{l+s+s1}{\PYZsq{}}\PY{l+s+s1}{f1}\PY{l+s+s1}{\PYZsq{}}\PY{p}{,} \PY{n}{folts}\PY{o}{=}\PY{l+m+mi}{5}\PY{p}{,} \PY{n}{max\PYZus{}features}\PY{o}{=}\PY{k+kc}{None}\PY{p}{)}\PY{p}{:}

    \PY{c+c1}{\PYZsh{}\PYZhy{}\PYZhy{}\PYZhy{}\PYZhy{}\PYZhy{}\PYZhy{}\PYZhy{}\PYZhy{}\PYZhy{}\PYZhy{}\PYZhy{}\PYZhy{}\PYZhy{}\PYZhy{}\PYZhy{}\PYZhy{}\PYZhy{}\PYZhy{}\PYZhy{}\PYZhy{}\PYZhy{}\PYZhy{}\PYZhy{}\PYZhy{}\PYZhy{}\PYZhy{}\PYZhy{}\PYZhy{}\PYZhy{}\PYZhy{}\PYZhy{}\PYZhy{}}
    \PY{c+c1}{\PYZsh{} Pipeline erzeugen}
    \PY{n}{prediction\PYZus{}pipe} \PY{o}{=} \PY{n}{Pipeline}\PY{p}{(}\PY{p}{[}\PY{p}{(}\PY{l+s+s1}{\PYZsq{}}\PY{l+s+s1}{scaler}\PY{l+s+s1}{\PYZsq{}}\PY{p}{,} \PY{n}{my\PYZus{}scaler}\PY{p}{(}\PY{p}{)}\PY{p}{)}\PY{p}{,} 
                                \PY{p}{(}\PY{l+s+s1}{\PYZsq{}}\PY{l+s+s1}{add\PYZus{}x\PYZus{}square}\PY{l+s+s1}{\PYZsq{}}\PY{p}{,} \PY{n}{PolynomialFeatures}\PY{p}{(}\PY{n}{degree}\PY{o}{=}\PY{n}{i}\PY{p}{)}\PY{p}{)}\PY{p}{,}
                                \PY{p}{(}\PY{l+s+s1}{\PYZsq{}}\PY{l+s+s1}{classifier}\PY{l+s+s1}{\PYZsq{}}\PY{p}{,} \PY{n}{DecisionTreeClassifier}\PY{p}{(}\PY{p}{)}\PY{p}{)}\PY{p}{]}\PY{p}{)}

    \PY{c+c1}{\PYZsh{}\PYZhy{}\PYZhy{}\PYZhy{}\PYZhy{}\PYZhy{}\PYZhy{}\PYZhy{}\PYZhy{}\PYZhy{}\PYZhy{}\PYZhy{}\PYZhy{}\PYZhy{}\PYZhy{}\PYZhy{}\PYZhy{}\PYZhy{}\PYZhy{}\PYZhy{}\PYZhy{}\PYZhy{}\PYZhy{}\PYZhy{}\PYZhy{}\PYZhy{}\PYZhy{}\PYZhy{}\PYZhy{}\PYZhy{}\PYZhy{}\PYZhy{}\PYZhy{}}
    \PY{c+c1}{\PYZsh{} Parameter Grid}
    \PY{n}{param\PYZus{}grid} \PY{o}{=} \PY{p}{[}\PY{p}{\PYZob{}}\PY{l+s+s1}{\PYZsq{}}\PY{l+s+s1}{classifier}\PY{l+s+s1}{\PYZsq{}}\PY{p}{:} \PY{p}{[}\PY{n}{DecisionTreeClassifier}\PY{p}{(}\PY{p}{)}\PY{p}{]}\PY{p}{,}
                   \PY{l+s+s1}{\PYZsq{}}\PY{l+s+s1}{classifier\PYZus{}\PYZus{}max\PYZus{}features}\PY{l+s+s1}{\PYZsq{}}\PY{p}{:} \PY{p}{[}\PY{n}{max\PYZus{}features}\PY{p}{]}\PY{p}{,}
                   \PY{l+s+s1}{\PYZsq{}}\PY{l+s+s1}{classifier\PYZus{}\PYZus{}random\PYZus{}state}\PY{l+s+s1}{\PYZsq{}}\PY{p}{:} \PY{p}{[}\PY{l+m+mi}{2111}\PY{p}{]}\PY{p}{\PYZcb{}}
                  \PY{p}{]}

    \PY{c+c1}{\PYZsh{}\PYZhy{}\PYZhy{}\PYZhy{}\PYZhy{}\PYZhy{}\PYZhy{}\PYZhy{}\PYZhy{}\PYZhy{}\PYZhy{}\PYZhy{}\PYZhy{}\PYZhy{}\PYZhy{}\PYZhy{}\PYZhy{}\PYZhy{}\PYZhy{}\PYZhy{}\PYZhy{}\PYZhy{}\PYZhy{}\PYZhy{}\PYZhy{}\PYZhy{}\PYZhy{}\PYZhy{}\PYZhy{}\PYZhy{}\PYZhy{}\PYZhy{}\PYZhy{}}
    \PY{c+c1}{\PYZsh{} StratifiedKFold für unbalancierten Datensatz}
    \PY{n}{scv} \PY{o}{=} \PY{n}{StratifiedKFold}\PY{p}{(}\PY{n}{n\PYZus{}splits}\PY{o}{=}\PY{n}{folts}\PY{p}{)}

    \PY{c+c1}{\PYZsh{}\PYZhy{}\PYZhy{}\PYZhy{}\PYZhy{}\PYZhy{}\PYZhy{}\PYZhy{}\PYZhy{}\PYZhy{}\PYZhy{}\PYZhy{}\PYZhy{}\PYZhy{}\PYZhy{}\PYZhy{}\PYZhy{}\PYZhy{}\PYZhy{}\PYZhy{}\PYZhy{}\PYZhy{}\PYZhy{}\PYZhy{}\PYZhy{}\PYZhy{}\PYZhy{}\PYZhy{}\PYZhy{}\PYZhy{}\PYZhy{}\PYZhy{}\PYZhy{}}
    \PY{c+c1}{\PYZsh{} Gittersuche}
    \PY{n}{grid\PYZus{}search} \PY{o}{=} \PY{n}{GridSearchCV}\PY{p}{(}
            \PY{n}{estimator}\PY{o}{=}\PY{n}{prediction\PYZus{}pipe}\PY{p}{,} 
            \PY{n}{param\PYZus{}grid}\PY{o}{=}\PY{n}{param\PYZus{}grid}\PY{p}{,}
            \PY{n}{scoring}\PY{o}{=}\PY{n}{gs\PYZus{}scoring}\PY{p}{,}
            \PY{n}{cv}\PY{o}{=}\PY{n}{scv}\PY{p}{,}
            \PY{n}{verbose}\PY{o}{=}\PY{k+kc}{True}\PY{p}{,}
            \PY{n}{n\PYZus{}jobs}\PY{o}{=}\PY{n}{jobs}\PY{p}{,}
            \PY{n}{iid}\PY{o}{=}\PY{k+kc}{False}\PY{p}{)}

    \PY{c+c1}{\PYZsh{}\PYZhy{}\PYZhy{}\PYZhy{}\PYZhy{}\PYZhy{}\PYZhy{}\PYZhy{}\PYZhy{}\PYZhy{}\PYZhy{}\PYZhy{}\PYZhy{}\PYZhy{}\PYZhy{}\PYZhy{}\PYZhy{}\PYZhy{}\PYZhy{}\PYZhy{}\PYZhy{}\PYZhy{}\PYZhy{}\PYZhy{}\PYZhy{}\PYZhy{}\PYZhy{}\PYZhy{}\PYZhy{}\PYZhy{}\PYZhy{}\PYZhy{}\PYZhy{}}
    \PY{c+c1}{\PYZsh{} Fit}
    \PY{n}{model} \PY{o}{=} \PY{n}{grid\PYZus{}search}\PY{o}{.}\PY{n}{fit}\PY{p}{(}\PY{n}{X}\PY{p}{,}\PY{n}{y\PYZus{}true}\PY{p}{)}

    \PY{k}{return} \PY{n}{model}\PY{p}{,} \PY{n}{grid\PYZus{}search}\PY{o}{.}\PY{n}{best\PYZus{}score\PYZus{}}
\end{Verbatim}
\end{tcolorbox}

    \hypertarget{modelaufruf-und-scoring}{%
\subsubsection{Modelaufruf und Scoring}\label{modelaufruf-und-scoring}}

\hypertarget{modell-1}{%
\paragraph{Modell 1}\label{modell-1}}

    \begin{tcolorbox}[breakable, size=fbox, boxrule=1pt, pad at break*=1mm,colback=cellbackground, colframe=cellborder]
\prompt{In}{incolor}{14}{\boxspacing}
\begin{Verbatim}[commandchars=\\\{\}]
\PY{n}{dt\PYZus{}model}\PY{p}{,} \PY{n}{dt\PYZus{}score} \PY{o}{=} \PY{n}{dt\PYZus{}1}\PY{p}{(}\PY{l+m+mi}{1}\PY{p}{,} \PY{n}{X\PYZus{}train}\PY{p}{,} \PY{n}{y\PYZus{}train}\PY{p}{,} \PY{n}{my\PYZus{}scaler}\PY{o}{=}\PY{n}{MinMaxScaler}\PY{p}{,} \PY{n}{jobs}\PY{o}{=}\PY{o}{\PYZhy{}}\PY{l+m+mi}{3}\PY{p}{,} \PY{n}{gs\PYZus{}scoring}\PY{o}{=}\PY{l+s+s1}{\PYZsq{}}\PY{l+s+s1}{f1}\PY{l+s+s1}{\PYZsq{}}\PY{p}{,} \PY{n}{folts}\PY{o}{=}\PY{l+m+mi}{5}\PY{p}{,} \PY{n}{max\PYZus{}features}\PY{o}{=}\PY{k+kc}{None}\PY{p}{)}
\PY{n}{dt\PYZus{}score}
\end{Verbatim}
\end{tcolorbox}

    \begin{Verbatim}[commandchars=\\\{\}]
Fitting 5 folds for each of 1 candidates, totalling 5 fits
    \end{Verbatim}

    \begin{Verbatim}[commandchars=\\\{\}]
[Parallel(n\_jobs=-3)]: Using backend LokyBackend with 10 concurrent workers.
[Parallel(n\_jobs=-3)]: Done   5 out of   5 | elapsed:    1.3s finished
    \end{Verbatim}

            \begin{tcolorbox}[breakable, size=fbox, boxrule=.5pt, pad at break*=1mm, opacityfill=0]
\prompt{Out}{outcolor}{14}{\boxspacing}
\begin{Verbatim}[commandchars=\\\{\}]
0.8343291872700128
\end{Verbatim}
\end{tcolorbox}
        
    \hypertarget{modell-2}{%
\paragraph{Modell 2}\label{modell-2}}

    \begin{tcolorbox}[breakable, size=fbox, boxrule=1pt, pad at break*=1mm,colback=cellbackground, colframe=cellborder]
\prompt{In}{incolor}{15}{\boxspacing}
\begin{Verbatim}[commandchars=\\\{\}]
\PY{n}{dt\PYZus{}model2}\PY{p}{,} \PY{n}{dt\PYZus{}score2} \PY{o}{=} \PY{n}{dt\PYZus{}1}\PY{p}{(}\PY{l+m+mi}{1}\PY{p}{,} \PY{n}{X\PYZus{}train}\PY{p}{,} \PY{n}{y\PYZus{}train}\PY{p}{,} \PY{n}{my\PYZus{}scaler}\PY{o}{=}\PY{n}{StandardScaler}\PY{p}{,} \PY{n}{jobs}\PY{o}{=}\PY{o}{\PYZhy{}}\PY{l+m+mi}{3}\PY{p}{,} \PY{n}{gs\PYZus{}scoring}\PY{o}{=}\PY{l+s+s1}{\PYZsq{}}\PY{l+s+s1}{f1}\PY{l+s+s1}{\PYZsq{}}\PY{p}{,} \PY{n}{folts}\PY{o}{=}\PY{l+m+mi}{5}\PY{p}{,} \PY{n}{max\PYZus{}features}\PY{o}{=}\PY{k+kc}{None}\PY{p}{)}
\PY{n}{dt\PYZus{}score}
\end{Verbatim}
\end{tcolorbox}

    \begin{Verbatim}[commandchars=\\\{\}]
[Parallel(n\_jobs=-3)]: Using backend LokyBackend with 10 concurrent workers.
    \end{Verbatim}

    \begin{Verbatim}[commandchars=\\\{\}]
Fitting 5 folds for each of 1 candidates, totalling 5 fits
    \end{Verbatim}

    \begin{Verbatim}[commandchars=\\\{\}]
[Parallel(n\_jobs=-3)]: Done   5 out of   5 | elapsed:    1.0s finished
    \end{Verbatim}

            \begin{tcolorbox}[breakable, size=fbox, boxrule=.5pt, pad at break*=1mm, opacityfill=0]
\prompt{Out}{outcolor}{15}{\boxspacing}
\begin{Verbatim}[commandchars=\\\{\}]
0.8343291872700128
\end{Verbatim}
\end{tcolorbox}
        
    \begin{tcolorbox}[breakable, size=fbox, boxrule=1pt, pad at break*=1mm,colback=cellbackground, colframe=cellborder]
\prompt{In}{incolor}{16}{\boxspacing}
\begin{Verbatim}[commandchars=\\\{\}]
\PY{n}{dt\PYZus{}model3}\PY{p}{,} \PY{n}{dt\PYZus{}score3} \PY{o}{=} \PY{n}{dt\PYZus{}1}\PY{p}{(}\PY{l+m+mi}{1}\PY{p}{,} \PY{n}{X\PYZus{}train}\PY{p}{,} \PY{n}{y\PYZus{}train}\PY{p}{,} \PY{n}{my\PYZus{}scaler}\PY{o}{=}\PY{n}{RobustScaler}\PY{p}{,} \PY{n}{jobs}\PY{o}{=}\PY{o}{\PYZhy{}}\PY{l+m+mi}{3}\PY{p}{,} \PY{n}{gs\PYZus{}scoring}\PY{o}{=}\PY{l+s+s1}{\PYZsq{}}\PY{l+s+s1}{f1}\PY{l+s+s1}{\PYZsq{}}\PY{p}{,} \PY{n}{folts}\PY{o}{=}\PY{l+m+mi}{5}\PY{p}{,} \PY{n}{max\PYZus{}features}\PY{o}{=}\PY{k+kc}{None}\PY{p}{)}
\PY{n}{dt\PYZus{}score3}
\end{Verbatim}
\end{tcolorbox}

    \begin{Verbatim}[commandchars=\\\{\}]
[Parallel(n\_jobs=-3)]: Using backend LokyBackend with 10 concurrent workers.
    \end{Verbatim}

    \begin{Verbatim}[commandchars=\\\{\}]
Fitting 5 folds for each of 1 candidates, totalling 5 fits
    \end{Verbatim}

    \begin{Verbatim}[commandchars=\\\{\}]
[Parallel(n\_jobs=-3)]: Done   5 out of   5 | elapsed:    0.0s finished
    \end{Verbatim}

            \begin{tcolorbox}[breakable, size=fbox, boxrule=.5pt, pad at break*=1mm, opacityfill=0]
\prompt{Out}{outcolor}{16}{\boxspacing}
\begin{Verbatim}[commandchars=\\\{\}]
0.836113675945373
\end{Verbatim}
\end{tcolorbox}
        
    \hypertarget{scoring-model-1-modell-2}{%
\paragraph{Scoring Model 1 / Modell 2}\label{scoring-model-1-modell-2}}

    \begin{tcolorbox}[breakable, size=fbox, boxrule=1pt, pad at break*=1mm,colback=cellbackground, colframe=cellborder]
\prompt{In}{incolor}{17}{\boxspacing}
\begin{Verbatim}[commandchars=\\\{\}]
\PY{n}{model} \PY{o}{=} \PY{n}{dt\PYZus{}model}
\PY{n+nb}{print}\PY{p}{(}\PY{n}{model}\PY{o}{.}\PY{n}{best\PYZus{}params\PYZus{}}\PY{p}{)}

\PY{n}{model} \PY{o}{=} \PY{n}{dt\PYZus{}model2}
\PY{n+nb}{print}\PY{p}{(}\PY{n}{model}\PY{o}{.}\PY{n}{best\PYZus{}params\PYZus{}}\PY{p}{)}

\PY{n}{model} \PY{o}{=} \PY{n}{dt\PYZus{}model3}
\PY{n+nb}{print}\PY{p}{(}\PY{n}{model}\PY{o}{.}\PY{n}{best\PYZus{}params\PYZus{}}\PY{p}{)}
\end{Verbatim}
\end{tcolorbox}

    \begin{Verbatim}[commandchars=\\\{\}]
\{'classifier': DecisionTreeClassifier(class\_weight=None, criterion='gini',
max\_depth=None,
                       max\_features=None, max\_leaf\_nodes=None,
                       min\_impurity\_decrease=0.0, min\_impurity\_split=None,
                       min\_samples\_leaf=1, min\_samples\_split=2,
                       min\_weight\_fraction\_leaf=0.0, presort=False,
                       random\_state=2111, splitter='best'),
'classifier\_\_max\_features': None, 'classifier\_\_random\_state': 2111\}
\{'classifier': DecisionTreeClassifier(class\_weight=None, criterion='gini',
max\_depth=None,
                       max\_features=None, max\_leaf\_nodes=None,
                       min\_impurity\_decrease=0.0, min\_impurity\_split=None,
                       min\_samples\_leaf=1, min\_samples\_split=2,
                       min\_weight\_fraction\_leaf=0.0, presort=False,
                       random\_state=2111, splitter='best'),
'classifier\_\_max\_features': None, 'classifier\_\_random\_state': 2111\}
\{'classifier': DecisionTreeClassifier(class\_weight=None, criterion='gini',
max\_depth=None,
                       max\_features=None, max\_leaf\_nodes=None,
                       min\_impurity\_decrease=0.0, min\_impurity\_split=None,
                       min\_samples\_leaf=1, min\_samples\_split=2,
                       min\_weight\_fraction\_leaf=0.0, presort=False,
                       random\_state=2111, splitter='best'),
'classifier\_\_max\_features': None, 'classifier\_\_random\_state': 2111\}
    \end{Verbatim}

    \begin{tcolorbox}[breakable, size=fbox, boxrule=1pt, pad at break*=1mm,colback=cellbackground, colframe=cellborder]
\prompt{In}{incolor}{18}{\boxspacing}
\begin{Verbatim}[commandchars=\\\{\}]
\PY{n}{model} \PY{o}{=} \PY{n}{dt\PYZus{}model3}
\PY{n}{c} \PY{o}{=} \PY{l+m+mf}{0.35}

\PY{n}{class\PYZus{}names} \PY{o}{=} \PY{p}{[}\PY{l+s+s1}{\PYZsq{}}\PY{l+s+s1}{Stückgut fehlerfrei}\PY{l+s+s1}{\PYZsq{}}\PY{p}{,} \PY{l+s+s1}{\PYZsq{}}\PY{l+s+s1}{Stückgut fehlerhaft}\PY{l+s+s1}{\PYZsq{}}\PY{p}{]}
\PY{n}{confusion\PYZus{}matrix1}\PY{p}{,} \PY{n}{report1} \PY{o}{=} \PY{n}{get\PYZus{}confusion\PYZus{}matrix}\PY{p}{(}\PY{n}{X\PYZus{}train}\PY{p}{,} \PY{n}{y\PYZus{}train}\PY{p}{,} \PY{n}{model}\PY{p}{,} \PY{n}{class\PYZus{}names}\PY{p}{,} \PY{n}{c}\PY{p}{)}
\PY{n}{confusion\PYZus{}matrix1}
\end{Verbatim}
\end{tcolorbox}

            \begin{tcolorbox}[breakable, size=fbox, boxrule=.5pt, pad at break*=1mm, opacityfill=0]
\prompt{Out}{outcolor}{18}{\boxspacing}
\begin{Verbatim}[commandchars=\\\{\}]
                                Prognose des Modells
                                 Stückgut fehlerfrei Stückgut fehlerhaft
Wahrer Wert Stückgut fehlerfrei                16168                   0
            Stückgut fehlerhaft                    0                 225
\end{Verbatim}
\end{tcolorbox}
        
    \begin{tcolorbox}[breakable, size=fbox, boxrule=1pt, pad at break*=1mm,colback=cellbackground, colframe=cellborder]
\prompt{In}{incolor}{19}{\boxspacing}
\begin{Verbatim}[commandchars=\\\{\}]
\PY{n}{roc\PYZus{}curve\PYZus{}func}\PY{p}{(}\PY{n}{X\PYZus{}train}\PY{p}{,} \PY{n}{y\PYZus{}train}\PY{p}{,} \PY{n}{model}\PY{p}{)}
\end{Verbatim}
\end{tcolorbox}

    \begin{center}
    \adjustimage{max size={0.9\linewidth}{0.9\paperheight}}{output_32_0.png}
    \end{center}
    { \hspace*{\fill} \\}
    
    \begin{tcolorbox}[breakable, size=fbox, boxrule=1pt, pad at break*=1mm,colback=cellbackground, colframe=cellborder]
\prompt{In}{incolor}{20}{\boxspacing}
\begin{Verbatim}[commandchars=\\\{\}]
\PY{n+nb}{print}\PY{p}{(}\PY{n}{report1}\PY{p}{)}
\end{Verbatim}
\end{tcolorbox}

    \begin{Verbatim}[commandchars=\\\{\}]
              precision    recall  f1-score   support

           0       1.00      1.00      1.00     16168
           1       1.00      1.00      1.00       225

    accuracy                           1.00     16393
   macro avg       1.00      1.00      1.00     16393
weighted avg       1.00      1.00      1.00     16393

    \end{Verbatim}

    \hypertarget{scoring-auf-validerungsdatensatz}{%
\paragraph{Scoring auf
Validerungsdatensatz}\label{scoring-auf-validerungsdatensatz}}

    \begin{tcolorbox}[breakable, size=fbox, boxrule=1pt, pad at break*=1mm,colback=cellbackground, colframe=cellborder]
\prompt{In}{incolor}{21}{\boxspacing}
\begin{Verbatim}[commandchars=\\\{\}]
\PY{n}{model} \PY{o}{=} \PY{n}{dt\PYZus{}model3}
\PY{n}{c} \PY{o}{=} \PY{l+m+mf}{0.35}

\PY{n}{class\PYZus{}names} \PY{o}{=} \PY{p}{[}\PY{l+s+s1}{\PYZsq{}}\PY{l+s+s1}{Stückgut fehlerfrei}\PY{l+s+s1}{\PYZsq{}}\PY{p}{,} \PY{l+s+s1}{\PYZsq{}}\PY{l+s+s1}{Stückgut fehlerhaft}\PY{l+s+s1}{\PYZsq{}}\PY{p}{]}
\PY{n}{confusion\PYZus{}matrix2}\PY{p}{,} \PY{n}{report2} \PY{o}{=} \PY{n}{get\PYZus{}confusion\PYZus{}matrix}\PY{p}{(}\PY{n}{X\PYZus{}validierung}\PY{p}{,} \PY{n}{y\PYZus{}validierung}\PY{p}{,} \PY{n}{model}\PY{p}{,} \PY{n}{class\PYZus{}names}\PY{p}{,} \PY{n}{c}\PY{p}{)}
\PY{n}{confusion\PYZus{}matrix2}
\end{Verbatim}
\end{tcolorbox}

            \begin{tcolorbox}[breakable, size=fbox, boxrule=.5pt, pad at break*=1mm, opacityfill=0]
\prompt{Out}{outcolor}{21}{\boxspacing}
\begin{Verbatim}[commandchars=\\\{\}]
                                Prognose des Modells
                                 Stückgut fehlerfrei Stückgut fehlerhaft
Wahrer Wert Stückgut fehlerfrei                 4032                   8
            Stückgut fehlerhaft                   14                  45
\end{Verbatim}
\end{tcolorbox}
        
    \begin{tcolorbox}[breakable, size=fbox, boxrule=1pt, pad at break*=1mm,colback=cellbackground, colframe=cellborder]
\prompt{In}{incolor}{22}{\boxspacing}
\begin{Verbatim}[commandchars=\\\{\}]
\PY{n}{roc\PYZus{}curve\PYZus{}func}\PY{p}{(}\PY{n}{X\PYZus{}validierung}\PY{p}{,} \PY{n}{y\PYZus{}validierung}\PY{p}{,} \PY{n}{model}\PY{p}{)}
\end{Verbatim}
\end{tcolorbox}

    \begin{center}
    \adjustimage{max size={0.9\linewidth}{0.9\paperheight}}{output_36_0.png}
    \end{center}
    { \hspace*{\fill} \\}
    
    \begin{tcolorbox}[breakable, size=fbox, boxrule=1pt, pad at break*=1mm,colback=cellbackground, colframe=cellborder]
\prompt{In}{incolor}{23}{\boxspacing}
\begin{Verbatim}[commandchars=\\\{\}]
\PY{n+nb}{print}\PY{p}{(}\PY{n}{report2}\PY{p}{)}
\end{Verbatim}
\end{tcolorbox}

    \begin{Verbatim}[commandchars=\\\{\}]
              precision    recall  f1-score   support

           0       1.00      1.00      1.00      4040
           1       0.85      0.76      0.80        59

    accuracy                           0.99      4099
   macro avg       0.92      0.88      0.90      4099
weighted avg       0.99      0.99      0.99      4099

    \end{Verbatim}

    \hypertarget{submit}{%
\subsection{Submit}\label{submit}}

\hypertarget{kontrolle-modellwahl-modell-1-oder-2-anhand-der-konfusionsmatrix}{%
\subsubsection{Kontrolle Modellwahl (Modell 1 oder 2) anhand der
Konfusionsmatrix}\label{kontrolle-modellwahl-modell-1-oder-2-anhand-der-konfusionsmatrix}}

    \begin{tcolorbox}[breakable, size=fbox, boxrule=1pt, pad at break*=1mm,colback=cellbackground, colframe=cellborder]
\prompt{In}{incolor}{24}{\boxspacing}
\begin{Verbatim}[commandchars=\\\{\}]
\PY{l+s+sd}{\PYZdq{}\PYZdq{}\PYZdq{}}
\PY{l+s+sd}{model = dt\PYZus{}model3}
\PY{l+s+sd}{c = 0.35}

\PY{l+s+sd}{class\PYZus{}names = [\PYZsq{}Stückgut fehlerfrei\PYZsq{}, \PYZsq{}Stückgut fehlerhaft\PYZsq{}]}
\PY{l+s+sd}{confusion\PYZus{}matrix3, report3 = get\PYZus{}confusion\PYZus{}matrix(X\PYZus{}validierung, y\PYZus{}validierung, model, class\PYZus{}names, c)}
\PY{l+s+sd}{confusion\PYZus{}matrix3}
\PY{l+s+sd}{\PYZdq{}\PYZdq{}\PYZdq{}}
\end{Verbatim}
\end{tcolorbox}

            \begin{tcolorbox}[breakable, size=fbox, boxrule=.5pt, pad at break*=1mm, opacityfill=0]
\prompt{Out}{outcolor}{24}{\boxspacing}
\begin{Verbatim}[commandchars=\\\{\}]
"\textbackslash{}nmodel = dt\_model3\textbackslash{}nc = 0.35\textbackslash{}n\textbackslash{}nclass\_names = ['Stückgut fehlerfrei',
'Stückgut fehlerhaft']\textbackslash{}nconfusion\_matrix3, report3 =
get\_confusion\_matrix(X\_validierung, y\_validierung, model, class\_names,
c)\textbackslash{}nconfusion\_matrix3\textbackslash{}n"
\end{Verbatim}
\end{tcolorbox}
        
    \hypertarget{submit-der-prognose}{%
\subsubsection{Submit der Prognose}\label{submit-der-prognose}}

    \begin{tcolorbox}[breakable, size=fbox, boxrule=1pt, pad at break*=1mm,colback=cellbackground, colframe=cellborder]
\prompt{In}{incolor}{25}{\boxspacing}
\begin{Verbatim}[commandchars=\\\{\}]
\PY{l+s+sd}{\PYZdq{}\PYZdq{}\PYZdq{}}
\PY{l+s+sd}{submission\PYZus{}head, submission\PYZus{}fehlerhaft = submit(model, c, save=True, manu\PYZus{}name=True)}
\PY{l+s+sd}{submission\PYZus{}head}
\PY{l+s+sd}{\PYZdq{}\PYZdq{}\PYZdq{}}
\end{Verbatim}
\end{tcolorbox}

            \begin{tcolorbox}[breakable, size=fbox, boxrule=.5pt, pad at break*=1mm, opacityfill=0]
\prompt{Out}{outcolor}{25}{\boxspacing}
\begin{Verbatim}[commandchars=\\\{\}]
'\textbackslash{}nsubmission\_head, submission\_fehlerhaft = submit(model, c, save=True,
manu\_name=True)\textbackslash{}nsubmission\_head\textbackslash{}n'
\end{Verbatim}
\end{tcolorbox}
        
    \hypertarget{ausgabe-dataframe-mit-als-defekt-klassifizierten-stuxfcckguxfctern-im-testdatensatz}{%
\subsubsection{Ausgabe DataFrame mit als defekt klassifizierten
Stückgütern im
Testdatensatz}\label{ausgabe-dataframe-mit-als-defekt-klassifizierten-stuxfcckguxfctern-im-testdatensatz}}

    \begin{tcolorbox}[breakable, size=fbox, boxrule=1pt, pad at break*=1mm,colback=cellbackground, colframe=cellborder]
\prompt{In}{incolor}{26}{\boxspacing}
\begin{Verbatim}[commandchars=\\\{\}]
\PY{l+s+sd}{\PYZdq{}\PYZdq{}\PYZdq{}}
\PY{l+s+sd}{submission\PYZus{}fehlerhaft}
\PY{l+s+sd}{\PYZdq{}\PYZdq{}\PYZdq{}}
\end{Verbatim}
\end{tcolorbox}

            \begin{tcolorbox}[breakable, size=fbox, boxrule=.5pt, pad at break*=1mm, opacityfill=0]
\prompt{Out}{outcolor}{26}{\boxspacing}
\begin{Verbatim}[commandchars=\\\{\}]
'\textbackslash{}nsubmission\_fehlerhaft\textbackslash{}n'
\end{Verbatim}
\end{tcolorbox}
        

    % Add a bibliography block to the postdoc
    
    
    
\end{document}
