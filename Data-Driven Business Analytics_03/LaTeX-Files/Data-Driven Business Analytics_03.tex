\documentclass[paper=landscape]{scrartcl}
	\usepackage[paperwidth=40cm,paperheight=500cm,margin=1in]{geometry}


    \usepackage[breakable]{tcolorbox}
    \usepackage{parskip} % Stop auto-indenting (to mimic markdown behaviour)
    
    \usepackage{iftex}
    \ifPDFTeX
    	\usepackage[T1]{fontenc}
    	\usepackage{mathpazo}
    \else
    	\usepackage{fontspec}
    \fi

    % Basic figure setup, for now with no caption control since it's done
    % automatically by Pandoc (which extracts ![](path) syntax from Markdown).
    \usepackage{graphicx}
    % Maintain compatibility with old templates. Remove in nbconvert 6.0
    \let\Oldincludegraphics\includegraphics
    % Ensure that by default, figures have no caption (until we provide a
    % proper Figure object with a Caption API and a way to capture that
    % in the conversion process - todo).
    \usepackage{caption}
    \DeclareCaptionFormat{nocaption}{}
    \captionsetup{format=nocaption,aboveskip=0pt,belowskip=0pt}

    \usepackage[Export]{adjustbox} % Used to constrain images to a maximum size
    \adjustboxset{max size={0.9\linewidth}{0.9\paperheight}}
    \usepackage{float}
    \floatplacement{figure}{H} % forces figures to be placed at the correct location
    \usepackage{xcolor} % Allow colors to be defined
    \usepackage{enumerate} % Needed for markdown enumerations to work
    \usepackage{geometry} % Used to adjust the document margins
    \usepackage{amsmath} % Equations
    \usepackage{amssymb} % Equations
    \usepackage{textcomp} % defines textquotesingle
    % Hack from http://tex.stackexchange.com/a/47451/13684:
    \AtBeginDocument{%
        \def\PYZsq{\textquotesingle}% Upright quotes in Pygmentized code
    }
    \usepackage{upquote} % Upright quotes for verbatim code
    \usepackage{eurosym} % defines \euro
    \usepackage[mathletters]{ucs} % Extended unicode (utf-8) support
    \usepackage{fancyvrb} % verbatim replacement that allows latex
    \usepackage{grffile} % extends the file name processing of package graphics 
                         % to support a larger range
    \makeatletter % fix for grffile with XeLaTeX
    \def\Gread@@xetex#1{%
      \IfFileExists{"\Gin@base".bb}%
      {\Gread@eps{\Gin@base.bb}}%
      {\Gread@@xetex@aux#1}%
    }
    \makeatother

    % The hyperref package gives us a pdf with properly built
    % internal navigation ('pdf bookmarks' for the table of contents,
    % internal cross-reference links, web links for URLs, etc.)
    \usepackage{hyperref}
    % The default LaTeX title has an obnoxious amount of whitespace. By default,
    % titling removes some of it. It also provides customization options.
    \usepackage{titling}
    \usepackage{longtable} % longtable support required by pandoc >1.10
    \usepackage{booktabs}  % table support for pandoc > 1.12.2
    \usepackage[inline]{enumitem} % IRkernel/repr support (it uses the enumerate* environment)
    \usepackage[normalem]{ulem} % ulem is needed to support strikethroughs (\sout)
                                % normalem makes italics be italics, not underlines
    \usepackage{mathrsfs}
    

    
    % Colors for the hyperref package
    \definecolor{urlcolor}{rgb}{0,.145,.698}
    \definecolor{linkcolor}{rgb}{.71,0.21,0.01}
    \definecolor{citecolor}{rgb}{.12,.54,.11}

    % ANSI colors
    \definecolor{ansi-black}{HTML}{3E424D}
    \definecolor{ansi-black-intense}{HTML}{282C36}
    \definecolor{ansi-red}{HTML}{E75C58}
    \definecolor{ansi-red-intense}{HTML}{B22B31}
    \definecolor{ansi-green}{HTML}{00A250}
    \definecolor{ansi-green-intense}{HTML}{007427}
    \definecolor{ansi-yellow}{HTML}{DDB62B}
    \definecolor{ansi-yellow-intense}{HTML}{B27D12}
    \definecolor{ansi-blue}{HTML}{208FFB}
    \definecolor{ansi-blue-intense}{HTML}{0065CA}
    \definecolor{ansi-magenta}{HTML}{D160C4}
    \definecolor{ansi-magenta-intense}{HTML}{A03196}
    \definecolor{ansi-cyan}{HTML}{60C6C8}
    \definecolor{ansi-cyan-intense}{HTML}{258F8F}
    \definecolor{ansi-white}{HTML}{C5C1B4}
    \definecolor{ansi-white-intense}{HTML}{A1A6B2}
    \definecolor{ansi-default-inverse-fg}{HTML}{FFFFFF}
    \definecolor{ansi-default-inverse-bg}{HTML}{000000}

    % commands and environments needed by pandoc snippets
    % extracted from the output of `pandoc -s`
    \providecommand{\tightlist}{%
      \setlength{\itemsep}{0pt}\setlength{\parskip}{0pt}}
    \DefineVerbatimEnvironment{Highlighting}{Verbatim}{commandchars=\\\{\}}
    % Add ',fontsize=\small' for more characters per line
    \newenvironment{Shaded}{}{}
    \newcommand{\KeywordTok}[1]{\textcolor[rgb]{0.00,0.44,0.13}{\textbf{{#1}}}}
    \newcommand{\DataTypeTok}[1]{\textcolor[rgb]{0.56,0.13,0.00}{{#1}}}
    \newcommand{\DecValTok}[1]{\textcolor[rgb]{0.25,0.63,0.44}{{#1}}}
    \newcommand{\BaseNTok}[1]{\textcolor[rgb]{0.25,0.63,0.44}{{#1}}}
    \newcommand{\FloatTok}[1]{\textcolor[rgb]{0.25,0.63,0.44}{{#1}}}
    \newcommand{\CharTok}[1]{\textcolor[rgb]{0.25,0.44,0.63}{{#1}}}
    \newcommand{\StringTok}[1]{\textcolor[rgb]{0.25,0.44,0.63}{{#1}}}
    \newcommand{\CommentTok}[1]{\textcolor[rgb]{0.38,0.63,0.69}{\textit{{#1}}}}
    \newcommand{\OtherTok}[1]{\textcolor[rgb]{0.00,0.44,0.13}{{#1}}}
    \newcommand{\AlertTok}[1]{\textcolor[rgb]{1.00,0.00,0.00}{\textbf{{#1}}}}
    \newcommand{\FunctionTok}[1]{\textcolor[rgb]{0.02,0.16,0.49}{{#1}}}
    \newcommand{\RegionMarkerTok}[1]{{#1}}
    \newcommand{\ErrorTok}[1]{\textcolor[rgb]{1.00,0.00,0.00}{\textbf{{#1}}}}
    \newcommand{\NormalTok}[1]{{#1}}
    
    % Additional commands for more recent versions of Pandoc
    \newcommand{\ConstantTok}[1]{\textcolor[rgb]{0.53,0.00,0.00}{{#1}}}
    \newcommand{\SpecialCharTok}[1]{\textcolor[rgb]{0.25,0.44,0.63}{{#1}}}
    \newcommand{\VerbatimStringTok}[1]{\textcolor[rgb]{0.25,0.44,0.63}{{#1}}}
    \newcommand{\SpecialStringTok}[1]{\textcolor[rgb]{0.73,0.40,0.53}{{#1}}}
    \newcommand{\ImportTok}[1]{{#1}}
    \newcommand{\DocumentationTok}[1]{\textcolor[rgb]{0.73,0.13,0.13}{\textit{{#1}}}}
    \newcommand{\AnnotationTok}[1]{\textcolor[rgb]{0.38,0.63,0.69}{\textbf{\textit{{#1}}}}}
    \newcommand{\CommentVarTok}[1]{\textcolor[rgb]{0.38,0.63,0.69}{\textbf{\textit{{#1}}}}}
    \newcommand{\VariableTok}[1]{\textcolor[rgb]{0.10,0.09,0.49}{{#1}}}
    \newcommand{\ControlFlowTok}[1]{\textcolor[rgb]{0.00,0.44,0.13}{\textbf{{#1}}}}
    \newcommand{\OperatorTok}[1]{\textcolor[rgb]{0.40,0.40,0.40}{{#1}}}
    \newcommand{\BuiltInTok}[1]{{#1}}
    \newcommand{\ExtensionTok}[1]{{#1}}
    \newcommand{\PreprocessorTok}[1]{\textcolor[rgb]{0.74,0.48,0.00}{{#1}}}
    \newcommand{\AttributeTok}[1]{\textcolor[rgb]{0.49,0.56,0.16}{{#1}}}
    \newcommand{\InformationTok}[1]{\textcolor[rgb]{0.38,0.63,0.69}{\textbf{\textit{{#1}}}}}
    \newcommand{\WarningTok}[1]{\textcolor[rgb]{0.38,0.63,0.69}{\textbf{\textit{{#1}}}}}
    
    
    % Define a nice break command that doesn't care if a line doesn't already
    % exist.
    \def\br{\hspace*{\fill} \\* }
    % Math Jax compatibility definitions
    \def\gt{>}
    \def\lt{<}
    \let\Oldtex\TeX
    \let\Oldlatex\LaTeX
    \renewcommand{\TeX}{\textrm{\Oldtex}}
    \renewcommand{\LaTeX}{\textrm{\Oldlatex}}
    % Document parameters
    % Document title
    \title{Data-Driven Business Analytics\_03}
    
    
    
    
    
% Pygments definitions
\makeatletter
\def\PY@reset{\let\PY@it=\relax \let\PY@bf=\relax%
    \let\PY@ul=\relax \let\PY@tc=\relax%
    \let\PY@bc=\relax \let\PY@ff=\relax}
\def\PY@tok#1{\csname PY@tok@#1\endcsname}
\def\PY@toks#1+{\ifx\relax#1\empty\else%
    \PY@tok{#1}\expandafter\PY@toks\fi}
\def\PY@do#1{\PY@bc{\PY@tc{\PY@ul{%
    \PY@it{\PY@bf{\PY@ff{#1}}}}}}}
\def\PY#1#2{\PY@reset\PY@toks#1+\relax+\PY@do{#2}}

\expandafter\def\csname PY@tok@w\endcsname{\def\PY@tc##1{\textcolor[rgb]{0.73,0.73,0.73}{##1}}}
\expandafter\def\csname PY@tok@c\endcsname{\let\PY@it=\textit\def\PY@tc##1{\textcolor[rgb]{0.25,0.50,0.50}{##1}}}
\expandafter\def\csname PY@tok@cp\endcsname{\def\PY@tc##1{\textcolor[rgb]{0.74,0.48,0.00}{##1}}}
\expandafter\def\csname PY@tok@k\endcsname{\let\PY@bf=\textbf\def\PY@tc##1{\textcolor[rgb]{0.00,0.50,0.00}{##1}}}
\expandafter\def\csname PY@tok@kp\endcsname{\def\PY@tc##1{\textcolor[rgb]{0.00,0.50,0.00}{##1}}}
\expandafter\def\csname PY@tok@kt\endcsname{\def\PY@tc##1{\textcolor[rgb]{0.69,0.00,0.25}{##1}}}
\expandafter\def\csname PY@tok@o\endcsname{\def\PY@tc##1{\textcolor[rgb]{0.40,0.40,0.40}{##1}}}
\expandafter\def\csname PY@tok@ow\endcsname{\let\PY@bf=\textbf\def\PY@tc##1{\textcolor[rgb]{0.67,0.13,1.00}{##1}}}
\expandafter\def\csname PY@tok@nb\endcsname{\def\PY@tc##1{\textcolor[rgb]{0.00,0.50,0.00}{##1}}}
\expandafter\def\csname PY@tok@nf\endcsname{\def\PY@tc##1{\textcolor[rgb]{0.00,0.00,1.00}{##1}}}
\expandafter\def\csname PY@tok@nc\endcsname{\let\PY@bf=\textbf\def\PY@tc##1{\textcolor[rgb]{0.00,0.00,1.00}{##1}}}
\expandafter\def\csname PY@tok@nn\endcsname{\let\PY@bf=\textbf\def\PY@tc##1{\textcolor[rgb]{0.00,0.00,1.00}{##1}}}
\expandafter\def\csname PY@tok@ne\endcsname{\let\PY@bf=\textbf\def\PY@tc##1{\textcolor[rgb]{0.82,0.25,0.23}{##1}}}
\expandafter\def\csname PY@tok@nv\endcsname{\def\PY@tc##1{\textcolor[rgb]{0.10,0.09,0.49}{##1}}}
\expandafter\def\csname PY@tok@no\endcsname{\def\PY@tc##1{\textcolor[rgb]{0.53,0.00,0.00}{##1}}}
\expandafter\def\csname PY@tok@nl\endcsname{\def\PY@tc##1{\textcolor[rgb]{0.63,0.63,0.00}{##1}}}
\expandafter\def\csname PY@tok@ni\endcsname{\let\PY@bf=\textbf\def\PY@tc##1{\textcolor[rgb]{0.60,0.60,0.60}{##1}}}
\expandafter\def\csname PY@tok@na\endcsname{\def\PY@tc##1{\textcolor[rgb]{0.49,0.56,0.16}{##1}}}
\expandafter\def\csname PY@tok@nt\endcsname{\let\PY@bf=\textbf\def\PY@tc##1{\textcolor[rgb]{0.00,0.50,0.00}{##1}}}
\expandafter\def\csname PY@tok@nd\endcsname{\def\PY@tc##1{\textcolor[rgb]{0.67,0.13,1.00}{##1}}}
\expandafter\def\csname PY@tok@s\endcsname{\def\PY@tc##1{\textcolor[rgb]{0.73,0.13,0.13}{##1}}}
\expandafter\def\csname PY@tok@sd\endcsname{\let\PY@it=\textit\def\PY@tc##1{\textcolor[rgb]{0.73,0.13,0.13}{##1}}}
\expandafter\def\csname PY@tok@si\endcsname{\let\PY@bf=\textbf\def\PY@tc##1{\textcolor[rgb]{0.73,0.40,0.53}{##1}}}
\expandafter\def\csname PY@tok@se\endcsname{\let\PY@bf=\textbf\def\PY@tc##1{\textcolor[rgb]{0.73,0.40,0.13}{##1}}}
\expandafter\def\csname PY@tok@sr\endcsname{\def\PY@tc##1{\textcolor[rgb]{0.73,0.40,0.53}{##1}}}
\expandafter\def\csname PY@tok@ss\endcsname{\def\PY@tc##1{\textcolor[rgb]{0.10,0.09,0.49}{##1}}}
\expandafter\def\csname PY@tok@sx\endcsname{\def\PY@tc##1{\textcolor[rgb]{0.00,0.50,0.00}{##1}}}
\expandafter\def\csname PY@tok@m\endcsname{\def\PY@tc##1{\textcolor[rgb]{0.40,0.40,0.40}{##1}}}
\expandafter\def\csname PY@tok@gh\endcsname{\let\PY@bf=\textbf\def\PY@tc##1{\textcolor[rgb]{0.00,0.00,0.50}{##1}}}
\expandafter\def\csname PY@tok@gu\endcsname{\let\PY@bf=\textbf\def\PY@tc##1{\textcolor[rgb]{0.50,0.00,0.50}{##1}}}
\expandafter\def\csname PY@tok@gd\endcsname{\def\PY@tc##1{\textcolor[rgb]{0.63,0.00,0.00}{##1}}}
\expandafter\def\csname PY@tok@gi\endcsname{\def\PY@tc##1{\textcolor[rgb]{0.00,0.63,0.00}{##1}}}
\expandafter\def\csname PY@tok@gr\endcsname{\def\PY@tc##1{\textcolor[rgb]{1.00,0.00,0.00}{##1}}}
\expandafter\def\csname PY@tok@ge\endcsname{\let\PY@it=\textit}
\expandafter\def\csname PY@tok@gs\endcsname{\let\PY@bf=\textbf}
\expandafter\def\csname PY@tok@gp\endcsname{\let\PY@bf=\textbf\def\PY@tc##1{\textcolor[rgb]{0.00,0.00,0.50}{##1}}}
\expandafter\def\csname PY@tok@go\endcsname{\def\PY@tc##1{\textcolor[rgb]{0.53,0.53,0.53}{##1}}}
\expandafter\def\csname PY@tok@gt\endcsname{\def\PY@tc##1{\textcolor[rgb]{0.00,0.27,0.87}{##1}}}
\expandafter\def\csname PY@tok@err\endcsname{\def\PY@bc##1{\setlength{\fboxsep}{0pt}\fcolorbox[rgb]{1.00,0.00,0.00}{1,1,1}{\strut ##1}}}
\expandafter\def\csname PY@tok@kc\endcsname{\let\PY@bf=\textbf\def\PY@tc##1{\textcolor[rgb]{0.00,0.50,0.00}{##1}}}
\expandafter\def\csname PY@tok@kd\endcsname{\let\PY@bf=\textbf\def\PY@tc##1{\textcolor[rgb]{0.00,0.50,0.00}{##1}}}
\expandafter\def\csname PY@tok@kn\endcsname{\let\PY@bf=\textbf\def\PY@tc##1{\textcolor[rgb]{0.00,0.50,0.00}{##1}}}
\expandafter\def\csname PY@tok@kr\endcsname{\let\PY@bf=\textbf\def\PY@tc##1{\textcolor[rgb]{0.00,0.50,0.00}{##1}}}
\expandafter\def\csname PY@tok@bp\endcsname{\def\PY@tc##1{\textcolor[rgb]{0.00,0.50,0.00}{##1}}}
\expandafter\def\csname PY@tok@fm\endcsname{\def\PY@tc##1{\textcolor[rgb]{0.00,0.00,1.00}{##1}}}
\expandafter\def\csname PY@tok@vc\endcsname{\def\PY@tc##1{\textcolor[rgb]{0.10,0.09,0.49}{##1}}}
\expandafter\def\csname PY@tok@vg\endcsname{\def\PY@tc##1{\textcolor[rgb]{0.10,0.09,0.49}{##1}}}
\expandafter\def\csname PY@tok@vi\endcsname{\def\PY@tc##1{\textcolor[rgb]{0.10,0.09,0.49}{##1}}}
\expandafter\def\csname PY@tok@vm\endcsname{\def\PY@tc##1{\textcolor[rgb]{0.10,0.09,0.49}{##1}}}
\expandafter\def\csname PY@tok@sa\endcsname{\def\PY@tc##1{\textcolor[rgb]{0.73,0.13,0.13}{##1}}}
\expandafter\def\csname PY@tok@sb\endcsname{\def\PY@tc##1{\textcolor[rgb]{0.73,0.13,0.13}{##1}}}
\expandafter\def\csname PY@tok@sc\endcsname{\def\PY@tc##1{\textcolor[rgb]{0.73,0.13,0.13}{##1}}}
\expandafter\def\csname PY@tok@dl\endcsname{\def\PY@tc##1{\textcolor[rgb]{0.73,0.13,0.13}{##1}}}
\expandafter\def\csname PY@tok@s2\endcsname{\def\PY@tc##1{\textcolor[rgb]{0.73,0.13,0.13}{##1}}}
\expandafter\def\csname PY@tok@sh\endcsname{\def\PY@tc##1{\textcolor[rgb]{0.73,0.13,0.13}{##1}}}
\expandafter\def\csname PY@tok@s1\endcsname{\def\PY@tc##1{\textcolor[rgb]{0.73,0.13,0.13}{##1}}}
\expandafter\def\csname PY@tok@mb\endcsname{\def\PY@tc##1{\textcolor[rgb]{0.40,0.40,0.40}{##1}}}
\expandafter\def\csname PY@tok@mf\endcsname{\def\PY@tc##1{\textcolor[rgb]{0.40,0.40,0.40}{##1}}}
\expandafter\def\csname PY@tok@mh\endcsname{\def\PY@tc##1{\textcolor[rgb]{0.40,0.40,0.40}{##1}}}
\expandafter\def\csname PY@tok@mi\endcsname{\def\PY@tc##1{\textcolor[rgb]{0.40,0.40,0.40}{##1}}}
\expandafter\def\csname PY@tok@il\endcsname{\def\PY@tc##1{\textcolor[rgb]{0.40,0.40,0.40}{##1}}}
\expandafter\def\csname PY@tok@mo\endcsname{\def\PY@tc##1{\textcolor[rgb]{0.40,0.40,0.40}{##1}}}
\expandafter\def\csname PY@tok@ch\endcsname{\let\PY@it=\textit\def\PY@tc##1{\textcolor[rgb]{0.25,0.50,0.50}{##1}}}
\expandafter\def\csname PY@tok@cm\endcsname{\let\PY@it=\textit\def\PY@tc##1{\textcolor[rgb]{0.25,0.50,0.50}{##1}}}
\expandafter\def\csname PY@tok@cpf\endcsname{\let\PY@it=\textit\def\PY@tc##1{\textcolor[rgb]{0.25,0.50,0.50}{##1}}}
\expandafter\def\csname PY@tok@c1\endcsname{\let\PY@it=\textit\def\PY@tc##1{\textcolor[rgb]{0.25,0.50,0.50}{##1}}}
\expandafter\def\csname PY@tok@cs\endcsname{\let\PY@it=\textit\def\PY@tc##1{\textcolor[rgb]{0.25,0.50,0.50}{##1}}}

\def\PYZbs{\char`\\}
\def\PYZus{\char`\_}
\def\PYZob{\char`\{}
\def\PYZcb{\char`\}}
\def\PYZca{\char`\^}
\def\PYZam{\char`\&}
\def\PYZlt{\char`\<}
\def\PYZgt{\char`\>}
\def\PYZsh{\char`\#}
\def\PYZpc{\char`\%}
\def\PYZdl{\char`\$}
\def\PYZhy{\char`\-}
\def\PYZsq{\char`\'}
\def\PYZdq{\char`\"}
\def\PYZti{\char`\~}
% for compatibility with earlier versions
\def\PYZat{@}
\def\PYZlb{[}
\def\PYZrb{]}
\makeatother


    % For linebreaks inside Verbatim environment from package fancyvrb. 
    \makeatletter
        \newbox\Wrappedcontinuationbox 
        \newbox\Wrappedvisiblespacebox 
        \newcommand*\Wrappedvisiblespace {\textcolor{red}{\textvisiblespace}} 
        \newcommand*\Wrappedcontinuationsymbol {\textcolor{red}{\llap{\tiny$\m@th\hookrightarrow$}}} 
        \newcommand*\Wrappedcontinuationindent {3ex } 
        \newcommand*\Wrappedafterbreak {\kern\Wrappedcontinuationindent\copy\Wrappedcontinuationbox} 
        % Take advantage of the already applied Pygments mark-up to insert 
        % potential linebreaks for TeX processing. 
        %        {, <, #, %, $, ' and ": go to next line. 
        %        _, }, ^, &, >, - and ~: stay at end of broken line. 
        % Use of \textquotesingle for straight quote. 
        \newcommand*\Wrappedbreaksatspecials {% 
            \def\PYGZus{\discretionary{\char`\_}{\Wrappedafterbreak}{\char`\_}}% 
            \def\PYGZob{\discretionary{}{\Wrappedafterbreak\char`\{}{\char`\{}}% 
            \def\PYGZcb{\discretionary{\char`\}}{\Wrappedafterbreak}{\char`\}}}% 
            \def\PYGZca{\discretionary{\char`\^}{\Wrappedafterbreak}{\char`\^}}% 
            \def\PYGZam{\discretionary{\char`\&}{\Wrappedafterbreak}{\char`\&}}% 
            \def\PYGZlt{\discretionary{}{\Wrappedafterbreak\char`\<}{\char`\<}}% 
            \def\PYGZgt{\discretionary{\char`\>}{\Wrappedafterbreak}{\char`\>}}% 
            \def\PYGZsh{\discretionary{}{\Wrappedafterbreak\char`\#}{\char`\#}}% 
            \def\PYGZpc{\discretionary{}{\Wrappedafterbreak\char`\%}{\char`\%}}% 
            \def\PYGZdl{\discretionary{}{\Wrappedafterbreak\char`\$}{\char`\$}}% 
            \def\PYGZhy{\discretionary{\char`\-}{\Wrappedafterbreak}{\char`\-}}% 
            \def\PYGZsq{\discretionary{}{\Wrappedafterbreak\textquotesingle}{\textquotesingle}}% 
            \def\PYGZdq{\discretionary{}{\Wrappedafterbreak\char`\"}{\char`\"}}% 
            \def\PYGZti{\discretionary{\char`\~}{\Wrappedafterbreak}{\char`\~}}% 
        } 
        % Some characters . , ; ? ! / are not pygmentized. 
        % This macro makes them "active" and they will insert potential linebreaks 
        \newcommand*\Wrappedbreaksatpunct {% 
            \lccode`\~`\.\lowercase{\def~}{\discretionary{\hbox{\char`\.}}{\Wrappedafterbreak}{\hbox{\char`\.}}}% 
            \lccode`\~`\,\lowercase{\def~}{\discretionary{\hbox{\char`\,}}{\Wrappedafterbreak}{\hbox{\char`\,}}}% 
            \lccode`\~`\;\lowercase{\def~}{\discretionary{\hbox{\char`\;}}{\Wrappedafterbreak}{\hbox{\char`\;}}}% 
            \lccode`\~`\:\lowercase{\def~}{\discretionary{\hbox{\char`\:}}{\Wrappedafterbreak}{\hbox{\char`\:}}}% 
            \lccode`\~`\?\lowercase{\def~}{\discretionary{\hbox{\char`\?}}{\Wrappedafterbreak}{\hbox{\char`\?}}}% 
            \lccode`\~`\!\lowercase{\def~}{\discretionary{\hbox{\char`\!}}{\Wrappedafterbreak}{\hbox{\char`\!}}}% 
            \lccode`\~`\/\lowercase{\def~}{\discretionary{\hbox{\char`\/}}{\Wrappedafterbreak}{\hbox{\char`\/}}}% 
            \catcode`\.\active
            \catcode`\,\active 
            \catcode`\;\active
            \catcode`\:\active
            \catcode`\?\active
            \catcode`\!\active
            \catcode`\/\active 
            \lccode`\~`\~ 	
        }
    \makeatother

    \let\OriginalVerbatim=\Verbatim
    \makeatletter
    \renewcommand{\Verbatim}[1][1]{%
        %\parskip\z@skip
        \sbox\Wrappedcontinuationbox {\Wrappedcontinuationsymbol}%
        \sbox\Wrappedvisiblespacebox {\FV@SetupFont\Wrappedvisiblespace}%
        \def\FancyVerbFormatLine ##1{\hsize\linewidth
            \vtop{\raggedright\hyphenpenalty\z@\exhyphenpenalty\z@
                \doublehyphendemerits\z@\finalhyphendemerits\z@
                \strut ##1\strut}%
        }%
        % If the linebreak is at a space, the latter will be displayed as visible
        % space at end of first line, and a continuation symbol starts next line.
        % Stretch/shrink are however usually zero for typewriter font.
        \def\FV@Space {%
            \nobreak\hskip\z@ plus\fontdimen3\font minus\fontdimen4\font
            \discretionary{\copy\Wrappedvisiblespacebox}{\Wrappedafterbreak}
            {\kern\fontdimen2\font}%
        }%
        
        % Allow breaks at special characters using \PYG... macros.
        \Wrappedbreaksatspecials
        % Breaks at punctuation characters . , ; ? ! and / need catcode=\active 	
        \OriginalVerbatim[#1,codes*=\Wrappedbreaksatpunct]%
    }
    \makeatother

    % Exact colors from NB
    \definecolor{incolor}{HTML}{303F9F}
    \definecolor{outcolor}{HTML}{D84315}
    \definecolor{cellborder}{HTML}{CFCFCF}
    \definecolor{cellbackground}{HTML}{F7F7F7}
    
    % prompt
    \makeatletter
    \newcommand{\boxspacing}{\kern\kvtcb@left@rule\kern\kvtcb@boxsep}
    \makeatother
    \newcommand{\prompt}[4]{
        \ttfamily\llap{{\color{#2}[#3]:\hspace{3pt}#4}}\vspace{-\baselineskip}
    }
    

    
    % Prevent overflowing lines due to hard-to-break entities
    \sloppy 
    % Setup hyperref package
    \hypersetup{
      breaklinks=true,  % so long urls are correctly broken across lines
      colorlinks=true,
      urlcolor=urlcolor,
      linkcolor=linkcolor,
      citecolor=citecolor,
      }
    % Slightly bigger margins than the latex defaults
    
    \geometry{verbose,tmargin=1in,bmargin=1in,lmargin=1in,rmargin=1in}
    
    

\begin{document}
    
    \maketitle
    
    

    
    \hypertarget{projektarbeit---regression-von-huxe4userpreisen-mittels-knn}{%
\section{3. Projektarbeit - Regression von Häuserpreisen mittels
KNN}\label{projektarbeit---regression-von-huxe4userpreisen-mittels-knn}}

© Thomas Robert Holy 2020 Version 1.0 Visit me on GitHub:
https://github.com/trh0ly Kaggle Link:
https://www.kaggle.com/c/dda-p3/leaderboard \#\# Grundlegene
Einstellungen \#\#\# Import der Bibliotheken

    \begin{tcolorbox}[breakable, size=fbox, boxrule=1pt, pad at break*=1mm,colback=cellbackground, colframe=cellborder]
\prompt{In}{incolor}{1}{\boxspacing}
\begin{Verbatim}[commandchars=\\\{\}]
\PY{k+kn}{import} \PY{n+nn}{numpy} \PY{k}{as} \PY{n+nn}{np}
\PY{k+kn}{import} \PY{n+nn}{pandas} \PY{k}{as} \PY{n+nn}{pd}
\PY{k+kn}{import} \PY{n+nn}{matplotlib}\PY{n+nn}{.}\PY{n+nn}{pyplot} \PY{k}{as} \PY{n+nn}{plt}
\PY{k+kn}{from} \PY{n+nn}{sklearn}\PY{n+nn}{.}\PY{n+nn}{compose} \PY{k}{import} \PY{n}{ColumnTransformer}
\PY{k+kn}{from} \PY{n+nn}{sklearn}\PY{n+nn}{.}\PY{n+nn}{preprocessing} \PY{k}{import} \PY{n}{OneHotEncoder}
\PY{k+kn}{from} \PY{n+nn}{sklearn}\PY{n+nn}{.}\PY{n+nn}{preprocessing} \PY{k}{import} \PY{n}{LabelEncoder}
\PY{k+kn}{from} \PY{n+nn}{sklearn}\PY{n+nn}{.}\PY{n+nn}{metrics} \PY{k}{import} \PY{n}{mean\PYZus{}squared\PYZus{}error}
\PY{k+kn}{from} \PY{n+nn}{sklearn} \PY{k}{import} \PY{n}{preprocessing}
\PY{k+kn}{import} \PY{n+nn}{seaborn} \PY{k}{as} \PY{n+nn}{sns}
\PY{k+kn}{from} \PY{n+nn}{keras} \PY{k}{import} \PY{n}{metrics}
\PY{k+kn}{import} \PY{n+nn}{keras}
\PY{k+kn}{import} \PY{n+nn}{sys}
\PY{k+kn}{from} \PY{n+nn}{keras}\PY{n+nn}{.}\PY{n+nn}{models} \PY{k}{import} \PY{n}{Sequential}
\PY{k+kn}{import} \PY{n+nn}{operator}
\PY{k+kn}{from} \PY{n+nn}{keras}\PY{n+nn}{.}\PY{n+nn}{layers} \PY{k}{import} \PY{n}{Dense}
\PY{k+kn}{from} \PY{n+nn}{keras} \PY{k}{import} \PY{n}{optimizers} 
\PY{k+kn}{from} \PY{n+nn}{keras} \PY{k}{import} \PY{n}{losses}
\PY{k+kn}{from} \PY{n+nn}{keras} \PY{k}{import} \PY{n}{activations}
\PY{k+kn}{from} \PY{n+nn}{keras}\PY{n+nn}{.}\PY{n+nn}{constraints} \PY{k}{import} \PY{n}{maxnorm}
\PY{k+kn}{from} \PY{n+nn}{keras}\PY{n+nn}{.}\PY{n+nn}{wrappers}\PY{n+nn}{.}\PY{n+nn}{scikit\PYZus{}learn} \PY{k}{import} \PY{n}{KerasRegressor}
\PY{k+kn}{from} \PY{n+nn}{keras}\PY{n+nn}{.}\PY{n+nn}{layers} \PY{k}{import} \PY{n}{BatchNormalization}
\PY{k+kn}{from} \PY{n+nn}{sklearn}\PY{n+nn}{.}\PY{n+nn}{model\PYZus{}selection} \PY{k}{import} \PY{n}{KFold}
\PY{k+kn}{from} \PY{n+nn}{sklearn}\PY{n+nn}{.}\PY{n+nn}{model\PYZus{}selection} \PY{k}{import} \PY{n}{GridSearchCV}
\PY{k+kn}{from} \PY{n+nn}{keras}\PY{n+nn}{.}\PY{n+nn}{layers} \PY{k}{import} \PY{n}{Dropout}
\PY{k+kn}{from} \PY{n+nn}{keras} \PY{k}{import} \PY{n}{regularizers}
\PY{k+kn}{from} \PY{n+nn}{sklearn}\PY{n+nn}{.}\PY{n+nn}{preprocessing} \PY{k}{import} \PY{n}{RobustScaler}
\PY{k+kn}{from} \PY{n+nn}{sklearn}\PY{n+nn}{.}\PY{n+nn}{metrics} \PY{k}{import} \PY{n}{mean\PYZus{}squared\PYZus{}error}
\PY{k+kn}{from} \PY{n+nn}{sklearn}\PY{n+nn}{.}\PY{n+nn}{model\PYZus{}selection} \PY{k}{import} \PY{n}{train\PYZus{}test\PYZus{}split}
\PY{k+kn}{from} \PY{n+nn}{IPython}\PY{n+nn}{.}\PY{n+nn}{core}\PY{n+nn}{.}\PY{n+nn}{display} \PY{k}{import} \PY{n}{display}\PY{p}{,} \PY{n}{HTML}
\end{Verbatim}
\end{tcolorbox}

    \begin{Verbatim}[commandchars=\\\{\}]
Using TensorFlow backend.
    \end{Verbatim}

    \hypertarget{optikeinstellungen}{%
\subsubsection{Optikeinstellungen}\label{optikeinstellungen}}

    \begin{tcolorbox}[breakable, size=fbox, boxrule=1pt, pad at break*=1mm,colback=cellbackground, colframe=cellborder]
\prompt{In}{incolor}{2}{\boxspacing}
\begin{Verbatim}[commandchars=\\\{\}]
\PY{o}{\PYZpc{}\PYZpc{}javascript}
\PY{n+nx}{IPython}\PY{p}{.}\PY{n+nx}{OutputArea}\PY{p}{.}\PY{n+nx}{auto\PYZus{}scroll\PYZus{}threshold} \PY{o}{=} \PY{l+m+mi}{9999}\PY{p}{;}
\end{Verbatim}
\end{tcolorbox}

    
    \begin{verbatim}
<IPython.core.display.Javascript object>
    \end{verbatim}

    
    \begin{tcolorbox}[breakable, size=fbox, boxrule=1pt, pad at break*=1mm,colback=cellbackground, colframe=cellborder]
\prompt{In}{incolor}{3}{\boxspacing}
\begin{Verbatim}[commandchars=\\\{\}]
\PY{n}{display}\PY{p}{(}\PY{n}{HTML}\PY{p}{(}\PY{l+s+s2}{\PYZdq{}}\PY{l+s+s2}{\PYZlt{}style\PYZgt{}.container }\PY{l+s+s2}{\PYZob{}}\PY{l+s+s2}{ width:100}\PY{l+s+s2}{\PYZpc{}}\PY{l+s+s2}{ !important; \PYZcb{}\PYZlt{}/style\PYZgt{}}\PY{l+s+s2}{\PYZdq{}}\PY{p}{)}\PY{p}{)}
\PY{n}{pd}\PY{o}{.}\PY{n}{options}\PY{o}{.}\PY{n}{display}\PY{o}{.}\PY{n}{width} \PY{o}{=} \PY{l+m+mi}{500}
\PY{n}{pd}\PY{o}{.}\PY{n}{options}\PY{o}{.}\PY{n}{display}\PY{o}{.}\PY{n}{max\PYZus{}columns} \PY{o}{=} \PY{l+m+mi}{999}
\PY{n}{pd}\PY{o}{.}\PY{n}{options}\PY{o}{.}\PY{n}{display}\PY{o}{.}\PY{n}{max\PYZus{}rows} \PY{o}{=} \PY{l+m+mi}{999}
\PY{n}{plt}\PY{o}{.}\PY{n}{rcParams}\PY{p}{[}\PY{l+s+s1}{\PYZsq{}}\PY{l+s+s1}{figure.figsize}\PY{l+s+s1}{\PYZsq{}}\PY{p}{]} \PY{o}{=} \PY{p}{(}\PY{l+m+mi}{12}\PY{p}{,} \PY{l+m+mi}{6}\PY{p}{)}
\PY{n}{SCREEN\PYZus{}WIDTH} \PY{o}{=} \PY{l+m+mi}{100}
\PY{n}{centered} \PY{o}{=} \PY{n}{operator}\PY{o}{.}\PY{n}{methodcaller}\PY{p}{(}\PY{l+s+s1}{\PYZsq{}}\PY{l+s+s1}{center}\PY{l+s+s1}{\PYZsq{}}\PY{p}{,} \PY{n}{SCREEN\PYZus{}WIDTH}\PY{p}{)}
\end{Verbatim}
\end{tcolorbox}

    
    \begin{verbatim}
<IPython.core.display.HTML object>
    \end{verbatim}

    
    \hypertarget{definition-von-hilfsfunktionen}{%
\subsection{Definition von
Hilfsfunktionen}\label{definition-von-hilfsfunktionen}}

\hypertarget{definition-one-hot-encoding-funktion}{%
\subsubsection{Definition One-Hot-Encoding
Funktion}\label{definition-one-hot-encoding-funktion}}

    \begin{tcolorbox}[breakable, size=fbox, boxrule=1pt, pad at break*=1mm,colback=cellbackground, colframe=cellborder]
\prompt{In}{incolor}{4}{\boxspacing}
\begin{Verbatim}[commandchars=\\\{\}]
\PY{c+c1}{\PYZsh{} Definition einer Funktion, die das One\PYZhy{}Hot\PYZhy{}Encoding durchführt}
\PY{c+c1}{\PYZsh{}\PYZhy{}\PYZhy{}\PYZhy{}\PYZhy{}\PYZhy{}\PYZhy{}\PYZhy{}\PYZhy{}\PYZhy{}\PYZhy{}\PYZhy{}\PYZhy{}}
\PY{c+c1}{\PYZsh{} Argumente:}
\PY{c+c1}{\PYZsh{} \PYZhy{} df: DataFrame welcher bearbeitet werden soll}
\PY{c+c1}{\PYZsh{} \PYZhy{} df\PYZus{}row: DataFrame\PYZhy{}Spalte die bearbeitet werden soll}
\PY{c+c1}{\PYZsh{} \PYZhy{} make\PYZus{}df: }
\PY{c+c1}{\PYZsh{} \PYZhy{}\PYZhy{}\PYZhy{}\PYZgt{} Wenn False, dann werden die transformierten Werte als Array zurückgegeben}
\PY{c+c1}{\PYZsh{} \PYZhy{}\PYZhy{}\PYZhy{}\PYZgt{} Wenn True, dann wird ein DataFrame zurückgegeben}
\PY{c+c1}{\PYZsh{} \PYZhy{} overwrite\PYZus{}inv\PYZus{}encoded:}
\PY{c+c1}{\PYZsh{} \PYZhy{}\PYZhy{}\PYZhy{}\PYZgt{} Wenn None, dann werden Spaltennamen des DataFrames auf Grundlage der Merkmals\PYZhy{}}
\PY{c+c1}{\PYZsh{}      ausprägungen im originalen DataFrame ermittelt}
\PY{c+c1}{\PYZsh{} \PYZhy{}\PYZhy{}\PYZhy{}\PYZgt{} Wenn != None, dann werden die Spalten manuell überschrieben}
\PY{c+c1}{\PYZsh{}\PYZhy{}\PYZhy{}\PYZhy{}\PYZhy{}\PYZhy{}\PYZhy{}\PYZhy{}\PYZhy{}\PYZhy{}\PYZhy{}\PYZhy{}\PYZhy{}}

\PY{k}{def} \PY{n+nf}{onehot\PYZus{}encoder\PYZus{}func}\PY{p}{(}\PY{n}{df}\PY{p}{,} \PY{n}{df\PYZus{}row}\PY{p}{,} \PY{n}{make\PYZus{}df}\PY{o}{=}\PY{k+kc}{False}\PY{p}{,} \PY{n}{overwrite\PYZus{}inv\PYZus{}encoded}\PY{o}{=}\PY{k+kc}{None}\PY{p}{)}\PY{p}{:}
    \PY{n}{values} \PY{o}{=} \PY{n}{np}\PY{o}{.}\PY{n}{array}\PY{p}{(}\PY{n}{df}\PY{p}{[}\PY{n}{df\PYZus{}row}\PY{p}{]}\PY{p}{)} \PY{c+c1}{\PYZsh{} Transformation der DataFrame\PYZhy{}Saplte in ein Numpy Array}
    \PY{n}{label\PYZus{}encoder} \PY{o}{=} \PY{n}{LabelEncoder}\PY{p}{(}\PY{p}{)} \PY{c+c1}{\PYZsh{} Definition des Label\PYZhy{}Encoders}
    \PY{n}{integer\PYZus{}encoded} \PY{o}{=} \PY{n}{label\PYZus{}encoder}\PY{o}{.}\PY{n}{fit\PYZus{}transform}\PY{p}{(}\PY{n}{values}\PY{o}{.}\PY{n}{ravel}\PY{p}{(}\PY{p}{)}\PY{p}{)} \PY{c+c1}{\PYZsh{} Label\PYZhy{}Encoder auf Values fitten}
    \PY{n}{onehot\PYZus{}encoder} \PY{o}{=} \PY{n}{OneHotEncoder}\PY{p}{(}\PY{n}{sparse}\PY{o}{=}\PY{k+kc}{False}\PY{p}{,} \PY{n}{handle\PYZus{}unknown}\PY{o}{=}\PY{l+s+s1}{\PYZsq{}}\PY{l+s+s1}{ignore}\PY{l+s+s1}{\PYZsq{}}\PY{p}{)} \PY{c+c1}{\PYZsh{} Definition des One\PYZhy{}Hot\PYZhy{}Encoders}
    \PY{n}{integer\PYZus{}encoded} \PY{o}{=} \PY{n}{integer\PYZus{}encoded}\PY{o}{.}\PY{n}{reshape}\PY{p}{(}\PY{n+nb}{len}\PY{p}{(}\PY{n}{integer\PYZus{}encoded}\PY{p}{)}\PY{p}{,} \PY{l+m+mi}{1}\PY{p}{)} \PY{c+c1}{\PYZsh{} Reshape der Integer\PYZhy{}Encoded Values}
    \PY{n}{onehot\PYZus{}encoded} \PY{o}{=} \PY{n}{onehot\PYZus{}encoder}\PY{o}{.}\PY{n}{fit\PYZus{}transform}\PY{p}{(}\PY{n}{integer\PYZus{}encoded}\PY{p}{)} \PY{c+c1}{\PYZsh{} One\PYZhy{}Hot\PYZhy{}Encoder auf Values fitten}
    \PY{l+s+sd}{\PYZdq{}\PYZdq{}\PYZdq{}}
\PY{l+s+sd}{    Wenn make\PYZus{}df == False:}
\PY{l+s+sd}{    Rückgabe der One\PYZhy{}Hot\PYZhy{}Encoded }
\PY{l+s+sd}{    Values als Array}
\PY{l+s+sd}{    \PYZdq{}\PYZdq{}\PYZdq{}}
    \PY{k}{if} \PY{n}{make\PYZus{}df} \PY{o}{==} \PY{k+kc}{False}\PY{p}{:}
        \PY{k}{return} \PY{n}{onehot\PYZus{}encoded} \PY{c+c1}{\PYZsh{} Rückgabe One\PYZhy{}Hot\PYZhy{}Endcoded Array}
    \PY{l+s+sd}{\PYZdq{}\PYZdq{}\PYZdq{}}
\PY{l+s+sd}{    Wenn make\PYZus{}df == True:}
\PY{l+s+sd}{    Rückgabe der One\PYZhy{}Hot\PYZhy{}Encoded }
\PY{l+s+sd}{    Values als DataFrame}
\PY{l+s+sd}{    \PYZdq{}\PYZdq{}\PYZdq{}}
    \PY{k}{if} \PY{n}{make\PYZus{}df} \PY{o}{==} \PY{k+kc}{True}\PY{p}{:}
        \PY{l+s+sd}{\PYZdq{}\PYZdq{}\PYZdq{}}
\PY{l+s+sd}{        Wenn overwrite\PYZus{}inv\PYZus{}encoded == None:}
\PY{l+s+sd}{        Spalten des DataFrame werden nicht}
\PY{l+s+sd}{        manuell überschrieben. Die Spaltennamen}
\PY{l+s+sd}{        werden aus den Mermalsausprägungen des}
\PY{l+s+sd}{        originalen DataFrame gewonnen.}
\PY{l+s+sd}{        \PYZdq{}\PYZdq{}\PYZdq{}}
        \PY{k}{if} \PY{n}{overwrite\PYZus{}inv\PYZus{}encoded} \PY{o}{==} \PY{k+kc}{None}\PY{p}{:}
            \PY{n}{counter}\PY{p}{,} \PY{n}{array} \PY{o}{=} \PY{l+m+mi}{0}\PY{p}{,} \PY{p}{[}\PY{p}{]} \PY{c+c1}{\PYZsh{} Zähler, Main\PYZhy{}Array werden inizialisiert}
            \PY{c+c1}{\PYZsh{} For\PYZhy{}Schleife, welche ein Arrays mit Arrays generiert, welche}
            \PY{c+c1}{\PYZsh{} von i=0 bis i=len(onehot\PYZus{}encoded) jeweils eine 1 enthalten}
            \PY{k}{for} \PY{n}{i} \PY{o+ow}{in} \PY{n+nb}{range}\PY{p}{(}\PY{l+m+mi}{0}\PY{p}{,} \PY{n+nb}{len}\PY{p}{(}\PY{n}{onehot\PYZus{}encoded}\PY{p}{[}\PY{l+m+mi}{0}\PY{p}{]}\PY{p}{)}\PY{p}{)}\PY{p}{:}
                \PY{n}{temp} \PY{o}{=} \PY{p}{[}\PY{l+m+mi}{0}\PY{p}{]} \PY{o}{*} \PY{n+nb}{len}\PY{p}{(}\PY{n}{onehot\PYZus{}encoded}\PY{p}{[}\PY{l+m+mi}{0}\PY{p}{]}\PY{p}{)} \PY{c+c1}{\PYZsh{} Temporäres Array}
                \PY{n}{temp}\PY{p}{[}\PY{n}{i}\PY{p}{]} \PY{o}{=} \PY{l+m+mi}{1} \PY{c+c1}{\PYZsh{} Variable i im temporären Array wird 1 gesetzt}
                \PY{n}{array}\PY{o}{.}\PY{n}{append}\PY{p}{(}\PY{n}{temp}\PY{p}{)} \PY{c+c1}{\PYZsh{} Generiertes Array wird dem Main\PYZhy{}Array angefügt}
            \PY{n}{inv\PYZus{}encoded} \PY{o}{=} \PY{n}{onehot\PYZus{}encoder}\PY{o}{.}\PY{n}{inverse\PYZus{}transform}\PY{p}{(}\PY{n}{array}\PY{p}{)} \PY{c+c1}{\PYZsh{} Inverse Transformation One\PYZhy{}Hot\PYZhy{}Encoder}
            \PY{n}{inv\PYZus{}encoded} \PY{o}{=} \PY{n}{label\PYZus{}encoder}\PY{o}{.}\PY{n}{inverse\PYZus{}transform}\PY{p}{(}\PY{n}{inv\PYZus{}encoded}\PY{o}{.}\PY{n}{astype}\PY{p}{(}\PY{n+nb}{int}\PY{p}{)}\PY{o}{.}\PY{n}{ravel}\PY{p}{(}\PY{p}{)}\PY{p}{)} \PY{c+c1}{\PYZsh{} Inverse Transformation Label\PYZhy{}Encoder }
            \PY{c+c1}{\PYZsh{} Generierung des DataFrames mit den One\PYZhy{}Hot\PYZhy{}Encoded Values und originalen Merkmalsausprägungen als Spalten\PYZhy{}Namen}
            \PY{n}{encoded\PYZus{}df} \PY{o}{=} \PY{n}{pd}\PY{o}{.}\PY{n}{DataFrame}\PY{p}{(}\PY{n}{onehot\PYZus{}encoded}\PY{p}{,} \PY{n}{dtype}\PY{o}{=}\PY{n+nb}{float}\PY{p}{,} \PY{n}{columns}\PY{o}{=}\PY{n+nb}{list}\PY{p}{(}\PY{n}{inv\PYZus{}encoded}\PY{p}{)}\PY{p}{,} \PY{n}{index}\PY{o}{=}\PY{n}{df}\PY{o}{.}\PY{n}{index}\PY{p}{)}  
            \PY{n}{df\PYZus{}ids} \PY{o}{=} \PY{n}{df}\PY{o}{.}\PY{n}{Id}\PY{o}{.}\PY{n}{values}\PY{o}{.}\PY{n}{tolist}\PY{p}{(}\PY{p}{)} \PY{c+c1}{\PYZsh{} Indexspalte bzw. Einträge extrahieren        }
            \PY{n}{encoded\PYZus{}df}\PY{p}{[}\PY{l+s+s1}{\PYZsq{}}\PY{l+s+s1}{Id}\PY{l+s+s1}{\PYZsq{}}\PY{p}{]} \PY{o}{=} \PY{n}{df\PYZus{}ids} \PY{c+c1}{\PYZsh{} Im encoded\PYZus{}df eine Indexspalteanlegen...}
            \PY{n}{encoded\PYZus{}df} \PY{o}{=} \PY{n}{encoded\PYZus{}df}\PY{o}{.}\PY{n}{set\PYZus{}index}\PY{p}{(}\PY{l+s+s1}{\PYZsq{}}\PY{l+s+s1}{Id}\PY{l+s+s1}{\PYZsq{}}\PY{p}{)} \PY{c+c1}{\PYZsh{} ... und als Index serzen}
            \PY{k}{return} \PY{n}{encoded\PYZus{}df} \PY{c+c1}{\PYZsh{} Rückgabe DataFrame}
        \PY{l+s+sd}{\PYZdq{}\PYZdq{}\PYZdq{}}
\PY{l+s+sd}{        Wenn overwrite\PYZus{}inv\PYZus{}encoded != None:}
\PY{l+s+sd}{        Spalten des DataFrame werden manuell}
\PY{l+s+sd}{        überschrieben sofern Übergebene Array}
\PY{l+s+sd}{        genügend Spaltennamen enthält.}
\PY{l+s+sd}{        \PYZdq{}\PYZdq{}\PYZdq{}}
        \PY{k}{if} \PY{n}{overwrite\PYZus{}inv\PYZus{}encoded} \PY{o}{!=} \PY{k+kc}{None}\PY{p}{:}
            \PY{c+c1}{\PYZsh{} Prüfung ob die Länge des manuell festgelegten Arrays gleich }
            \PY{c+c1}{\PYZsh{} der Länge eines Arrays ist, welches aus den One\PYZhy{}Hot\PYZhy{}Endcoded}
            \PY{c+c1}{\PYZsh{} Values resultiert.}
            \PY{n}{case} \PY{o}{=} \PY{n+nb}{len}\PY{p}{(}\PY{n}{overwrite\PYZus{}inv\PYZus{}encoded}\PY{p}{)} \PY{o}{==} \PY{p}{(}\PY{n+nb}{len}\PY{p}{(}\PY{n}{onehot\PYZus{}encoded}\PY{p}{[}\PY{l+m+mi}{0}\PY{p}{]}\PY{p}{)}\PY{p}{)}
            \PY{l+s+sd}{\PYZdq{}\PYZdq{}\PYZdq{}}
\PY{l+s+sd}{            \PYZsh{} Sofern case = True werden die Spalten}
\PY{l+s+sd}{            wie gewünscht überschrieben}
\PY{l+s+sd}{            \PYZdq{}\PYZdq{}\PYZdq{}}
            \PY{k}{if} \PY{n}{case} \PY{o}{==} \PY{k+kc}{True}\PY{p}{:}            
                \PY{c+c1}{\PYZsh{} Generierung des DataFrames mit den One\PYZhy{}Hot\PYZhy{}Encoded Values und manuell überschriebenen Spalten\PYZhy{}Namen}
                \PY{n}{encoded\PYZus{}df} \PY{o}{=} \PY{n}{pd}\PY{o}{.}\PY{n}{DataFrame}\PY{p}{(}\PY{n}{onehot\PYZus{}encoded}\PY{p}{,} \PY{n}{dtype}\PY{o}{=}\PY{n+nb}{float}\PY{p}{,} \PY{n}{columns}\PY{o}{=}\PY{n+nb}{list}\PY{p}{(}\PY{n}{overwrite\PYZus{}inv\PYZus{}encoded}\PY{p}{)}\PY{p}{,} \PY{n}{index}\PY{o}{=}\PY{n}{df}\PY{o}{.}\PY{n}{index}\PY{p}{)}
                \PY{k}{return} \PY{n}{encoded\PYZus{}df} \PY{c+c1}{\PYZsh{} Rückgabe DataFrame}
                \PY{l+s+sd}{\PYZdq{}\PYZdq{}\PYZdq{}}
\PY{l+s+sd}{                ERROR MESSAGE sofern \PYZdq{}overwrite\PYZus{}inv\PYZus{}encoded\PYZdq{}}
\PY{l+s+sd}{                nicht lang genug.}
\PY{l+s+sd}{                \PYZdq{}\PYZdq{}\PYZdq{}}
            \PY{k}{else}\PY{p}{:} 
                \PY{n}{ERROR} \PY{o}{=} \PY{l+s+s1}{\PYZsq{}}\PY{l+s+s1}{ERROR Len of }\PY{l+s+s1}{\PYZdq{}}\PY{l+s+s1}{overwrite\PYZus{}inv\PYZus{}encoded}\PY{l+s+s1}{\PYZdq{}}\PY{l+s+s1}{ have to be }\PY{l+s+si}{\PYZob{}\PYZcb{}}\PY{l+s+s1}{!}\PY{l+s+s1}{\PYZsq{}}\PY{o}{.}\PY{n}{format}\PY{p}{(}\PY{n+nb}{len}\PY{p}{(}\PY{n}{onehot\PYZus{}encoded}\PY{p}{[}\PY{l+m+mi}{0}\PY{p}{]}\PY{p}{)}\PY{p}{)}
                \PY{k}{return} \PY{n}{ERROR}
        \PY{l+s+sd}{\PYZdq{}\PYZdq{}\PYZdq{}   }
\PY{l+s+sd}{        ERROR MESSAGE sofern \PYZdq{}make\PYZus{}df\PYZdq{}}
\PY{l+s+sd}{        kein gültiges Argument erhält.}
\PY{l+s+sd}{        \PYZdq{}\PYZdq{}\PYZdq{}}
    \PY{k}{else}\PY{p}{:}
        \PY{n+nb}{print}\PY{p}{(}\PY{l+s+s1}{\PYZsq{}}\PY{l+s+s1}{ERROR: }\PY{l+s+s1}{\PYZdq{}}\PY{l+s+s1}{make\PYZus{}df}\PY{l+s+s1}{\PYZdq{}}\PY{l+s+s1}{ needs an Argument!}\PY{l+s+s1}{\PYZsq{}}\PY{p}{)}
        
\end{Verbatim}
\end{tcolorbox}

    \hypertarget{funktion-zur-durchfuxfchrung-des-encodings-und-joinen-der-teil-dataframes}{%
\subsubsection{Funktion zur Durchführung des Encodings und joinen der
Teil-DataFrames}\label{funktion-zur-durchfuxfchrung-des-encodings-und-joinen-der-teil-dataframes}}

    \begin{tcolorbox}[breakable, size=fbox, boxrule=1pt, pad at break*=1mm,colback=cellbackground, colframe=cellborder]
\prompt{In}{incolor}{5}{\boxspacing}
\begin{Verbatim}[commandchars=\\\{\}]
\PY{c+c1}{\PYZsh{} Definition einer Funktion, die einen DataFrame und eine Liste als Argumente erhält und anschließend}
\PY{c+c1}{\PYZsh{} für jedes Element der Liste das One\PYZhy{}Hot\PYZhy{}Enconding im DataFrame durchführt.}
\PY{c+c1}{\PYZsh{}\PYZhy{}\PYZhy{}\PYZhy{}\PYZhy{}\PYZhy{}\PYZhy{}\PYZhy{}\PYZhy{}\PYZhy{}\PYZhy{}\PYZhy{}\PYZhy{}}
\PY{c+c1}{\PYZsh{} Argumente:}
\PY{c+c1}{\PYZsh{} \PYZhy{} df1: DataFrame welcher bearbeitet werden soll}
\PY{c+c1}{\PYZsh{} \PYZhy{} liste\PYZus{} Liste, die abgearbeitet werden soll}
\PY{c+c1}{\PYZsh{}\PYZhy{}\PYZhy{}\PYZhy{}\PYZhy{}\PYZhy{}\PYZhy{}\PYZhy{}\PYZhy{}\PYZhy{}\PYZhy{}\PYZhy{}\PYZhy{}}

\PY{k}{def} \PY{n+nf}{encode}\PY{p}{(}\PY{n}{df1}\PY{p}{,} \PY{n}{liste}\PY{p}{)}\PY{p}{:}
    \PY{c+c1}{\PYZsh{}\PYZhy{}\PYZhy{}\PYZhy{}\PYZhy{}\PYZhy{}\PYZhy{}\PYZhy{}\PYZhy{}\PYZhy{}\PYZhy{}\PYZhy{}\PYZhy{}\PYZhy{}\PYZhy{}\PYZhy{}\PYZhy{}\PYZhy{}\PYZhy{}\PYZhy{}\PYZhy{}\PYZhy{}\PYZhy{}\PYZhy{}\PYZhy{}\PYZhy{}\PYZhy{}\PYZhy{}\PYZhy{}\PYZhy{}\PYZhy{}\PYZhy{}\PYZhy{}}
    \PY{c+c1}{\PYZsh{} Kopie des originalen DataFrames anlegen und Spalte Id als Index setzen}
    \PY{n}{df2} \PY{o}{=} \PY{n}{df1}\PY{o}{.}\PY{n}{copy}\PY{p}{(}\PY{p}{)} 
    \PY{n}{df2} \PY{o}{=} \PY{n}{df2}\PY{o}{.}\PY{n}{set\PYZus{}index}\PY{p}{(}\PY{l+s+s1}{\PYZsq{}}\PY{l+s+s1}{Id}\PY{l+s+s1}{\PYZsq{}}\PY{p}{)}   
    \PY{c+c1}{\PYZsh{}\PYZhy{}\PYZhy{}\PYZhy{}\PYZhy{}\PYZhy{}\PYZhy{}\PYZhy{}\PYZhy{}\PYZhy{}\PYZhy{}\PYZhy{}\PYZhy{}\PYZhy{}\PYZhy{}\PYZhy{}\PYZhy{}\PYZhy{}\PYZhy{}\PYZhy{}\PYZhy{}\PYZhy{}\PYZhy{}\PYZhy{}\PYZhy{}\PYZhy{}\PYZhy{}\PYZhy{}\PYZhy{}\PYZhy{}\PYZhy{}\PYZhy{}\PYZhy{}}
    \PY{c+c1}{\PYZsh{} One\PYZhy{}Hote\PYZhy{}Encoding auf Grundlage der Datentypliste mit Objekten durchführen}
    \PY{n}{onehot\PYZus{}encoded\PYZus{}df\PYZus{}list} \PY{o}{=} \PY{p}{[}\PY{p}{]}
    \PY{k}{for} \PY{n}{i} \PY{o+ow}{in} \PY{n}{liste}\PY{p}{:}
        \PY{c+c1}{\PYZsh{} Versuche ein One\PYZhy{}Hot\PYZhy{}Encoding durchzuführen}
        \PY{k}{try}\PY{p}{:}
            \PY{c+c1}{\PYZsh{}\PYZhy{}\PYZhy{}\PYZhy{}\PYZhy{}\PYZhy{}\PYZhy{}\PYZhy{}\PYZhy{}\PYZhy{}\PYZhy{}\PYZhy{}\PYZhy{}\PYZhy{}\PYZhy{}\PYZhy{}\PYZhy{}\PYZhy{}\PYZhy{}\PYZhy{}\PYZhy{}\PYZhy{}\PYZhy{}\PYZhy{}\PYZhy{}\PYZhy{}\PYZhy{}\PYZhy{}\PYZhy{}\PYZhy{}\PYZhy{}\PYZhy{}\PYZhy{}}
            \PY{c+c1}{\PYZsh{} Führe das Encoding durch und füge die Werte einer Liste an, dann entferne die originale Spalte     }
            \PY{n}{onehot\PYZus{}encoded} \PY{o}{=} \PY{n}{onehot\PYZus{}encoder\PYZus{}func}\PY{p}{(}\PY{n}{df1}\PY{p}{,} \PY{n+nb}{str}\PY{p}{(}\PY{n}{i}\PY{p}{)}\PY{p}{,} \PY{n}{make\PYZus{}df}\PY{o}{=}\PY{k+kc}{True}\PY{p}{)}
            \PY{n}{onehot\PYZus{}encoded\PYZus{}df\PYZus{}list}\PY{o}{.}\PY{n}{append}\PY{p}{(}\PY{n}{onehot\PYZus{}encoded}\PY{p}{)}            
            \PY{n}{df2} \PY{o}{=} \PY{n}{df2}\PY{o}{.}\PY{n}{drop}\PY{p}{(}\PY{n+nb}{str}\PY{p}{(}\PY{n}{i}\PY{p}{)}\PY{p}{,} \PY{n}{axis}\PY{o}{=}\PY{l+m+mi}{1}\PY{p}{)}   
        \PY{c+c1}{\PYZsh{}\PYZhy{}\PYZhy{}\PYZhy{}\PYZhy{}\PYZhy{}\PYZhy{}\PYZhy{}\PYZhy{}\PYZhy{}\PYZhy{}\PYZhy{}\PYZhy{}\PYZhy{}\PYZhy{}\PYZhy{}\PYZhy{}\PYZhy{}\PYZhy{}\PYZhy{}\PYZhy{}\PYZhy{}\PYZhy{}\PYZhy{}\PYZhy{}\PYZhy{}\PYZhy{}\PYZhy{}\PYZhy{}\PYZhy{}\PYZhy{}\PYZhy{}\PYZhy{}}
        \PY{c+c1}{\PYZsh{} Bei Scheritern Meldung ausgeben und Spalte (zunächst) entferen}
        \PY{k}{except}\PY{p}{:}
            \PY{n+nb}{print}\PY{p}{(}\PY{l+s+s1}{\PYZsq{}}\PY{l+s+s1}{ERROR in }\PY{l+s+si}{\PYZob{}\PYZcb{}}\PY{l+s+s1}{\PYZsq{}}\PY{o}{.}\PY{n}{format}\PY{p}{(}\PY{n}{i}\PY{p}{)}\PY{p}{)}
            \PY{n}{df2} \PY{o}{=} \PY{n}{df2}\PY{o}{.}\PY{n}{drop}\PY{p}{(}\PY{n+nb}{str}\PY{p}{(}\PY{n}{i}\PY{p}{)}\PY{p}{,} \PY{n}{axis}\PY{o}{=}\PY{l+m+mi}{1}\PY{p}{)}
    \PY{c+c1}{\PYZsh{}\PYZhy{}\PYZhy{}\PYZhy{}\PYZhy{}\PYZhy{}\PYZhy{}\PYZhy{}\PYZhy{}\PYZhy{}\PYZhy{}\PYZhy{}\PYZhy{}\PYZhy{}\PYZhy{}\PYZhy{}\PYZhy{}\PYZhy{}\PYZhy{}\PYZhy{}\PYZhy{}\PYZhy{}\PYZhy{}\PYZhy{}\PYZhy{}\PYZhy{}\PYZhy{}\PYZhy{}\PYZhy{}\PYZhy{}\PYZhy{}\PYZhy{}\PYZhy{}}
    \PY{c+c1}{\PYZsh{} Teil\PYZhy{}DataFrames mit One\PYZhy{}Hot\PYZhy{}Endocding\PYZhy{}Werten zusammenführen und ausgeben}
    \PY{n}{counter} \PY{o}{=} \PY{l+m+mi}{0}
    \PY{k}{for} \PY{n}{i} \PY{o+ow}{in} \PY{n}{onehot\PYZus{}encoded\PYZus{}df\PYZus{}list}\PY{p}{:}
        \PY{n}{df2} \PY{o}{=} \PY{n}{df2}\PY{o}{.}\PY{n}{join}\PY{p}{(}\PY{n}{i}\PY{p}{,} \PY{n}{lsuffix}\PY{o}{=}\PY{n+nb}{str}\PY{p}{(}\PY{n}{counter}\PY{p}{)}\PY{p}{)}
        \PY{n}{counter} \PY{o}{+}\PY{o}{=} \PY{l+m+mi}{1}        
    \PY{k}{return} \PY{n}{df2}
\end{Verbatim}
\end{tcolorbox}

    \hypertarget{funktion-zur-datenbereinigung-nan-werte}{%
\subsubsection{Funktion zur Datenbereinigung
(NaN-Werte)}\label{funktion-zur-datenbereinigung-nan-werte}}

    \begin{tcolorbox}[breakable, size=fbox, boxrule=1pt, pad at break*=1mm,colback=cellbackground, colframe=cellborder]
\prompt{In}{incolor}{6}{\boxspacing}
\begin{Verbatim}[commandchars=\\\{\}]
\PY{c+c1}{\PYZsh{} Definition einer Funktion, die eine automatisierte Entferunung von Spalten mit zu vielen NaN\PYZhy{}Werten (über einem festgelegten Schwellenwert)}
\PY{c+c1}{\PYZsh{} vornimmt und automatisch Spalten mit einer vorgegebenen Methode füllt, welche den Schwellenwert nicht überschreitzen.}
\PY{c+c1}{\PYZsh{}\PYZhy{}\PYZhy{}\PYZhy{}\PYZhy{}\PYZhy{}\PYZhy{}\PYZhy{}\PYZhy{}\PYZhy{}\PYZhy{}\PYZhy{}\PYZhy{}}
\PY{c+c1}{\PYZsh{} Argumente:}
\PY{c+c1}{\PYZsh{} \PYZhy{} df: DataFrame welcher bearbeitet werden soll}
\PY{c+c1}{\PYZsh{} \PYZhy{} upper\PYZus{}limit: Schwellenwert ab welchem die Spalte gelöscht werden soll}
\PY{c+c1}{\PYZsh{} \PYZhy{} method: Methode mit der NaN\PYZhy{}Werte in Spalten, welche nicht gelöscht werden, gefüllt werden}
\PY{c+c1}{\PYZsh{}\PYZhy{}\PYZhy{}\PYZhy{}\PYZhy{}\PYZhy{}\PYZhy{}\PYZhy{}\PYZhy{}\PYZhy{}\PYZhy{}\PYZhy{}\PYZhy{}}

\PY{k}{def} \PY{n+nf}{clean\PYZus{}my\PYZus{}data}\PY{p}{(}\PY{n}{df}\PY{p}{,} \PY{n}{upper\PYZus{}limit}\PY{o}{=}\PY{l+m+mi}{30}\PY{p}{)}\PY{p}{:}    
    \PY{c+c1}{\PYZsh{}\PYZhy{}\PYZhy{}\PYZhy{}\PYZhy{}\PYZhy{}\PYZhy{}\PYZhy{}\PYZhy{}\PYZhy{}\PYZhy{}\PYZhy{}\PYZhy{}\PYZhy{}\PYZhy{}\PYZhy{}\PYZhy{}\PYZhy{}\PYZhy{}\PYZhy{}\PYZhy{}\PYZhy{}\PYZhy{}\PYZhy{}\PYZhy{}\PYZhy{}\PYZhy{}\PYZhy{}\PYZhy{}\PYZhy{}\PYZhy{}\PYZhy{}\PYZhy{}}
    \PY{c+c1}{\PYZsh{} Get isnull from dataframe and safe in list    }
    \PY{n}{isnull\PYZus{}list} \PY{o}{=} \PY{n}{df}\PY{o}{.}\PY{n}{isnull}\PY{p}{(}\PY{p}{)}\PY{o}{.}\PY{n}{sum}\PY{p}{(}\PY{p}{)}        
    \PY{c+c1}{\PYZsh{}\PYZhy{}\PYZhy{}\PYZhy{}\PYZhy{}\PYZhy{}\PYZhy{}\PYZhy{}\PYZhy{}\PYZhy{}\PYZhy{}\PYZhy{}\PYZhy{}\PYZhy{}\PYZhy{}\PYZhy{}\PYZhy{}\PYZhy{}\PYZhy{}\PYZhy{}\PYZhy{}\PYZhy{}\PYZhy{}\PYZhy{}\PYZhy{}\PYZhy{}\PYZhy{}\PYZhy{}\PYZhy{}\PYZhy{}\PYZhy{}\PYZhy{}\PYZhy{}}
    \PY{c+c1}{\PYZsh{} get values and index of isnull\PYZus{}list}
    \PY{n}{value} \PY{o}{=} \PY{n}{isnull\PYZus{}list}\PY{o}{.}\PY{n}{values}\PY{o}{.}\PY{n}{tolist}\PY{p}{(}\PY{p}{)}
    \PY{n}{index} \PY{o}{=} \PY{n}{isnull\PYZus{}list}\PY{o}{.}\PY{n}{index}\PY{o}{.}\PY{n}{tolist}\PY{p}{(}\PY{p}{)}    
    \PY{c+c1}{\PYZsh{}\PYZhy{}\PYZhy{}\PYZhy{}\PYZhy{}\PYZhy{}\PYZhy{}\PYZhy{}\PYZhy{}\PYZhy{}\PYZhy{}\PYZhy{}\PYZhy{}\PYZhy{}\PYZhy{}\PYZhy{}\PYZhy{}\PYZhy{}\PYZhy{}\PYZhy{}\PYZhy{}\PYZhy{}\PYZhy{}\PYZhy{}\PYZhy{}\PYZhy{}\PYZhy{}\PYZhy{}\PYZhy{}\PYZhy{}\PYZhy{}\PYZhy{}\PYZhy{}}
    \PY{c+c1}{\PYZsh{} match value and index in one list}
    \PY{n}{isnull\PYZus{}list} \PY{o}{=} \PY{p}{[}\PY{p}{]}
    \PY{k}{for} \PY{n}{i} \PY{o+ow}{in} \PY{n+nb}{range}\PY{p}{(}\PY{l+m+mi}{0}\PY{p}{,} \PY{n+nb}{len}\PY{p}{(}\PY{n}{index}\PY{p}{)}\PY{p}{)}\PY{p}{:}
        \PY{n}{temp} \PY{o}{=} \PY{p}{(}\PY{n}{index}\PY{p}{[}\PY{n}{i}\PY{p}{]}\PY{p}{,} \PY{n}{value}\PY{p}{[}\PY{n}{i}\PY{p}{]}\PY{p}{)}
        \PY{n}{isnull\PYZus{}list}\PY{o}{.}\PY{n}{append}\PY{p}{(}\PY{n}{temp}\PY{p}{)}        
    \PY{c+c1}{\PYZsh{}\PYZhy{}\PYZhy{}\PYZhy{}\PYZhy{}\PYZhy{}\PYZhy{}\PYZhy{}\PYZhy{}\PYZhy{}\PYZhy{}\PYZhy{}\PYZhy{}\PYZhy{}\PYZhy{}\PYZhy{}\PYZhy{}\PYZhy{}\PYZhy{}\PYZhy{}\PYZhy{}\PYZhy{}\PYZhy{}\PYZhy{}\PYZhy{}\PYZhy{}\PYZhy{}\PYZhy{}\PYZhy{}\PYZhy{}\PYZhy{}\PYZhy{}\PYZhy{}}
    \PY{c+c1}{\PYZsh{} sort the list and kick values above upper\PYZus{}limit, safe that as capped\PYZus{}list}
    \PY{n}{isnull\PYZus{}list} \PY{o}{=} \PY{n+nb}{sorted}\PY{p}{(}\PY{n}{isnull\PYZus{}list}\PY{p}{,} \PY{n}{key} \PY{o}{=} \PY{k}{lambda} \PY{n}{x}\PY{p}{:} \PY{n+nb}{abs}\PY{p}{(}\PY{n}{x}\PY{p}{[}\PY{l+m+mi}{1}\PY{p}{]}\PY{p}{)}\PY{p}{,} \PY{n}{reverse}\PY{o}{=}\PY{k+kc}{True}\PY{p}{)}  
    \PY{n}{capped\PYZus{}list} \PY{o}{=} \PY{p}{[}\PY{n}{i} \PY{k}{for} \PY{n}{i} \PY{o+ow}{in} \PY{n}{isnull\PYZus{}list} \PY{k}{if} \PY{n+nb}{abs}\PY{p}{(}\PY{n}{i}\PY{p}{[}\PY{l+m+mi}{1}\PY{p}{]}\PY{p}{)} \PY{o}{\PYZlt{}}\PY{o}{=} \PY{l+m+mi}{30}\PY{p}{]}    
    \PY{c+c1}{\PYZsh{}\PYZhy{}\PYZhy{}\PYZhy{}\PYZhy{}\PYZhy{}\PYZhy{}\PYZhy{}\PYZhy{}\PYZhy{}\PYZhy{}\PYZhy{}\PYZhy{}\PYZhy{}\PYZhy{}\PYZhy{}\PYZhy{}\PYZhy{}\PYZhy{}\PYZhy{}\PYZhy{}\PYZhy{}\PYZhy{}\PYZhy{}\PYZhy{}\PYZhy{}\PYZhy{}\PYZhy{}\PYZhy{}\PYZhy{}\PYZhy{}\PYZhy{}\PYZhy{}}
    \PY{c+c1}{\PYZsh{} get all elements from cappend list which got at least one nan\PYZhy{}value}
    \PY{n}{nan\PYZus{}list} \PY{o}{=} \PY{p}{[}\PY{n}{i} \PY{k}{for} \PY{n}{i} \PY{o+ow}{in} \PY{n}{capped\PYZus{}list} \PY{k}{if} \PY{n+nb}{abs}\PY{p}{(}\PY{n}{i}\PY{p}{[}\PY{l+m+mi}{1}\PY{p}{]}\PY{p}{)} \PY{o}{\PYZgt{}} \PY{l+m+mi}{0}\PY{p}{]}    
    \PY{c+c1}{\PYZsh{}\PYZhy{}\PYZhy{}\PYZhy{}\PYZhy{}\PYZhy{}\PYZhy{}\PYZhy{}\PYZhy{}\PYZhy{}\PYZhy{}\PYZhy{}\PYZhy{}\PYZhy{}\PYZhy{}\PYZhy{}\PYZhy{}\PYZhy{}\PYZhy{}\PYZhy{}\PYZhy{}\PYZhy{}\PYZhy{}\PYZhy{}\PYZhy{}\PYZhy{}\PYZhy{}\PYZhy{}\PYZhy{}\PYZhy{}\PYZhy{}\PYZhy{}\PYZhy{}}
    \PY{c+c1}{\PYZsh{} extrakt the index from nan\PYZus{}list and safe  in to\PYZus{}fix\PYZus{}list}
    \PY{n}{to\PYZus{}fix\PYZus{}list} \PY{o}{=} \PY{p}{[}\PY{p}{]}
    \PY{k}{for} \PY{n}{i} \PY{o+ow}{in} \PY{n}{nan\PYZus{}list}\PY{p}{:}
        \PY{n}{to\PYZus{}fix\PYZus{}list}\PY{o}{.}\PY{n}{append}\PY{p}{(}\PY{n}{i}\PY{p}{[}\PY{l+m+mi}{0}\PY{p}{]}\PY{p}{)}        
    \PY{c+c1}{\PYZsh{}\PYZhy{}\PYZhy{}\PYZhy{}\PYZhy{}\PYZhy{}\PYZhy{}\PYZhy{}\PYZhy{}\PYZhy{}\PYZhy{}\PYZhy{}\PYZhy{}\PYZhy{}\PYZhy{}\PYZhy{}\PYZhy{}\PYZhy{}\PYZhy{}\PYZhy{}\PYZhy{}\PYZhy{}\PYZhy{}\PYZhy{}\PYZhy{}\PYZhy{}\PYZhy{}\PYZhy{}\PYZhy{}\PYZhy{}\PYZhy{}\PYZhy{}\PYZhy{}}
    \PY{c+c1}{\PYZsh{} fill all to\PYZus{}fix\PYZus{}colums with most common value }
    \PY{k}{for} \PY{n}{i} \PY{o+ow}{in} \PY{n}{to\PYZus{}fix\PYZus{}list}\PY{p}{:}
        \PY{n}{temp} \PY{o}{=} \PY{n}{df}\PY{p}{[}\PY{n+nb}{str}\PY{p}{(}\PY{n}{i}\PY{p}{)}\PY{p}{]}\PY{o}{.}\PY{n}{value\PYZus{}counts}\PY{p}{(}\PY{p}{)}\PY{o}{.}\PY{n}{idxmax}\PY{p}{(}\PY{p}{)}
        \PY{n}{df}\PY{p}{[}\PY{n+nb}{str}\PY{p}{(}\PY{n}{i}\PY{p}{)}\PY{p}{]} \PY{o}{=} \PY{n}{df}\PY{p}{[}\PY{n+nb}{str}\PY{p}{(}\PY{n}{i}\PY{p}{)}\PY{p}{]}\PY{o}{.}\PY{n}{replace}\PY{p}{(}\PY{n}{np}\PY{o}{.}\PY{n}{nan}\PY{p}{,} \PY{n}{temp}\PY{p}{)}  
    \PY{c+c1}{\PYZsh{}\PYZhy{}\PYZhy{}\PYZhy{}\PYZhy{}\PYZhy{}\PYZhy{}\PYZhy{}\PYZhy{}\PYZhy{}\PYZhy{}\PYZhy{}\PYZhy{}\PYZhy{}\PYZhy{}\PYZhy{}\PYZhy{}\PYZhy{}\PYZhy{}\PYZhy{}\PYZhy{}\PYZhy{}\PYZhy{}\PYZhy{}\PYZhy{}\PYZhy{}\PYZhy{}\PYZhy{}\PYZhy{}\PYZhy{}\PYZhy{}\PYZhy{}\PYZhy{}}
    \PY{c+c1}{\PYZsh{} extrakt the index from isnull\PYZus{}list and safe it in feature\PYZus{}list}
    \PY{n}{feature\PYZus{}list} \PY{o}{=} \PY{p}{[}\PY{p}{]}
    \PY{k}{for} \PY{n}{i} \PY{o+ow}{in} \PY{n}{capped\PYZus{}list}\PY{p}{:}
        \PY{n}{feature\PYZus{}list}\PY{o}{.}\PY{n}{append}\PY{p}{(}\PY{n}{i}\PY{p}{[}\PY{l+m+mi}{0}\PY{p}{]}\PY{p}{)}
    \PY{c+c1}{\PYZsh{}\PYZhy{}\PYZhy{}\PYZhy{}\PYZhy{}\PYZhy{}\PYZhy{}\PYZhy{}\PYZhy{}\PYZhy{}\PYZhy{}\PYZhy{}\PYZhy{}\PYZhy{}\PYZhy{}\PYZhy{}\PYZhy{}\PYZhy{}\PYZhy{}\PYZhy{}\PYZhy{}\PYZhy{}\PYZhy{}\PYZhy{}\PYZhy{}\PYZhy{}\PYZhy{}\PYZhy{}\PYZhy{}\PYZhy{}\PYZhy{}\PYZhy{}\PYZhy{}}
    \PY{c+c1}{\PYZsh{} filter the dataframe by the feature\PYZus{}list}
    \PY{n}{df} \PY{o}{=} \PY{n}{df}\PY{o}{.}\PY{n}{filter}\PY{p}{(}\PY{n}{items}\PY{o}{=}\PY{n}{feature\PYZus{}list}\PY{p}{)}        
    \PY{k}{return} \PY{n}{df}
\end{Verbatim}
\end{tcolorbox}

    \hypertarget{funktion-fuxfcr-die-separierung-von-datentypen-eines-dataframes}{%
\subsubsection{Funktion für die Separierung von Datentypen eines
DataFrames}\label{funktion-fuxfcr-die-separierung-von-datentypen-eines-dataframes}}

    \begin{tcolorbox}[breakable, size=fbox, boxrule=1pt, pad at break*=1mm,colback=cellbackground, colframe=cellborder]
\prompt{In}{incolor}{7}{\boxspacing}
\begin{Verbatim}[commandchars=\\\{\}]
\PY{c+c1}{\PYZsh{} Definition einer Funktion, die einen DataFrame erhält und daraufhin automatisch die Datentypen}
\PY{c+c1}{\PYZsh{} in verschiedene Listen separiert, um diese mittels One\PYZhy{}Hot\PYZhy{}Endocding aufbereiten zu können}
\PY{c+c1}{\PYZsh{}\PYZhy{}\PYZhy{}\PYZhy{}\PYZhy{}\PYZhy{}\PYZhy{}\PYZhy{}\PYZhy{}\PYZhy{}\PYZhy{}\PYZhy{}\PYZhy{}}
\PY{c+c1}{\PYZsh{} Argumente:}
\PY{c+c1}{\PYZsh{} \PYZhy{} df: DataFrame welcher bearbeitet werden soll}
\PY{c+c1}{\PYZsh{}\PYZhy{}\PYZhy{}\PYZhy{}\PYZhy{}\PYZhy{}\PYZhy{}\PYZhy{}\PYZhy{}\PYZhy{}\PYZhy{}\PYZhy{}\PYZhy{}}

\PY{k}{def} \PY{n+nf}{separate\PYZus{}dtypes}\PY{p}{(}\PY{n}{df}\PY{p}{)}\PY{p}{:}
    \PY{c+c1}{\PYZsh{}\PYZhy{}\PYZhy{}\PYZhy{}\PYZhy{}\PYZhy{}\PYZhy{}\PYZhy{}\PYZhy{}\PYZhy{}\PYZhy{}\PYZhy{}\PYZhy{}\PYZhy{}\PYZhy{}\PYZhy{}\PYZhy{}\PYZhy{}\PYZhy{}\PYZhy{}\PYZhy{}\PYZhy{}\PYZhy{}\PYZhy{}\PYZhy{}\PYZhy{}\PYZhy{}\PYZhy{}\PYZhy{}\PYZhy{}\PYZhy{}\PYZhy{}\PYZhy{}}
    \PY{c+c1}{\PYZsh{} Get dtypes from dataframe and safe in list  }
    \PY{n}{values} \PY{o}{=} \PY{n}{df}\PY{o}{.}\PY{n}{dtypes}\PY{o}{.}\PY{n}{values}\PY{o}{.}\PY{n}{tolist}\PY{p}{(}\PY{p}{)}
    \PY{n}{index} \PY{o}{=} \PY{n}{df}\PY{o}{.}\PY{n}{dtypes}\PY{o}{.}\PY{n}{index}\PY{o}{.}\PY{n}{tolist}\PY{p}{(}\PY{p}{)}
    \PY{c+c1}{\PYZsh{}\PYZhy{}\PYZhy{}\PYZhy{}\PYZhy{}\PYZhy{}\PYZhy{}\PYZhy{}\PYZhy{}\PYZhy{}\PYZhy{}\PYZhy{}\PYZhy{}\PYZhy{}\PYZhy{}\PYZhy{}\PYZhy{}\PYZhy{}\PYZhy{}\PYZhy{}\PYZhy{}\PYZhy{}\PYZhy{}\PYZhy{}\PYZhy{}\PYZhy{}\PYZhy{}\PYZhy{}\PYZhy{}\PYZhy{}\PYZhy{}\PYZhy{}\PYZhy{}}
    \PY{c+c1}{\PYZsh{} match value and index in one list}
    \PY{n}{type\PYZus{}list} \PY{o}{=} \PY{p}{[}\PY{p}{]}
    \PY{k}{for} \PY{n}{i} \PY{o+ow}{in} \PY{n+nb}{range}\PY{p}{(}\PY{l+m+mi}{1}\PY{p}{,} \PY{n+nb}{len}\PY{p}{(}\PY{n}{index}\PY{p}{)}\PY{p}{)}\PY{p}{:}
        \PY{n}{temp} \PY{o}{=} \PY{p}{(}\PY{n}{index}\PY{p}{[}\PY{n}{i}\PY{p}{]}\PY{p}{,} \PY{n}{values}\PY{p}{[}\PY{n}{i}\PY{p}{]}\PY{p}{)}
        \PY{n}{type\PYZus{}list}\PY{o}{.}\PY{n}{append}\PY{p}{(}\PY{n}{temp}\PY{p}{)}
    \PY{c+c1}{\PYZsh{}\PYZhy{}\PYZhy{}\PYZhy{}\PYZhy{}\PYZhy{}\PYZhy{}\PYZhy{}\PYZhy{}\PYZhy{}\PYZhy{}\PYZhy{}\PYZhy{}\PYZhy{}\PYZhy{}\PYZhy{}\PYZhy{}\PYZhy{}\PYZhy{}\PYZhy{}\PYZhy{}\PYZhy{}\PYZhy{}\PYZhy{}\PYZhy{}\PYZhy{}\PYZhy{}\PYZhy{}\PYZhy{}\PYZhy{}\PYZhy{}\PYZhy{}\PYZhy{}}
    \PY{c+c1}{\PYZsh{} for each element of type\PYZus{}list determine type and safe in specific list}
    \PY{n}{int\PYZus{}list}\PY{p}{,} \PY{n}{object\PYZus{}list}\PY{p}{,} \PY{n}{float\PYZus{}list}\PY{p}{,} \PY{n}{other\PYZus{}list} \PY{o}{=} \PY{p}{[}\PY{p}{]}\PY{p}{,} \PY{p}{[}\PY{p}{]}\PY{p}{,} \PY{p}{[}\PY{p}{]}\PY{p}{,} \PY{p}{[}\PY{p}{]}
    \PY{k}{for} \PY{n}{i} \PY{o+ow}{in} \PY{n+nb}{range}\PY{p}{(}\PY{l+m+mi}{0}\PY{p}{,} \PY{n+nb}{len}\PY{p}{(}\PY{n}{type\PYZus{}list}\PY{p}{)}\PY{p}{)}\PY{p}{:}
        \PY{c+c1}{\PYZsh{} Int\PYZhy{}Werte}
        \PY{k}{if} \PY{n}{type\PYZus{}list}\PY{p}{[}\PY{n}{i}\PY{p}{]}\PY{p}{[}\PY{l+m+mi}{1}\PY{p}{]} \PY{o}{==} \PY{l+s+s1}{\PYZsq{}}\PY{l+s+s1}{int64}\PY{l+s+s1}{\PYZsq{}}\PY{p}{:}
            \PY{n}{int\PYZus{}list}\PY{o}{.}\PY{n}{append}\PY{p}{(}\PY{n}{type\PYZus{}list}\PY{p}{[}\PY{n}{i}\PY{p}{]}\PY{p}{[}\PY{l+m+mi}{0}\PY{p}{]}\PY{p}{)}
        \PY{c+c1}{\PYZsh{} Object\PYZhy{}Werte}
        \PY{k}{elif} \PY{n}{type\PYZus{}list}\PY{p}{[}\PY{n}{i}\PY{p}{]}\PY{p}{[}\PY{l+m+mi}{1}\PY{p}{]} \PY{o}{==} \PY{l+s+s1}{\PYZsq{}}\PY{l+s+s1}{object}\PY{l+s+s1}{\PYZsq{}}\PY{p}{:}
            \PY{n}{object\PYZus{}list}\PY{o}{.}\PY{n}{append}\PY{p}{(}\PY{n}{type\PYZus{}list}\PY{p}{[}\PY{n}{i}\PY{p}{]}\PY{p}{[}\PY{l+m+mi}{0}\PY{p}{]}\PY{p}{)}
        \PY{c+c1}{\PYZsh{} Float\PYZhy{}Werte}
        \PY{k}{elif} \PY{n}{type\PYZus{}list}\PY{p}{[}\PY{n}{i}\PY{p}{]}\PY{p}{[}\PY{l+m+mi}{1}\PY{p}{]} \PY{o}{==} \PY{l+s+s1}{\PYZsq{}}\PY{l+s+s1}{float64}\PY{l+s+s1}{\PYZsq{}}\PY{p}{:}
            \PY{n}{float\PYZus{}list}\PY{o}{.}\PY{n}{append}\PY{p}{(}\PY{n}{type\PYZus{}list}\PY{p}{[}\PY{n}{i}\PY{p}{]}\PY{p}{[}\PY{l+m+mi}{0}\PY{p}{]}\PY{p}{)}
        \PY{c+c1}{\PYZsh{} Andere Werte}
        \PY{k}{else}\PY{p}{:}
            \PY{n}{other\PYZus{}list}\PY{o}{.}\PY{n}{append}\PY{p}{(}\PY{n}{type\PYZus{}list}\PY{p}{[}\PY{n}{i}\PY{p}{]}\PY{p}{[}\PY{l+m+mi}{0}\PY{p}{]}\PY{p}{)}
            
    \PY{c+c1}{\PYZsh{}\PYZhy{}\PYZhy{}\PYZhy{}\PYZhy{}\PYZhy{}\PYZhy{}\PYZhy{}\PYZhy{}\PYZhy{}\PYZhy{}\PYZhy{}\PYZhy{}\PYZhy{}\PYZhy{}\PYZhy{}\PYZhy{}\PYZhy{}\PYZhy{}\PYZhy{}\PYZhy{}\PYZhy{}\PYZhy{}\PYZhy{}\PYZhy{}\PYZhy{}\PYZhy{}\PYZhy{}\PYZhy{}\PYZhy{}\PYZhy{}\PYZhy{}\PYZhy{}}
    \PY{c+c1}{\PYZsh{} return lists as dictonary   }
    \PY{n}{my\PYZus{}dict} \PY{o}{=} \PY{p}{\PYZob{}}\PY{l+s+s1}{\PYZsq{}}\PY{l+s+s1}{object}\PY{l+s+s1}{\PYZsq{}}\PY{p}{:}\PY{n}{object\PYZus{}list}\PY{p}{,} \PY{l+s+s1}{\PYZsq{}}\PY{l+s+s1}{int64}\PY{l+s+s1}{\PYZsq{}}\PY{p}{:}\PY{n}{int\PYZus{}list}\PY{p}{,} \PY{l+s+s1}{\PYZsq{}}\PY{l+s+s1}{float64}\PY{l+s+s1}{\PYZsq{}}\PY{p}{:}\PY{n}{float\PYZus{}list}\PY{p}{,} \PY{l+s+s1}{\PYZsq{}}\PY{l+s+s1}{other}\PY{l+s+s1}{\PYZsq{}}\PY{p}{:}\PY{n}{other\PYZus{}list}\PY{p}{\PYZcb{}}
    \PY{k}{return} \PY{n}{my\PYZus{}dict}
\end{Verbatim}
\end{tcolorbox}

    \hypertarget{funktion-fuxfcr-das-submitten}{%
\subsubsection{Funktion für das
Submitten}\label{funktion-fuxfcr-das-submitten}}

    \begin{tcolorbox}[breakable, size=fbox, boxrule=1pt, pad at break*=1mm,colback=cellbackground, colframe=cellborder]
\prompt{In}{incolor}{8}{\boxspacing}
\begin{Verbatim}[commandchars=\\\{\}]
\PY{c+c1}{\PYZsh{} Definition einer Funktion, welche das Submitten der Prognose auf dem Testdatensatz erleichert}
\PY{c+c1}{\PYZsh{}\PYZhy{}\PYZhy{}\PYZhy{}\PYZhy{}\PYZhy{}\PYZhy{}\PYZhy{}\PYZhy{}\PYZhy{}\PYZhy{}\PYZhy{}\PYZhy{}}
\PY{c+c1}{\PYZsh{} Argumente:}
\PY{c+c1}{\PYZsh{} \PYZhy{} model: Modell auf Basis dessen die Prognose erfolgt}
\PY{c+c1}{\PYZsh{} \PYZhy{} X\PYZus{}test: Datensatz aus Basis dessen die Prognose erfolgt}
\PY{c+c1}{\PYZsh{} \PYZhy{} save:}
\PY{c+c1}{\PYZsh{} \PYZhy{}\PYZhy{}\PYZhy{}\PYZgt{} Wenn False, dann werden die prognostizierten Daten nicht gespeichert}
\PY{c+c1}{\PYZsh{} \PYZhy{}\PYZhy{}\PYZhy{}\PYZgt{} Wenn True, dann werden die prognostizierten Daten als .csv gespeichert}
\PY{c+c1}{\PYZsh{} \PYZhy{} manu\PYZus{}name:}
\PY{c+c1}{\PYZsh{} \PYZhy{}\PYZhy{}\PYZhy{}\PYZgt{} Wenn None, dann wird ein nicht eindeutiger Standardname als Bezeichnung der .csv gewählt}
\PY{c+c1}{\PYZsh{} \PYZhy{}\PYZhy{}\PYZhy{}\PYZgt{} Wenn != None, dann wird die zu speichernde .csv mit einem timestamp versehen}
\PY{c+c1}{\PYZsh{}\PYZhy{}\PYZhy{}\PYZhy{}\PYZhy{}\PYZhy{}\PYZhy{}\PYZhy{}\PYZhy{}\PYZhy{}\PYZhy{}\PYZhy{}\PYZhy{}}

\PY{k}{def} \PY{n+nf}{submit}\PY{p}{(}\PY{n}{model}\PY{p}{,} \PY{n}{X\PYZus{}test}\PY{p}{,} \PY{n}{y\PYZus{}scaler}\PY{p}{,} \PY{n}{save}\PY{o}{=}\PY{k+kc}{False}\PY{p}{,} \PY{n}{manu\PYZus{}name}\PY{o}{=}\PY{k+kc}{False}\PY{p}{)}\PY{p}{:}    
    \PY{c+c1}{\PYZsh{}\PYZhy{}\PYZhy{}\PYZhy{}\PYZhy{}\PYZhy{}\PYZhy{}\PYZhy{}\PYZhy{}\PYZhy{}\PYZhy{}\PYZhy{}\PYZhy{}\PYZhy{}\PYZhy{}\PYZhy{}\PYZhy{}\PYZhy{}\PYZhy{}\PYZhy{}\PYZhy{}\PYZhy{}\PYZhy{}\PYZhy{}\PYZhy{}\PYZhy{}\PYZhy{}\PYZhy{}\PYZhy{}\PYZhy{}\PYZhy{}\PYZhy{}\PYZhy{}}
    \PY{c+c1}{\PYZsh{} predict on test set and inverse transformation for y values}
    \PY{n}{predicted\PYZus{}test} \PY{o}{=} \PY{n}{model}\PY{o}{.}\PY{n}{predict}\PY{p}{(}\PY{n}{X\PYZus{}test}\PY{p}{)}    
    \PY{n}{predicted\PYZus{}test} \PY{o}{=} \PY{n}{y\PYZus{}scalar}\PY{o}{.}\PY{n}{inverse\PYZus{}transform}\PY{p}{(}\PY{n}{predicted\PYZus{}test}\PY{o}{.}\PY{n}{reshape}\PY{p}{(}\PY{o}{\PYZhy{}}\PY{l+m+mi}{1}\PY{p}{,} \PY{l+m+mi}{1}\PY{p}{)}\PY{p}{)}
    \PY{c+c1}{\PYZsh{}\PYZhy{}\PYZhy{}\PYZhy{}\PYZhy{}\PYZhy{}\PYZhy{}\PYZhy{}\PYZhy{}\PYZhy{}\PYZhy{}\PYZhy{}\PYZhy{}\PYZhy{}\PYZhy{}\PYZhy{}\PYZhy{}\PYZhy{}\PYZhy{}\PYZhy{}\PYZhy{}\PYZhy{}\PYZhy{}\PYZhy{}\PYZhy{}\PYZhy{}\PYZhy{}\PYZhy{}\PYZhy{}\PYZhy{}\PYZhy{}\PYZhy{}\PYZhy{}}
    \PY{c+c1}{\PYZsh{} Submissiondatensatz einlesen und prognostizierte Werte hineinschreiben}
    \PY{n}{submission} \PY{o}{=} \PY{n}{pd}\PY{o}{.}\PY{n}{read\PYZus{}csv}\PY{p}{(}\PY{l+s+s1}{\PYZsq{}}\PY{l+s+s1}{sample\PYZus{}submission.csv}\PY{l+s+s1}{\PYZsq{}}\PY{p}{)}
    \PY{n}{submission}\PY{p}{[}\PY{l+s+s1}{\PYZsq{}}\PY{l+s+s1}{SalePrice}\PY{l+s+s1}{\PYZsq{}}\PY{p}{]} \PY{o}{=} \PY{n}{predicted\PYZus{}test}    
    \PY{c+c1}{\PYZsh{}\PYZhy{}\PYZhy{}\PYZhy{}\PYZhy{}\PYZhy{}\PYZhy{}\PYZhy{}\PYZhy{}\PYZhy{}\PYZhy{}\PYZhy{}\PYZhy{}\PYZhy{}\PYZhy{}\PYZhy{}\PYZhy{}\PYZhy{}\PYZhy{}\PYZhy{}\PYZhy{}\PYZhy{}\PYZhy{}\PYZhy{}\PYZhy{}\PYZhy{}\PYZhy{}\PYZhy{}\PYZhy{}\PYZhy{}\PYZhy{}\PYZhy{}\PYZhy{}}
    \PY{c+c1}{\PYZsh{} In .csv speichern, wenn save=True}
    \PY{k}{if} \PY{n}{save} \PY{o}{==} \PY{k+kc}{True}\PY{p}{:}        
        \PY{c+c1}{\PYZsh{}\PYZhy{}\PYZhy{}\PYZhy{}\PYZhy{}\PYZhy{}\PYZhy{}\PYZhy{}\PYZhy{}\PYZhy{}\PYZhy{}\PYZhy{}\PYZhy{}\PYZhy{}\PYZhy{}\PYZhy{}\PYZhy{}\PYZhy{}\PYZhy{}\PYZhy{}\PYZhy{}\PYZhy{}\PYZhy{}\PYZhy{}\PYZhy{}\PYZhy{}\PYZhy{}\PYZhy{}\PYZhy{}\PYZhy{}\PYZhy{}\PYZhy{}\PYZhy{}}
        \PY{c+c1}{\PYZsh{} Standardnamen wählen, wenn manu\PYZus{}name == False}
        \PY{k}{if} \PY{n}{manu\PYZus{}name} \PY{o}{==} \PY{k+kc}{False}\PY{p}{:}
            \PY{n}{submission}\PY{o}{.}\PY{n}{to\PYZus{}csv}\PY{p}{(}\PY{l+s+s1}{\PYZsq{}}\PY{l+s+s1}{./predicted\PYZus{}values.csv}\PY{l+s+s1}{\PYZsq{}}\PY{p}{,} \PY{n}{index}\PY{o}{=}\PY{k+kc}{False}\PY{p}{)}            
        \PY{c+c1}{\PYZsh{}\PYZhy{}\PYZhy{}\PYZhy{}\PYZhy{}\PYZhy{}\PYZhy{}\PYZhy{}\PYZhy{}\PYZhy{}\PYZhy{}\PYZhy{}\PYZhy{}\PYZhy{}\PYZhy{}\PYZhy{}\PYZhy{}\PYZhy{}\PYZhy{}\PYZhy{}\PYZhy{}\PYZhy{}\PYZhy{}\PYZhy{}\PYZhy{}\PYZhy{}\PYZhy{}\PYZhy{}\PYZhy{}\PYZhy{}\PYZhy{}\PYZhy{}\PYZhy{}}
        \PY{c+c1}{\PYZsh{} Standardnamen mit timestamp kombinieren, wenn  manu\PYZus{}name == True}
        \PY{k}{if} \PY{n}{manu\PYZus{}name} \PY{o}{==} \PY{k+kc}{True}\PY{p}{:}            
            \PY{k+kn}{import} \PY{n+nn}{datetime} 
            \PY{n}{now} \PY{o}{=} \PY{n}{datetime}\PY{o}{.}\PY{n}{datetime}\PY{o}{.}\PY{n}{now}\PY{p}{(}\PY{p}{)}
            \PY{n}{name} \PY{o}{=} \PY{n}{now}\PY{o}{.}\PY{n}{strftime}\PY{p}{(}\PY{l+s+s1}{\PYZsq{}}\PY{l+s+s1}{\PYZpc{}}\PY{l+s+s1}{Y\PYZhy{}}\PY{l+s+s1}{\PYZpc{}}\PY{l+s+s1}{m\PYZhy{}}\PY{l+s+si}{\PYZpc{}d}\PY{l+s+s1}{T}\PY{l+s+s1}{\PYZpc{}}\PY{l+s+s1}{H}\PY{l+s+s1}{\PYZpc{}}\PY{l+s+s1}{M}\PY{l+s+s1}{\PYZpc{}}\PY{l+s+s1}{S}\PY{l+s+s1}{\PYZsq{}}\PY{p}{)} \PY{o}{+} \PY{p}{(}\PY{l+s+s1}{\PYZsq{}}\PY{l+s+s1}{\PYZhy{}}\PY{l+s+si}{\PYZpc{}02d}\PY{l+s+s1}{\PYZsq{}} \PY{o}{\PYZpc{}} \PY{p}{(}\PY{n}{now}\PY{o}{.}\PY{n}{microsecond} \PY{o}{/} \PY{l+m+mi}{10000}\PY{p}{)}\PY{p}{)}            
            \PY{n}{submission}\PY{o}{.}\PY{n}{to\PYZus{}csv}\PY{p}{(}\PY{l+s+s1}{\PYZsq{}}\PY{l+s+s1}{./predicted\PYZus{}values\PYZus{}}\PY{l+s+s1}{\PYZsq{}} \PY{o}{+} \PY{n+nb}{str}\PY{p}{(}\PY{n}{name}\PY{p}{)} \PY{o}{+} \PY{l+s+s1}{\PYZsq{}}\PY{l+s+s1}{.csv}\PY{l+s+s1}{\PYZsq{}}\PY{p}{,} \PY{n}{index}\PY{o}{=}\PY{k+kc}{False}\PY{p}{)}            
        
    \PY{c+c1}{\PYZsh{}\PYZhy{}\PYZhy{}\PYZhy{}\PYZhy{}\PYZhy{}\PYZhy{}\PYZhy{}\PYZhy{}\PYZhy{}\PYZhy{}\PYZhy{}\PYZhy{}\PYZhy{}\PYZhy{}\PYZhy{}\PYZhy{}\PYZhy{}\PYZhy{}\PYZhy{}\PYZhy{}\PYZhy{}\PYZhy{}\PYZhy{}\PYZhy{}\PYZhy{}\PYZhy{}\PYZhy{}\PYZhy{}\PYZhy{}\PYZhy{}\PYZhy{}\PYZhy{}}
    \PY{c+c1}{\PYZsh{} return submission dataframe}
    \PY{k}{return} \PY{n}{submission}
\end{Verbatim}
\end{tcolorbox}

    \hypertarget{funktion-zum-abgleich-zweier-listen}{%
\subsubsection{Funktion zum Abgleich zweier
Listen}\label{funktion-zum-abgleich-zweier-listen}}

    \begin{tcolorbox}[breakable, size=fbox, boxrule=1pt, pad at break*=1mm,colback=cellbackground, colframe=cellborder]
\prompt{In}{incolor}{9}{\boxspacing}
\begin{Verbatim}[commandchars=\\\{\}]
\PY{c+c1}{\PYZsh{} Definition einer Funktion, die zwei Listen miteinander abgleicht und deren}
\PY{c+c1}{\PYZsh{} gemeinsame Werten zurückgibt}
\PY{c+c1}{\PYZsh{}\PYZhy{}\PYZhy{}\PYZhy{}\PYZhy{}\PYZhy{}\PYZhy{}\PYZhy{}\PYZhy{}\PYZhy{}\PYZhy{}\PYZhy{}\PYZhy{}}
\PY{c+c1}{\PYZsh{} Argumente:}
\PY{c+c1}{\PYZsh{} \PYZhy{} a: Lsite 1}
\PY{c+c1}{\PYZsh{} \PYZhy{} b: Liste 2}
\PY{c+c1}{\PYZsh{}\PYZhy{}\PYZhy{}\PYZhy{}\PYZhy{}\PYZhy{}\PYZhy{}\PYZhy{}\PYZhy{}\PYZhy{}\PYZhy{}\PYZhy{}\PYZhy{}}

\PY{k}{def} \PY{n+nf}{returnMatches}\PY{p}{(}\PY{n}{a}\PY{p}{,} \PY{n}{b}\PY{p}{,} \PY{n}{or\PYZus{}not\PYZus{}matches}\PY{o}{=}\PY{k+kc}{False}\PY{p}{)}\PY{p}{:}
    \PY{c+c1}{\PYZsh{}\PYZhy{}\PYZhy{}\PYZhy{}\PYZhy{}\PYZhy{}\PYZhy{}\PYZhy{}\PYZhy{}\PYZhy{}\PYZhy{}\PYZhy{}\PYZhy{}\PYZhy{}\PYZhy{}\PYZhy{}\PYZhy{}\PYZhy{}\PYZhy{}\PYZhy{}\PYZhy{}\PYZhy{}\PYZhy{}\PYZhy{}\PYZhy{}\PYZhy{}\PYZhy{}\PYZhy{}\PYZhy{}\PYZhy{}\PYZhy{}\PYZhy{}\PYZhy{}}
    \PY{c+c1}{\PYZsh{} check lists for matches}
    \PY{k}{if} \PY{n}{or\PYZus{}not\PYZus{}matches} \PY{o}{==} \PY{k+kc}{False}\PY{p}{:}
        \PY{n}{matches} \PY{o}{=} \PY{p}{[}\PY{n}{x} \PY{k}{for} \PY{n}{x} \PY{o+ow}{in} \PY{n}{a} \PY{k}{if} \PY{n}{x} \PY{o+ow}{in} \PY{n}{b}\PY{p}{]}\PY{p}{,} \PY{p}{[}\PY{n}{x} \PY{k}{for} \PY{n}{x} \PY{o+ow}{in} \PY{n}{b} \PY{k}{if} \PY{n}{x} \PY{o+ow}{in} \PY{n}{a}\PY{p}{]}
    \PY{k}{else}\PY{p}{:}
        \PY{n}{matches} \PY{o}{=} \PY{p}{[}\PY{n}{x} \PY{k}{for} \PY{n}{x} \PY{o+ow}{in} \PY{n}{a} \PY{k}{if} \PY{n}{x} \PY{o+ow}{not} \PY{o+ow}{in} \PY{n}{b}\PY{p}{]}\PY{p}{,} \PY{p}{[}\PY{n}{x} \PY{k}{for} \PY{n}{x} \PY{o+ow}{in} \PY{n}{b} \PY{k}{if} \PY{n}{x} \PY{o+ow}{not} \PY{o+ow}{in} \PY{n}{a}\PY{p}{]}
    \PY{c+c1}{\PYZsh{}\PYZhy{}\PYZhy{}\PYZhy{}\PYZhy{}\PYZhy{}\PYZhy{}\PYZhy{}\PYZhy{}\PYZhy{}\PYZhy{}\PYZhy{}\PYZhy{}\PYZhy{}\PYZhy{}\PYZhy{}\PYZhy{}\PYZhy{}\PYZhy{}\PYZhy{}\PYZhy{}\PYZhy{}\PYZhy{}\PYZhy{}\PYZhy{}\PYZhy{}\PYZhy{}\PYZhy{}\PYZhy{}\PYZhy{}\PYZhy{}\PYZhy{}\PYZhy{}}
    \PY{c+c1}{\PYZsh{} return first or second array}
    \PY{k}{if} \PY{n+nb}{len}\PY{p}{(}\PY{n}{matches}\PY{p}{[}\PY{l+m+mi}{0}\PY{p}{]}\PY{p}{)} \PY{o}{==} \PY{l+m+mi}{0}\PY{p}{:}        
        \PY{k}{return} \PY{n}{matches}\PY{p}{[}\PY{l+m+mi}{1}\PY{p}{]}
    \PY{k}{else}\PY{p}{:}
        \PY{k}{return} \PY{n}{matches}\PY{p}{[}\PY{l+m+mi}{0}\PY{p}{]}
\end{Verbatim}
\end{tcolorbox}

    \hypertarget{funktion-zum-matchen-zweier-dataframes}{%
\subsubsection{Funktion zum matchen zweier
DataFrames}\label{funktion-zum-matchen-zweier-dataframes}}

    \begin{tcolorbox}[breakable, size=fbox, boxrule=1pt, pad at break*=1mm,colback=cellbackground, colframe=cellborder]
\prompt{In}{incolor}{10}{\boxspacing}
\begin{Verbatim}[commandchars=\\\{\}]
\PY{c+c1}{\PYZsh{} Definition einer Funktion, die zwei DataFrame miteinander matcht, indem }
\PY{c+c1}{\PYZsh{} nur gemeinsame Spalten übernommen werden}
\PY{c+c1}{\PYZsh{}\PYZhy{}\PYZhy{}\PYZhy{}\PYZhy{}\PYZhy{}\PYZhy{}\PYZhy{}\PYZhy{}\PYZhy{}\PYZhy{}\PYZhy{}\PYZhy{}}
\PY{c+c1}{\PYZsh{} Argumente:}
\PY{c+c1}{\PYZsh{} \PYZhy{} a: Lsite 1}
\PY{c+c1}{\PYZsh{} \PYZhy{} b: Liste 2}
\PY{c+c1}{\PYZsh{}\PYZhy{}\PYZhy{}\PYZhy{}\PYZhy{}\PYZhy{}\PYZhy{}\PYZhy{}\PYZhy{}\PYZhy{}\PYZhy{}\PYZhy{}\PYZhy{}}

\PY{k}{def} \PY{n+nf}{match\PYZus{}train\PYZus{}and\PYZus{}test}\PY{p}{(}\PY{n}{train\PYZus{}data}\PY{p}{,} \PY{n}{test\PYZus{}data}\PY{p}{)}\PY{p}{:}
    \PY{c+c1}{\PYZsh{}\PYZhy{}\PYZhy{}\PYZhy{}\PYZhy{}\PYZhy{}\PYZhy{}\PYZhy{}\PYZhy{}\PYZhy{}\PYZhy{}\PYZhy{}\PYZhy{}\PYZhy{}\PYZhy{}\PYZhy{}\PYZhy{}\PYZhy{}\PYZhy{}\PYZhy{}\PYZhy{}\PYZhy{}\PYZhy{}\PYZhy{}\PYZhy{}\PYZhy{}\PYZhy{}\PYZhy{}\PYZhy{}\PYZhy{}\PYZhy{}\PYZhy{}\PYZhy{}}
    \PY{c+c1}{\PYZsh{} get colum names form both dataframes}
    \PY{n}{a} \PY{o}{=} \PY{n}{train\PYZus{}data}\PY{o}{.}\PY{n}{columns}\PY{o}{.}\PY{n}{values}\PY{o}{.}\PY{n}{tolist}\PY{p}{(}\PY{p}{)}
    \PY{n}{b} \PY{o}{=} \PY{n}{test\PYZus{}data}\PY{o}{.}\PY{n}{columns}\PY{o}{.}\PY{n}{values}\PY{o}{.}\PY{n}{tolist}\PY{p}{(}\PY{p}{)}
    \PY{c+c1}{\PYZsh{}\PYZhy{}\PYZhy{}\PYZhy{}\PYZhy{}\PYZhy{}\PYZhy{}\PYZhy{}\PYZhy{}\PYZhy{}\PYZhy{}\PYZhy{}\PYZhy{}\PYZhy{}\PYZhy{}\PYZhy{}\PYZhy{}\PYZhy{}\PYZhy{}\PYZhy{}\PYZhy{}\PYZhy{}\PYZhy{}\PYZhy{}\PYZhy{}\PYZhy{}\PYZhy{}\PYZhy{}\PYZhy{}\PYZhy{}\PYZhy{}\PYZhy{}\PYZhy{}}
    \PY{c+c1}{\PYZsh{} return matches}
    \PY{n}{z} \PY{o}{=} \PY{n}{returnMatches}\PY{p}{(}\PY{n}{b}\PY{p}{,} \PY{n}{a}\PY{p}{)}
    \PY{c+c1}{\PYZsh{}\PYZhy{}\PYZhy{}\PYZhy{}\PYZhy{}\PYZhy{}\PYZhy{}\PYZhy{}\PYZhy{}\PYZhy{}\PYZhy{}\PYZhy{}\PYZhy{}\PYZhy{}\PYZhy{}\PYZhy{}\PYZhy{}\PYZhy{}\PYZhy{}\PYZhy{}\PYZhy{}\PYZhy{}\PYZhy{}\PYZhy{}\PYZhy{}\PYZhy{}\PYZhy{}\PYZhy{}\PYZhy{}\PYZhy{}\PYZhy{}\PYZhy{}\PYZhy{}}
    \PY{c+c1}{\PYZsh{} filter both dataframes to match them}
    \PY{n}{train\PYZus{}data} \PY{o}{=} \PY{n}{train\PYZus{}data}\PY{o}{.}\PY{n}{filter}\PY{p}{(}\PY{n}{items}\PY{o}{=}\PY{n}{z}\PY{p}{)}
    \PY{n}{test\PYZus{}data} \PY{o}{=} \PY{n}{test\PYZus{}data}\PY{o}{.}\PY{n}{filter}\PY{p}{(}\PY{n}{items}\PY{o}{=}\PY{n}{z}\PY{p}{)}
    \PY{c+c1}{\PYZsh{}\PYZhy{}\PYZhy{}\PYZhy{}\PYZhy{}\PYZhy{}\PYZhy{}\PYZhy{}\PYZhy{}\PYZhy{}\PYZhy{}\PYZhy{}\PYZhy{}\PYZhy{}\PYZhy{}\PYZhy{}\PYZhy{}\PYZhy{}\PYZhy{}\PYZhy{}\PYZhy{}\PYZhy{}\PYZhy{}\PYZhy{}\PYZhy{}\PYZhy{}\PYZhy{}\PYZhy{}\PYZhy{}\PYZhy{}\PYZhy{}\PYZhy{}\PYZhy{}}
    \PY{c+c1}{\PYZsh{} return matches dataframes    }
    \PY{k}{return} \PY{n}{train\PYZus{}data}\PY{p}{,} \PY{n}{test\PYZus{}data}
\end{Verbatim}
\end{tcolorbox}

    \hypertarget{funktion-zur-ermittlung-der-korrelation-der-dataframe-eintruxe4ge-mit-der-zielgruxf6uxdfe}{%
\subsubsection{Funktion zur Ermittlung der Korrelation der
DataFrame-Einträge mit der
Zielgröße}\label{funktion-zur-ermittlung-der-korrelation-der-dataframe-eintruxe4ge-mit-der-zielgruxf6uxdfe}}

    \begin{tcolorbox}[breakable, size=fbox, boxrule=1pt, pad at break*=1mm,colback=cellbackground, colframe=cellborder]
\prompt{In}{incolor}{11}{\boxspacing}
\begin{Verbatim}[commandchars=\\\{\}]
\PY{c+c1}{\PYZsh{} Definition einer Funktion, die die Korrelation aller Spaltenm mit der Zielgröße}
\PY{c+c1}{\PYZsh{} berechnet und ab einen bestimmten Schwellenwert Features entfernt}
\PY{c+c1}{\PYZsh{}\PYZhy{}\PYZhy{}\PYZhy{}\PYZhy{}\PYZhy{}\PYZhy{}\PYZhy{}\PYZhy{}\PYZhy{}\PYZhy{}\PYZhy{}\PYZhy{}}
\PY{c+c1}{\PYZsh{} Argumente:}
\PY{c+c1}{\PYZsh{} \PYZhy{} df: Zu bearbeitender DataFrame}
\PY{c+c1}{\PYZsh{} \PYZhy{} down\PYZus{}limit: Limit ab welchen Features zugelassen werden}
\PY{c+c1}{\PYZsh{}\PYZhy{}\PYZhy{}\PYZhy{}\PYZhy{}\PYZhy{}\PYZhy{}\PYZhy{}\PYZhy{}\PYZhy{}\PYZhy{}\PYZhy{}\PYZhy{}}

\PY{k}{def} \PY{n+nf}{determine\PYZus{}corr\PYZus{}with\PYZus{}target\PYZus{}value}\PY{p}{(}\PY{n}{df}\PY{p}{,} \PY{n}{down\PYZus{}limit}\PY{o}{=}\PY{l+m+mf}{0.01}\PY{p}{)}\PY{p}{:}
    \PY{c+c1}{\PYZsh{}\PYZhy{}\PYZhy{}\PYZhy{}\PYZhy{}\PYZhy{}\PYZhy{}\PYZhy{}\PYZhy{}\PYZhy{}\PYZhy{}\PYZhy{}\PYZhy{}\PYZhy{}\PYZhy{}\PYZhy{}\PYZhy{}\PYZhy{}\PYZhy{}\PYZhy{}\PYZhy{}\PYZhy{}\PYZhy{}\PYZhy{}\PYZhy{}\PYZhy{}\PYZhy{}\PYZhy{}\PYZhy{}\PYZhy{}\PYZhy{}\PYZhy{}\PYZhy{}}
    \PY{c+c1}{\PYZsh{} Get corr with target vlaue from dataframe and safe in list  }
    \PY{n}{index} \PY{o}{=} \PY{n}{df}\PY{o}{.}\PY{n}{corrwith}\PY{p}{(}\PY{n}{df}\PY{o}{.}\PY{n}{SalePrice}\PY{p}{,} \PY{n}{axis}\PY{o}{=}\PY{l+m+mi}{0}\PY{p}{)}\PY{o}{.}\PY{n}{index}\PY{o}{.}\PY{n}{tolist}\PY{p}{(}\PY{p}{)}
    \PY{n}{values} \PY{o}{=} \PY{n}{df}\PY{o}{.}\PY{n}{corrwith}\PY{p}{(}\PY{n}{df}\PY{o}{.}\PY{n}{SalePrice}\PY{p}{,} \PY{n}{axis}\PY{o}{=}\PY{l+m+mi}{0}\PY{p}{)}\PY{o}{.}\PY{n}{values}\PY{o}{.}\PY{n}{tolist}\PY{p}{(}\PY{p}{)}
    \PY{c+c1}{\PYZsh{}\PYZhy{}\PYZhy{}\PYZhy{}\PYZhy{}\PYZhy{}\PYZhy{}\PYZhy{}\PYZhy{}\PYZhy{}\PYZhy{}\PYZhy{}\PYZhy{}\PYZhy{}\PYZhy{}\PYZhy{}\PYZhy{}\PYZhy{}\PYZhy{}\PYZhy{}\PYZhy{}\PYZhy{}\PYZhy{}\PYZhy{}\PYZhy{}\PYZhy{}\PYZhy{}\PYZhy{}\PYZhy{}\PYZhy{}\PYZhy{}\PYZhy{}\PYZhy{}}
    \PY{c+c1}{\PYZsh{} match value and index in one list}
    \PY{n}{corr\PYZus{}list} \PY{o}{=} \PY{p}{[}\PY{p}{]}
    \PY{k}{for} \PY{n}{i} \PY{o+ow}{in} \PY{n+nb}{range}\PY{p}{(}\PY{l+m+mi}{1}\PY{p}{,} \PY{n+nb}{len}\PY{p}{(}\PY{n}{index}\PY{p}{)}\PY{p}{)}\PY{p}{:}
        \PY{n}{temp} \PY{o}{=} \PY{p}{(}\PY{n}{index}\PY{p}{[}\PY{n}{i}\PY{p}{]}\PY{p}{,} \PY{n}{values}\PY{p}{[}\PY{n}{i}\PY{p}{]}\PY{p}{)}
        \PY{n}{corr\PYZus{}list}\PY{o}{.}\PY{n}{append}\PY{p}{(}\PY{n}{temp}\PY{p}{)}
    \PY{c+c1}{\PYZsh{}\PYZhy{}\PYZhy{}\PYZhy{}\PYZhy{}\PYZhy{}\PYZhy{}\PYZhy{}\PYZhy{}\PYZhy{}\PYZhy{}\PYZhy{}\PYZhy{}\PYZhy{}\PYZhy{}\PYZhy{}\PYZhy{}\PYZhy{}\PYZhy{}\PYZhy{}\PYZhy{}\PYZhy{}\PYZhy{}\PYZhy{}\PYZhy{}\PYZhy{}\PYZhy{}\PYZhy{}\PYZhy{}\PYZhy{}\PYZhy{}\PYZhy{}\PYZhy{}}
    \PY{c+c1}{\PYZsh{} sort corr\PYZus{}list and kick elemennts with corr below down\PYZus{}limit}
    \PY{n}{corr\PYZus{}list} \PY{o}{=} \PY{n+nb}{sorted}\PY{p}{(}\PY{n}{corr\PYZus{}list}\PY{p}{,} \PY{n}{key} \PY{o}{=} \PY{k}{lambda} \PY{n}{x}\PY{p}{:} \PY{n+nb}{abs}\PY{p}{(}\PY{n}{x}\PY{p}{[}\PY{l+m+mi}{1}\PY{p}{]}\PY{p}{)}\PY{p}{,} \PY{n}{reverse}\PY{o}{=}\PY{k+kc}{True}\PY{p}{)}  
    \PY{n}{corr\PYZus{}list} \PY{o}{=} \PY{p}{[}\PY{n}{i} \PY{k}{for} \PY{n}{i} \PY{o+ow}{in} \PY{n}{corr\PYZus{}list} \PY{k}{if} \PY{n+nb}{abs}\PY{p}{(}\PY{n}{i}\PY{p}{[}\PY{l+m+mi}{1}\PY{p}{]}\PY{p}{)} \PY{o}{\PYZgt{}}\PY{o}{=} \PY{n}{down\PYZus{}limit}\PY{p}{]}
    \PY{c+c1}{\PYZsh{}\PYZhy{}\PYZhy{}\PYZhy{}\PYZhy{}\PYZhy{}\PYZhy{}\PYZhy{}\PYZhy{}\PYZhy{}\PYZhy{}\PYZhy{}\PYZhy{}\PYZhy{}\PYZhy{}\PYZhy{}\PYZhy{}\PYZhy{}\PYZhy{}\PYZhy{}\PYZhy{}\PYZhy{}\PYZhy{}\PYZhy{}\PYZhy{}\PYZhy{}\PYZhy{}\PYZhy{}\PYZhy{}\PYZhy{}\PYZhy{}\PYZhy{}\PYZhy{}}
    \PY{c+c1}{\PYZsh{} extrakt the elements that fulfill the specification and and safe themin feature\PYZus{}list}
    \PY{n}{feature\PYZus{}list} \PY{o}{=} \PY{p}{[}\PY{p}{]}
    \PY{k}{for} \PY{n}{i} \PY{o+ow}{in} \PY{n}{corr\PYZus{}list}\PY{p}{:}
        \PY{n}{feature\PYZus{}list}\PY{o}{.}\PY{n}{append}\PY{p}{(}\PY{n}{i}\PY{p}{[}\PY{l+m+mi}{0}\PY{p}{]}\PY{p}{)}
    \PY{c+c1}{\PYZsh{}\PYZhy{}\PYZhy{}\PYZhy{}\PYZhy{}\PYZhy{}\PYZhy{}\PYZhy{}\PYZhy{}\PYZhy{}\PYZhy{}\PYZhy{}\PYZhy{}\PYZhy{}\PYZhy{}\PYZhy{}\PYZhy{}\PYZhy{}\PYZhy{}\PYZhy{}\PYZhy{}\PYZhy{}\PYZhy{}\PYZhy{}\PYZhy{}\PYZhy{}\PYZhy{}\PYZhy{}\PYZhy{}\PYZhy{}\PYZhy{}\PYZhy{}\PYZhy{}}
    \PY{c+c1}{\PYZsh{} filter the dataframe by the feature\PYZus{}list}
    \PY{n}{df} \PY{o}{=} \PY{n}{df}\PY{o}{.}\PY{n}{filter}\PY{p}{(}\PY{n}{items}\PY{o}{=}\PY{n}{feature\PYZus{}list}\PY{p}{)}
    \PY{c+c1}{\PYZsh{}\PYZhy{}\PYZhy{}\PYZhy{}\PYZhy{}\PYZhy{}\PYZhy{}\PYZhy{}\PYZhy{}\PYZhy{}\PYZhy{}\PYZhy{}\PYZhy{}\PYZhy{}\PYZhy{}\PYZhy{}\PYZhy{}\PYZhy{}\PYZhy{}\PYZhy{}\PYZhy{}\PYZhy{}\PYZhy{}\PYZhy{}\PYZhy{}\PYZhy{}\PYZhy{}\PYZhy{}\PYZhy{}\PYZhy{}\PYZhy{}\PYZhy{}\PYZhy{}}
    \PY{c+c1}{\PYZsh{} dataframe}
    \PY{k}{return} \PY{n}{df}
\end{Verbatim}
\end{tcolorbox}

    \hypertarget{funktion-zum-skalieren-der-daten}{%
\subsubsection{Funktion zum Skalieren der
Daten}\label{funktion-zum-skalieren-der-daten}}

    \begin{tcolorbox}[breakable, size=fbox, boxrule=1pt, pad at break*=1mm,colback=cellbackground, colframe=cellborder]
\prompt{In}{incolor}{14}{\boxspacing}
\begin{Verbatim}[commandchars=\\\{\}]
\PY{c+c1}{\PYZsh{} Definition einer Funktion, die die Skalierung der Daten vornimmt}
\PY{c+c1}{\PYZsh{}\PYZhy{}\PYZhy{}\PYZhy{}\PYZhy{}\PYZhy{}\PYZhy{}\PYZhy{}\PYZhy{}\PYZhy{}\PYZhy{}\PYZhy{}\PYZhy{}}
\PY{c+c1}{\PYZsh{} Argumente:}
\PY{c+c1}{\PYZsh{} \PYZhy{} df: Zu bearbeitender DataFrame}
\PY{c+c1}{\PYZsh{} \PYZhy{} my\PYZus{}scaler: Scaler der zur Sklalierung verwendet werden soll}
\PY{c+c1}{\PYZsh{}\PYZhy{}\PYZhy{}\PYZhy{}\PYZhy{}\PYZhy{}\PYZhy{}\PYZhy{}\PYZhy{}\PYZhy{}\PYZhy{}\PYZhy{}\PYZhy{}}

\PY{k}{def} \PY{n+nf}{scale\PYZus{}it}\PY{p}{(}\PY{n}{df}\PY{p}{,} \PY{n}{my\PYZus{}scaler}\PY{o}{=}\PY{n}{StandardScaler}\PY{p}{)}\PY{p}{:}
    \PY{c+c1}{\PYZsh{}\PYZhy{}\PYZhy{}\PYZhy{}\PYZhy{}\PYZhy{}\PYZhy{}\PYZhy{}\PYZhy{}\PYZhy{}\PYZhy{}\PYZhy{}\PYZhy{}\PYZhy{}\PYZhy{}\PYZhy{}\PYZhy{}\PYZhy{}\PYZhy{}\PYZhy{}\PYZhy{}\PYZhy{}\PYZhy{}\PYZhy{}\PYZhy{}\PYZhy{}\PYZhy{}\PYZhy{}\PYZhy{}\PYZhy{}\PYZhy{}\PYZhy{}\PYZhy{}}
    \PY{c+c1}{\PYZsh{} Define x and y data}
    \PY{n}{data} \PY{o}{=} \PY{n}{df}\PY{o}{.}\PY{n}{values}
    \PY{n}{X} \PY{o}{=} \PY{n}{data}\PY{p}{[}\PY{p}{:}\PY{p}{,}\PY{p}{:}\PY{p}{(}\PY{n+nb}{len}\PY{p}{(}\PY{n}{df}\PY{o}{.}\PY{n}{columns}\PY{p}{)} \PY{o}{\PYZhy{}} \PY{l+m+mi}{2}\PY{p}{)}\PY{p}{]}
    \PY{n}{y} \PY{o}{=} \PY{n}{data}\PY{p}{[}\PY{p}{:}\PY{p}{,}\PY{p}{(}\PY{n+nb}{len}\PY{p}{(}\PY{n}{df}\PY{o}{.}\PY{n}{columns}\PY{p}{)} \PY{o}{\PYZhy{}} \PY{l+m+mi}{1}\PY{p}{)}\PY{p}{]}
    \PY{c+c1}{\PYZsh{}\PYZhy{}\PYZhy{}\PYZhy{}\PYZhy{}\PYZhy{}\PYZhy{}\PYZhy{}\PYZhy{}\PYZhy{}\PYZhy{}\PYZhy{}\PYZhy{}\PYZhy{}\PYZhy{}\PYZhy{}\PYZhy{}\PYZhy{}\PYZhy{}\PYZhy{}\PYZhy{}\PYZhy{}\PYZhy{}\PYZhy{}\PYZhy{}\PYZhy{}\PYZhy{}\PYZhy{}\PYZhy{}\PYZhy{}\PYZhy{}\PYZhy{}\PYZhy{}}
    \PY{c+c1}{\PYZsh{} Scaling input data X}
    \PY{n}{scalar} \PY{o}{=} \PY{n}{my\PYZus{}scaler}\PY{p}{(}\PY{p}{)}
    \PY{n}{scalar}\PY{o}{.}\PY{n}{fit}\PY{p}{(}\PY{n}{X}\PY{p}{)}
    \PY{n}{X} \PY{o}{=} \PY{n}{scalar}\PY{o}{.}\PY{n}{transform}\PY{p}{(}\PY{n}{X}\PY{p}{)}
    \PY{c+c1}{\PYZsh{}\PYZhy{}\PYZhy{}\PYZhy{}\PYZhy{}\PYZhy{}\PYZhy{}\PYZhy{}\PYZhy{}\PYZhy{}\PYZhy{}\PYZhy{}\PYZhy{}\PYZhy{}\PYZhy{}\PYZhy{}\PYZhy{}\PYZhy{}\PYZhy{}\PYZhy{}\PYZhy{}\PYZhy{}\PYZhy{}\PYZhy{}\PYZhy{}\PYZhy{}\PYZhy{}\PYZhy{}\PYZhy{}\PYZhy{}\PYZhy{}\PYZhy{}\PYZhy{}}
    \PY{c+c1}{\PYZsh{} Scaling input data y}
    \PY{n}{y} \PY{o}{=} \PY{n}{y}\PY{o}{.}\PY{n}{reshape}\PY{p}{(}\PY{o}{\PYZhy{}}\PY{l+m+mi}{1}\PY{p}{,}\PY{l+m+mi}{1}\PY{p}{)}
    \PY{n}{y\PYZus{}scalar} \PY{o}{=} \PY{n}{my\PYZus{}scaler}\PY{p}{(}\PY{p}{)}
    \PY{n}{y\PYZus{}scalar}\PY{o}{.}\PY{n}{fit}\PY{p}{(}\PY{n}{y}\PY{p}{)}
    \PY{n}{y} \PY{o}{=} \PY{n}{y\PYZus{}scalar}\PY{o}{.}\PY{n}{transform}\PY{p}{(}\PY{n}{y}\PY{p}{)}
    \PY{c+c1}{\PYZsh{}\PYZhy{}\PYZhy{}\PYZhy{}\PYZhy{}\PYZhy{}\PYZhy{}\PYZhy{}\PYZhy{}\PYZhy{}\PYZhy{}\PYZhy{}\PYZhy{}\PYZhy{}\PYZhy{}\PYZhy{}\PYZhy{}\PYZhy{}\PYZhy{}\PYZhy{}\PYZhy{}\PYZhy{}\PYZhy{}\PYZhy{}\PYZhy{}\PYZhy{}\PYZhy{}\PYZhy{}\PYZhy{}\PYZhy{}\PYZhy{}\PYZhy{}\PYZhy{}}
    \PY{c+c1}{\PYZsh{} return X, y and y\PYZus{}scaler for inverse scaling}
    \PY{k}{return} \PY{n}{X}\PY{p}{,} \PY{n}{y}\PY{p}{,} \PY{n}{y\PYZus{}scalar}
\end{Verbatim}
\end{tcolorbox}

    \hypertarget{funktion-zur-vorbereitung-der-testdaten}{%
\subsubsection{Funktion zur Vorbereitung der
Testdaten}\label{funktion-zur-vorbereitung-der-testdaten}}

    \begin{tcolorbox}[breakable, size=fbox, boxrule=1pt, pad at break*=1mm,colback=cellbackground, colframe=cellborder]
\prompt{In}{incolor}{15}{\boxspacing}
\begin{Verbatim}[commandchars=\\\{\}]
\PY{c+c1}{\PYZsh{} Definition einer Funktion, die die Testdaten für das Submitten vorbereitetz}
\PY{c+c1}{\PYZsh{}\PYZhy{}\PYZhy{}\PYZhy{}\PYZhy{}\PYZhy{}\PYZhy{}\PYZhy{}\PYZhy{}\PYZhy{}\PYZhy{}\PYZhy{}\PYZhy{}}
\PY{c+c1}{\PYZsh{} Argumente:}
\PY{c+c1}{\PYZsh{} \PYZhy{} test\PYZus{}data: Zu bearbeitender DataFrame}
\PY{c+c1}{\PYZsh{}\PYZhy{}\PYZhy{}\PYZhy{}\PYZhy{}\PYZhy{}\PYZhy{}\PYZhy{}\PYZhy{}\PYZhy{}\PYZhy{}\PYZhy{}\PYZhy{}}

\PY{k}{def} \PY{n+nf}{make\PYZus{}test\PYZus{}data\PYZus{}fit}\PY{p}{(}\PY{n}{test\PYZus{}data}\PY{p}{,} \PY{n}{my\PYZus{}scaler}\PY{o}{=}\PY{n}{StandardScaler}\PY{p}{)}\PY{p}{:}
    \PY{c+c1}{\PYZsh{}\PYZhy{}\PYZhy{}\PYZhy{}\PYZhy{}\PYZhy{}\PYZhy{}\PYZhy{}\PYZhy{}\PYZhy{}\PYZhy{}\PYZhy{}\PYZhy{}\PYZhy{}\PYZhy{}\PYZhy{}\PYZhy{}\PYZhy{}\PYZhy{}\PYZhy{}\PYZhy{}\PYZhy{}\PYZhy{}\PYZhy{}\PYZhy{}\PYZhy{}\PYZhy{}\PYZhy{}\PYZhy{}\PYZhy{}\PYZhy{}\PYZhy{}\PYZhy{}}
    \PY{c+c1}{\PYZsh{} Make array out of DataFrame}
    \PY{n}{data} \PY{o}{=} \PY{n}{test\PYZus{}data}\PY{o}{.}\PY{n}{values}
    \PY{n}{X} \PY{o}{=} \PY{n}{data}\PY{p}{[}\PY{p}{:}\PY{p}{,}\PY{p}{:}\PY{p}{(}\PY{n+nb}{len}\PY{p}{(}\PY{n}{test\PYZus{}data}\PY{o}{.}\PY{n}{columns}\PY{p}{)} \PY{o}{\PYZhy{}} \PY{l+m+mi}{1}\PY{p}{)}\PY{p}{]}
    \PY{c+c1}{\PYZsh{}\PYZhy{}\PYZhy{}\PYZhy{}\PYZhy{}\PYZhy{}\PYZhy{}\PYZhy{}\PYZhy{}\PYZhy{}\PYZhy{}\PYZhy{}\PYZhy{}\PYZhy{}\PYZhy{}\PYZhy{}\PYZhy{}\PYZhy{}\PYZhy{}\PYZhy{}\PYZhy{}\PYZhy{}\PYZhy{}\PYZhy{}\PYZhy{}\PYZhy{}\PYZhy{}\PYZhy{}\PYZhy{}\PYZhy{}\PYZhy{}\PYZhy{}\PYZhy{}}
    \PY{c+c1}{\PYZsh{} Scale and reutrn}
    \PY{n}{scalar} \PY{o}{=} \PY{n}{my\PYZus{}scaler}\PY{p}{(}\PY{p}{)}
    \PY{n}{scalar}\PY{o}{.}\PY{n}{fit}\PY{p}{(}\PY{n}{X}\PY{p}{)}
    \PY{n}{X} \PY{o}{=} \PY{n}{scalar}\PY{o}{.}\PY{n}{transform}\PY{p}{(}\PY{n}{X}\PY{p}{)}    
    \PY{k}{return} \PY{n}{X}
\end{Verbatim}
\end{tcolorbox}

    \hypertarget{daten-einlesen-und-modifizieren}{%
\subsection{Daten einlesen und
modifizieren}\label{daten-einlesen-und-modifizieren}}

\hypertarget{ausgangsdaten-einlesen-und-anzeigen}{%
\subsubsection{Ausgangsdaten einlesen und
anzeigen}\label{ausgangsdaten-einlesen-und-anzeigen}}

    \begin{tcolorbox}[breakable, size=fbox, boxrule=1pt, pad at break*=1mm,colback=cellbackground, colframe=cellborder]
\prompt{In}{incolor}{16}{\boxspacing}
\begin{Verbatim}[commandchars=\\\{\}]
\PY{n}{train\PYZus{}data} \PY{o}{=} \PY{n}{pd}\PY{o}{.}\PY{n}{read\PYZus{}csv}\PY{p}{(}\PY{l+s+s2}{\PYZdq{}}\PY{l+s+s2}{train.csv}\PY{l+s+s2}{\PYZdq{}}\PY{p}{)}
\PY{n}{test\PYZus{}data} \PY{o}{=} \PY{n}{pd}\PY{o}{.}\PY{n}{read\PYZus{}csv}\PY{p}{(}\PY{l+s+s2}{\PYZdq{}}\PY{l+s+s2}{test.csv}\PY{l+s+s2}{\PYZdq{}}\PY{p}{)}
\PY{n}{test\PYZus{}data}\PY{p}{[}\PY{l+s+s1}{\PYZsq{}}\PY{l+s+s1}{SalePrice}\PY{l+s+s1}{\PYZsq{}}\PY{p}{]} \PY{o}{=} \PY{l+m+mi}{0}
\PY{n}{train\PYZus{}data}\PY{o}{.}\PY{n}{head}\PY{p}{(}\PY{p}{)}
\end{Verbatim}
\end{tcolorbox}

    \hypertarget{datensatz-um-nan-werte-bereinigen}{%
\subsubsection{Datensatz um NaN-Werte
bereinigen}\label{datensatz-um-nan-werte-bereinigen}}

Saplten mit \textgreater{} 30 fehlenden Einträgen werden gelöscht. Bei
Spalten in denen maximal 30 Einträge fehlen, werden diese durch den
häufigsten Wert ersetzt.

    \begin{tcolorbox}[breakable, size=fbox, boxrule=1pt, pad at break*=1mm,colback=cellbackground, colframe=cellborder]
\prompt{In}{incolor}{17}{\boxspacing}
\begin{Verbatim}[commandchars=\\\{\}]
\PY{c+c1}{\PYZsh{}\PYZhy{}\PYZhy{}\PYZhy{}\PYZhy{}\PYZhy{}\PYZhy{}\PYZhy{}\PYZhy{}\PYZhy{}\PYZhy{}\PYZhy{}\PYZhy{}\PYZhy{}\PYZhy{}\PYZhy{}\PYZhy{}\PYZhy{}\PYZhy{}\PYZhy{}\PYZhy{}\PYZhy{}\PYZhy{}\PYZhy{}\PYZhy{}\PYZhy{}\PYZhy{}\PYZhy{}\PYZhy{}\PYZhy{}\PYZhy{}\PYZhy{}\PYZhy{}}
\PY{c+c1}{\PYZsh{} Datenbereinigung }
\PY{n}{train\PYZus{}data} \PY{o}{=} \PY{n}{clean\PYZus{}my\PYZus{}data}\PY{p}{(}\PY{n}{train\PYZus{}data}\PY{p}{,} \PY{n}{upper\PYZus{}limit}\PY{o}{=}\PY{l+m+mi}{30}\PY{p}{)} 
\PY{n}{test\PYZus{}data} \PY{o}{=} \PY{n}{clean\PYZus{}my\PYZus{}data}\PY{p}{(}\PY{n}{test\PYZus{}data}\PY{p}{,} \PY{n}{upper\PYZus{}limit}\PY{o}{=}\PY{l+m+mi}{30}\PY{p}{)} 
\PY{c+c1}{\PYZsh{}\PYZhy{}\PYZhy{}\PYZhy{}\PYZhy{}\PYZhy{}\PYZhy{}\PYZhy{}\PYZhy{}\PYZhy{}\PYZhy{}\PYZhy{}\PYZhy{}\PYZhy{}\PYZhy{}\PYZhy{}\PYZhy{}\PYZhy{}\PYZhy{}\PYZhy{}\PYZhy{}\PYZhy{}\PYZhy{}\PYZhy{}\PYZhy{}\PYZhy{}\PYZhy{}\PYZhy{}\PYZhy{}\PYZhy{}\PYZhy{}\PYZhy{}\PYZhy{}}
\PY{c+c1}{\PYZsh{} Matchen}
\PY{n}{train\PYZus{}data}\PY{p}{,} \PY{n}{test\PYZus{}data} \PY{o}{=} \PY{n}{match\PYZus{}train\PYZus{}and\PYZus{}test}\PY{p}{(}\PY{n}{train\PYZus{}data}\PY{p}{,} \PY{n}{test\PYZus{}data}\PY{p}{)}
\end{Verbatim}
\end{tcolorbox}

    \hypertarget{datensatz-um-ausreiuxdfer-bereinigen}{%
\subsection{Datensatz um Ausreißer
bereinigen}\label{datensatz-um-ausreiuxdfer-bereinigen}}

\hypertarget{ermittlung-der-datentypen}{%
\subsubsection{Ermittlung der
Datentypen}\label{ermittlung-der-datentypen}}

Die Datentypen in den DataFrames werden ermittelt sodass anschließend
das One-Hot-Endocing für Spalten, welche Objekte enhtalten, durchgeführt
werden kann.

    \begin{tcolorbox}[breakable, size=fbox, boxrule=1pt, pad at break*=1mm,colback=cellbackground, colframe=cellborder]
\prompt{In}{incolor}{18}{\boxspacing}
\begin{Verbatim}[commandchars=\\\{\}]
\PY{c+c1}{\PYZsh{}\PYZhy{}\PYZhy{}\PYZhy{}\PYZhy{}\PYZhy{}\PYZhy{}\PYZhy{}\PYZhy{}\PYZhy{}\PYZhy{}\PYZhy{}\PYZhy{}\PYZhy{}\PYZhy{}\PYZhy{}\PYZhy{}\PYZhy{}\PYZhy{}\PYZhy{}\PYZhy{}\PYZhy{}\PYZhy{}\PYZhy{}\PYZhy{}\PYZhy{}\PYZhy{}\PYZhy{}\PYZhy{}\PYZhy{}\PYZhy{}\PYZhy{}\PYZhy{}}
\PY{c+c1}{\PYZsh{} Ermittlung Datentypen}
\PY{n}{object\PYZus{}list} \PY{o}{=} \PY{n}{separate\PYZus{}dtypes}\PY{p}{(}\PY{n}{train\PYZus{}data}\PY{p}{)}\PY{p}{[}\PY{l+s+s1}{\PYZsq{}}\PY{l+s+s1}{object}\PY{l+s+s1}{\PYZsq{}}\PY{p}{]} 
\PY{n}{object\PYZus{}list}\PY{o}{.}\PY{n}{append}\PY{p}{(}\PY{l+s+s1}{\PYZsq{}}\PY{l+s+s1}{BsmtExposure}\PY{l+s+s1}{\PYZsq{}}\PY{p}{)}
\end{Verbatim}
\end{tcolorbox}

    \hypertarget{durchfuxfchrung-one-hot-encoding}{%
\subsubsection{Durchführung
One-Hot-Encoding}\label{durchfuxfchrung-one-hot-encoding}}

Nominale Features werden rechenbar gemacht.

    \begin{tcolorbox}[breakable, size=fbox, boxrule=1pt, pad at break*=1mm,colback=cellbackground, colframe=cellborder]
\prompt{In}{incolor}{19}{\boxspacing}
\begin{Verbatim}[commandchars=\\\{\}]
\PY{c+c1}{\PYZsh{}\PYZhy{}\PYZhy{}\PYZhy{}\PYZhy{}\PYZhy{}\PYZhy{}\PYZhy{}\PYZhy{}\PYZhy{}\PYZhy{}\PYZhy{}\PYZhy{}\PYZhy{}\PYZhy{}\PYZhy{}\PYZhy{}\PYZhy{}\PYZhy{}\PYZhy{}\PYZhy{}\PYZhy{}\PYZhy{}\PYZhy{}\PYZhy{}\PYZhy{}\PYZhy{}\PYZhy{}\PYZhy{}\PYZhy{}\PYZhy{}\PYZhy{}\PYZhy{}}
\PY{c+c1}{\PYZsh{} Durchführung One\PYZhy{}Hot\PYZhy{}Encoding}
\PY{n}{train\PYZus{}data} \PY{o}{=} \PY{n}{encode}\PY{p}{(}\PY{n}{train\PYZus{}data}\PY{p}{,} \PY{n}{object\PYZus{}list}\PY{p}{)}
\PY{n}{test\PYZus{}data} \PY{o}{=} \PY{n}{encode}\PY{p}{(}\PY{n}{test\PYZus{}data}\PY{p}{,} \PY{n}{object\PYZus{}list}\PY{p}{)}
\PY{c+c1}{\PYZsh{}\PYZhy{}\PYZhy{}\PYZhy{}\PYZhy{}\PYZhy{}\PYZhy{}\PYZhy{}\PYZhy{}\PYZhy{}\PYZhy{}\PYZhy{}\PYZhy{}\PYZhy{}\PYZhy{}\PYZhy{}\PYZhy{}\PYZhy{}\PYZhy{}\PYZhy{}\PYZhy{}\PYZhy{}\PYZhy{}\PYZhy{}\PYZhy{}\PYZhy{}\PYZhy{}\PYZhy{}\PYZhy{}\PYZhy{}\PYZhy{}\PYZhy{}\PYZhy{}}
\PY{c+c1}{\PYZsh{} Matchen}
\PY{n}{train\PYZus{}data}\PY{p}{,} \PY{n}{test\PYZus{}data} \PY{o}{=} \PY{n}{match\PYZus{}train\PYZus{}and\PYZus{}test}\PY{p}{(}\PY{n}{train\PYZus{}data}\PY{p}{,} \PY{n}{test\PYZus{}data}\PY{p}{)}
\end{Verbatim}
\end{tcolorbox}

    \hypertarget{feature-selektion-korrelation}{%
\subsubsection{Feature Selektion
(Korrelation)}\label{feature-selektion-korrelation}}

Entfernung von Spalten, die sehr wenig mit der Zielgröße korrelieren.

    \begin{tcolorbox}[breakable, size=fbox, boxrule=1pt, pad at break*=1mm,colback=cellbackground, colframe=cellborder]
\prompt{In}{incolor}{22}{\boxspacing}
\begin{Verbatim}[commandchars=\\\{\}]
\PY{n+nb}{print}\PY{p}{(}\PY{n}{train\PYZus{}data}\PY{o}{.}\PY{n}{shape}\PY{p}{)}

\PY{c+c1}{\PYZsh{}\PYZhy{}\PYZhy{}\PYZhy{}\PYZhy{}\PYZhy{}\PYZhy{}\PYZhy{}\PYZhy{}\PYZhy{}\PYZhy{}\PYZhy{}\PYZhy{}\PYZhy{}\PYZhy{}\PYZhy{}\PYZhy{}\PYZhy{}\PYZhy{}\PYZhy{}\PYZhy{}\PYZhy{}\PYZhy{}\PYZhy{}\PYZhy{}\PYZhy{}\PYZhy{}\PYZhy{}\PYZhy{}\PYZhy{}\PYZhy{}\PYZhy{}\PYZhy{}}
\PY{c+c1}{\PYZsh{} Feature Selektion anhand der Korrelation}
\PY{n}{train\PYZus{}data} \PY{o}{=} \PY{n}{determine\PYZus{}corr\PYZus{}with\PYZus{}target\PYZus{}value}\PY{p}{(}\PY{n}{train\PYZus{}data}\PY{p}{,} \PY{n}{down\PYZus{}limit}\PY{o}{=}\PY{l+m+mf}{0.4}\PY{p}{)}
\PY{c+c1}{\PYZsh{}\PYZhy{}\PYZhy{}\PYZhy{}\PYZhy{}\PYZhy{}\PYZhy{}\PYZhy{}\PYZhy{}\PYZhy{}\PYZhy{}\PYZhy{}\PYZhy{}\PYZhy{}\PYZhy{}\PYZhy{}\PYZhy{}\PYZhy{}\PYZhy{}\PYZhy{}\PYZhy{}\PYZhy{}\PYZhy{}\PYZhy{}\PYZhy{}\PYZhy{}\PYZhy{}\PYZhy{}\PYZhy{}\PYZhy{}\PYZhy{}\PYZhy{}\PYZhy{}}
\PY{c+c1}{\PYZsh{} Matchen}
\PY{n}{train\PYZus{}data}\PY{p}{,} \PY{n}{test\PYZus{}data} \PY{o}{=} \PY{n}{match\PYZus{}train\PYZus{}and\PYZus{}test}\PY{p}{(}\PY{n}{train\PYZus{}data}\PY{p}{,} \PY{n}{test\PYZus{}data}\PY{p}{)}
\PY{c+c1}{\PYZsh{}\PYZhy{}\PYZhy{}\PYZhy{}\PYZhy{}\PYZhy{}\PYZhy{}\PYZhy{}\PYZhy{}\PYZhy{}\PYZhy{}\PYZhy{}\PYZhy{}\PYZhy{}\PYZhy{}\PYZhy{}\PYZhy{}\PYZhy{}\PYZhy{}\PYZhy{}\PYZhy{}\PYZhy{}\PYZhy{}\PYZhy{}\PYZhy{}\PYZhy{}\PYZhy{}\PYZhy{}\PYZhy{}\PYZhy{}\PYZhy{}\PYZhy{}\PYZhy{}}
\PY{c+c1}{\PYZsh{} Bug beheben}
\PY{n}{trainSale} \PY{o}{=} \PY{n}{train\PYZus{}data}\PY{p}{[}\PY{l+s+s1}{\PYZsq{}}\PY{l+s+s1}{SalePrice}\PY{l+s+s1}{\PYZsq{}}\PY{p}{]}\PY{o}{.}\PY{n}{values}\PY{o}{.}\PY{n}{tolist}\PY{p}{(}\PY{p}{)}
\PY{n}{train\PYZus{}data} \PY{o}{=} \PY{n}{train\PYZus{}data}\PY{o}{.}\PY{n}{drop}\PY{p}{(}\PY{l+s+s1}{\PYZsq{}}\PY{l+s+s1}{SalePrice}\PY{l+s+s1}{\PYZsq{}}\PY{p}{,} \PY{n}{axis}\PY{o}{=}\PY{l+m+mi}{1}\PY{p}{)}
\PY{n}{test\PYZus{}data} \PY{o}{=} \PY{n}{test\PYZus{}data}\PY{o}{.}\PY{n}{drop}\PY{p}{(}\PY{l+s+s1}{\PYZsq{}}\PY{l+s+s1}{SalePrice}\PY{l+s+s1}{\PYZsq{}}\PY{p}{,} \PY{n}{axis}\PY{o}{=}\PY{l+m+mi}{1}\PY{p}{)}
\PY{n}{train\PYZus{}data}\PY{p}{[}\PY{l+s+s1}{\PYZsq{}}\PY{l+s+s1}{SalePrice}\PY{l+s+s1}{\PYZsq{}}\PY{p}{]} \PY{o}{=} \PY{n}{trainSale}

\PY{n+nb}{print}\PY{p}{(}\PY{n}{train\PYZus{}data}\PY{o}{.}\PY{n}{shape}\PY{p}{)}

\PY{n}{train\PYZus{}data}\PY{o}{.}\PY{n}{head}\PY{p}{(}\PY{p}{)}
\end{Verbatim}
\end{tcolorbox}

    \begin{Verbatim}[commandchars=\\\{\}]
(460, 224)
(460, 20)
    \end{Verbatim}

            \begin{tcolorbox}[breakable, size=fbox, boxrule=.5pt, pad at break*=1mm, opacityfill=0]
\prompt{Out}{outcolor}{22}{\boxspacing}
\begin{Verbatim}[commandchars=\\\{\}]
      OverallQual  YearBuilt  YearRemodAdd  TotalBsmtSF  1stFlrSF  GrLivArea
FullBath  TotRmsAbvGrd  Fireplaces  GarageCars  GarageArea  Ex21  TA2  NridgHt
Ex22  TA22  PConc   Ex   TA  SalePrice
Id
670             4       1922          1950          700      1180       1180
1             5           1           1         252   0.0  0.0      0.0   0.0
1.0    0.0  0.0  0.0     137500
958             5       1962          1962         1057      1057       1057
1             6           0           2         576   0.0  1.0      0.0   0.0
1.0    0.0  0.0  1.0     132000
614             5       2007          2007         1120      1120       1120
1             6           0           0           0   0.0  0.0      0.0   0.0
1.0    1.0  0.0  1.0     147000
83              8       2007          2007         1563      1563       1563
2             6           1           3         758   0.0  0.0      0.0   0.0
1.0    1.0  0.0  0.0     245000
1183           10       1996          1996         2396      2411       4476
3            10           2           3         813   1.0  0.0      0.0   0.0
0.0    1.0  1.0  0.0     745000
\end{Verbatim}
\end{tcolorbox}
        
    \hypertarget{korrelation-aller-features-untereinander}{%
\subsection{Korrelation aller Features
untereinander}\label{korrelation-aller-features-untereinander}}

Überprüfung inwieweit sich weitere Features entfernen lassen, da sie
stark mit einem anderen starker mit der Zielgröße korrelierenden Feature
zusammenhängen. \#\#\# Visualisierung

    \begin{tcolorbox}[breakable, size=fbox, boxrule=1pt, pad at break*=1mm,colback=cellbackground, colframe=cellborder]
\prompt{In}{incolor}{23}{\boxspacing}
\begin{Verbatim}[commandchars=\\\{\}]
\PY{c+c1}{\PYZsh{}\PYZhy{}\PYZhy{}\PYZhy{}\PYZhy{}\PYZhy{}\PYZhy{}\PYZhy{}\PYZhy{}\PYZhy{}\PYZhy{}\PYZhy{}\PYZhy{}\PYZhy{}\PYZhy{}\PYZhy{}\PYZhy{}\PYZhy{}\PYZhy{}\PYZhy{}\PYZhy{}\PYZhy{}\PYZhy{}\PYZhy{}\PYZhy{}\PYZhy{}\PYZhy{}\PYZhy{}\PYZhy{}\PYZhy{}\PYZhy{}\PYZhy{}\PYZhy{}}
\PY{c+c1}{\PYZsh{} Heatmap Plotten}
\PY{n}{X} \PY{o}{=} \PY{n}{train\PYZus{}data}\PY{o}{.}\PY{n}{drop}\PY{p}{(}\PY{l+s+s1}{\PYZsq{}}\PY{l+s+s1}{SalePrice}\PY{l+s+s1}{\PYZsq{}}\PY{p}{,}\PY{n}{axis}\PY{o}{=}\PY{l+s+s1}{\PYZsq{}}\PY{l+s+s1}{columns}\PY{l+s+s1}{\PYZsq{}}\PY{p}{)} \PY{c+c1}{\PYZsh{} Independent columns}
\PY{n}{y} \PY{o}{=} \PY{n}{train\PYZus{}data}\PY{p}{[}\PY{l+s+s1}{\PYZsq{}}\PY{l+s+s1}{SalePrice}\PY{l+s+s1}{\PYZsq{}}\PY{p}{]} \PY{c+c1}{\PYZsh{} Target column}
\PY{n}{corrmat} \PY{o}{=} \PY{n}{train\PYZus{}data}\PY{o}{.}\PY{n}{corr}\PY{p}{(}\PY{p}{)} \PY{c+c1}{\PYZsh{} Get correlations of each features in dataset}
\PY{n}{top\PYZus{}corr\PYZus{}features} \PY{o}{=} \PY{n}{corrmat}\PY{o}{.}\PY{n}{index}
\PY{n}{plt}\PY{o}{.}\PY{n}{figure}\PY{p}{(}\PY{n}{figsize}\PY{o}{=}\PY{p}{(}\PY{l+m+mi}{40}\PY{p}{,}\PY{l+m+mi}{20}\PY{p}{)}\PY{p}{)} \PY{c+c1}{\PYZsh{} Größe Grafik}
\PY{n}{ax} \PY{o}{=} \PY{n}{sns}\PY{o}{.}\PY{n}{heatmap}\PY{p}{(}\PY{n}{train\PYZus{}data}\PY{p}{[}\PY{n}{top\PYZus{}corr\PYZus{}features}\PY{p}{]}\PY{o}{.}\PY{n}{corr}\PY{p}{(}\PY{p}{)}\PY{p}{,}\PY{n}{annot}\PY{o}{=}\PY{k+kc}{True}\PY{p}{,}\PY{n}{cmap}\PY{o}{=}\PY{l+s+s2}{\PYZdq{}}\PY{l+s+s2}{RdYlGn}\PY{l+s+s2}{\PYZdq{}}\PY{p}{)} \PY{c+c1}{\PYZsh{} Plot heat map}
\PY{n}{bottom}\PY{p}{,} \PY{n}{top} \PY{o}{=} \PY{n}{ax}\PY{o}{.}\PY{n}{get\PYZus{}ylim}\PY{p}{(}\PY{p}{)} \PY{c+c1}{\PYZsh{} Get limits X\PYZhy{}, Y\PYZhy{}Achse                 }
\PY{n}{ax}\PY{o}{.}\PY{n}{set\PYZus{}ylim}\PY{p}{(}\PY{n}{bottom} \PY{o}{+} \PY{l+m+mf}{0.5}\PY{p}{,} \PY{n}{top} \PY{o}{\PYZhy{}} \PY{l+m+mf}{0.5}\PY{p}{)} \PY{c+c1}{\PYZsh{} Erweiterterung der Limits}
\end{Verbatim}
\end{tcolorbox}

            \begin{tcolorbox}[breakable, size=fbox, boxrule=.5pt, pad at break*=1mm, opacityfill=0]
\prompt{Out}{outcolor}{23}{\boxspacing}
\begin{Verbatim}[commandchars=\\\{\}]
(20.0, 0.0)
\end{Verbatim}
\end{tcolorbox}
        
    \begin{center}
    \adjustimage{max size={0.9\linewidth}{0.9\paperheight}}{output_36_1.png}
    \end{center}
    { \hspace*{\fill} \\}
    
    \hypertarget{entfernung-von-untereinander-stark-korrelierten-features}{%
\subsubsection{Entfernung von untereinander stark korrelierten
Features}\label{entfernung-von-untereinander-stark-korrelierten-features}}

    \begin{tcolorbox}[breakable, size=fbox, boxrule=1pt, pad at break*=1mm,colback=cellbackground, colframe=cellborder]
\prompt{In}{incolor}{24}{\boxspacing}
\begin{Verbatim}[commandchars=\\\{\}]
\PY{n}{\PYZus{}} \PY{o}{=} \PY{n}{train\PYZus{}data}\PY{o}{.}\PY{n}{drop}\PY{p}{(}\PY{l+s+s1}{\PYZsq{}}\PY{l+s+s1}{1stFlrSF}\PY{l+s+s1}{\PYZsq{}}\PY{p}{,}\PY{n}{axis}\PY{o}{=}\PY{l+s+s1}{\PYZsq{}}\PY{l+s+s1}{columns}\PY{l+s+s1}{\PYZsq{}}\PY{p}{)}
\PY{c+c1}{\PYZsh{}train\PYZus{}data.drop(\PYZsq{}TotalBsmtSF\PYZsq{},axis=\PYZsq{}columns\PYZsq{})}

\PY{n}{\PYZus{}} \PY{o}{=} \PY{n}{train\PYZus{}data}\PY{o}{.}\PY{n}{drop}\PY{p}{(}\PY{l+s+s1}{\PYZsq{}}\PY{l+s+s1}{GrLivArea}\PY{l+s+s1}{\PYZsq{}}\PY{p}{,}\PY{n}{axis}\PY{o}{=}\PY{l+s+s1}{\PYZsq{}}\PY{l+s+s1}{columns}\PY{l+s+s1}{\PYZsq{}}\PY{p}{)}
\PY{c+c1}{\PYZsh{}train\PYZus{}data.drop(\PYZsq{}TotRmsAbvGrd\PYZsq{},axis=\PYZsq{}columns\PYZsq{})}

\PY{n}{\PYZus{}} \PY{o}{=} \PY{n}{train\PYZus{}data}\PY{o}{.}\PY{n}{drop}\PY{p}{(}\PY{l+s+s1}{\PYZsq{}}\PY{l+s+s1}{GarageCars}\PY{l+s+s1}{\PYZsq{}}\PY{p}{,}\PY{n}{axis}\PY{o}{=}\PY{l+s+s1}{\PYZsq{}}\PY{l+s+s1}{columns}\PY{l+s+s1}{\PYZsq{}}\PY{p}{)}
\PY{c+c1}{\PYZsh{}train\PYZus{}data.drop(\PYZsq{}GarageArea\PYZsq{},axis=\PYZsq{}columns\PYZsq{})}
\PY{n}{train\PYZus{}data}\PY{o}{.}\PY{n}{shape}
\end{Verbatim}
\end{tcolorbox}

            \begin{tcolorbox}[breakable, size=fbox, boxrule=.5pt, pad at break*=1mm, opacityfill=0]
\prompt{Out}{outcolor}{24}{\boxspacing}
\begin{Verbatim}[commandchars=\\\{\}]
(460, 20)
\end{Verbatim}
\end{tcolorbox}
        
    \hypertarget{trainigsdatensatz-skalieren-und-in-teildatatensuxe4tze-splitten}{%
\subsubsection{Trainigsdatensatz skalieren und in Teildatatensätze
splitten}\label{trainigsdatensatz-skalieren-und-in-teildatatensuxe4tze-splitten}}

Hinweis: Zum finalen Traing (so wie hier geschehen) wird der gesamte zur
Verfügung stehende Traingsdatensatz verwednet, da dies bessere
Ergebnisse auf Kaggle mit sich brachte.

    \begin{tcolorbox}[breakable, size=fbox, boxrule=1pt, pad at break*=1mm,colback=cellbackground, colframe=cellborder]
\prompt{In}{incolor}{25}{\boxspacing}
\begin{Verbatim}[commandchars=\\\{\}]
\PY{c+c1}{\PYZsh{}\PYZhy{}\PYZhy{}\PYZhy{}\PYZhy{}\PYZhy{}\PYZhy{}\PYZhy{}\PYZhy{}\PYZhy{}\PYZhy{}\PYZhy{}\PYZhy{}\PYZhy{}\PYZhy{}\PYZhy{}\PYZhy{}\PYZhy{}\PYZhy{}\PYZhy{}\PYZhy{}\PYZhy{}\PYZhy{}\PYZhy{}\PYZhy{}\PYZhy{}\PYZhy{}\PYZhy{}\PYZhy{}\PYZhy{}\PYZhy{}\PYZhy{}\PYZhy{}}
\PY{c+c1}{\PYZsh{} Skalierung}
\PY{n}{X}\PY{p}{,} \PY{n}{y}\PY{p}{,} \PY{n}{y\PYZus{}scalar} \PY{o}{=} \PY{n}{scale\PYZus{}it}\PY{p}{(}\PY{n}{train\PYZus{}data}\PY{p}{,} \PY{n}{my\PYZus{}scaler}\PY{o}{=}\PY{n}{RobustScaler}\PY{p}{)}
\PY{c+c1}{\PYZsh{}\PYZhy{}\PYZhy{}\PYZhy{}\PYZhy{}\PYZhy{}\PYZhy{}\PYZhy{}\PYZhy{}\PYZhy{}\PYZhy{}\PYZhy{}\PYZhy{}\PYZhy{}\PYZhy{}\PYZhy{}\PYZhy{}\PYZhy{}\PYZhy{}\PYZhy{}\PYZhy{}\PYZhy{}\PYZhy{}\PYZhy{}\PYZhy{}\PYZhy{}\PYZhy{}\PYZhy{}\PYZhy{}\PYZhy{}\PYZhy{}\PYZhy{}\PYZhy{}}
\PY{c+c1}{\PYZsh{} Splitten}
\PY{n}{X\PYZus{}train}\PY{p}{,} \PY{n}{X\PYZus{}validierung}\PY{p}{,} \PY{n}{y\PYZus{}train}\PY{p}{,} \PY{n}{y\PYZus{}validierung} \PY{o}{=} \PY{n}{train\PYZus{}test\PYZus{}split}\PY{p}{(}\PY{n}{X}\PY{p}{,} \PY{n}{y}\PY{p}{,} \PY{n}{test\PYZus{}size}\PY{o}{=}\PY{l+m+mf}{0.2}\PY{p}{,} \PY{n}{random\PYZus{}state}\PY{o}{=}\PY{l+m+mi}{2121}\PY{p}{)}
\end{Verbatim}
\end{tcolorbox}

    \hypertarget{prognose-anhand-eines-knn-mittels-keras-regression}{%
\subsection{Prognose anhand eines KNN mittels Keras
(Regression)}\label{prognose-anhand-eines-knn-mittels-keras-regression}}

\hypertarget{modell-defininieren}{%
\subsubsection{Modell defininieren}\label{modell-defininieren}}

Die Architiektur des Modells wird definiert.

    \begin{tcolorbox}[breakable, size=fbox, boxrule=1pt, pad at break*=1mm,colback=cellbackground, colframe=cellborder]
\prompt{In}{incolor}{26}{\boxspacing}
\begin{Verbatim}[commandchars=\\\{\}]
\PY{k}{def} \PY{n+nf}{create\PYZus{}model}\PY{p}{(} \PY{n}{my\PYZus{}layer\PYZus{}units}\PY{p}{,} \PY{n}{my\PYZus{}learn\PYZus{}rate}\PY{p}{,} \PY{n}{my\PYZus{}dropout}\PY{p}{,}
                 \PY{n}{my\PYZus{}weight\PYZus{}constraint}\PY{p}{,} \PY{n}{my\PYZus{}regularizer}\PY{p}{,} \PY{n}{my\PYZus{}kernel\PYZus{}initializer}\PY{p}{,} \PY{n}{my\PYZus{}activation}\PY{p}{,} \PY{n}{input\PYZus{}dim}\PY{o}{=}\PY{n}{X\PYZus{}train}\PY{o}{.}\PY{n}{shape}\PY{p}{[}\PY{l+m+mi}{1}\PY{p}{]}\PY{p}{)}\PY{p}{:}
    
    \PY{c+c1}{\PYZsh{}\PYZhy{}\PYZhy{}\PYZhy{}\PYZhy{}\PYZhy{}\PYZhy{}\PYZhy{}\PYZhy{}\PYZhy{}\PYZhy{}\PYZhy{}\PYZhy{}\PYZhy{}\PYZhy{}\PYZhy{}\PYZhy{}\PYZhy{}\PYZhy{}\PYZhy{}\PYZhy{}\PYZhy{}\PYZhy{}\PYZhy{}\PYZhy{}\PYZhy{}\PYZhy{}\PYZhy{}\PYZhy{}\PYZhy{}\PYZhy{}\PYZhy{}\PYZhy{}}
    \PY{c+c1}{\PYZsh{} Create model and define optimizer}
    \PY{n}{model} \PY{o}{=} \PY{n}{Sequential}\PY{p}{(}\PY{p}{)}
    \PY{n}{my\PYZus{}optimizer} \PY{o}{=} \PY{n}{optimizers}\PY{o}{.}\PY{n}{Adam}\PY{p}{(}\PY{n}{lr}\PY{o}{=}\PY{n}{my\PYZus{}learn\PYZus{}rate}\PY{p}{)}
    
    \PY{c+c1}{\PYZsh{}\PYZhy{}\PYZhy{}\PYZhy{}\PYZhy{}\PYZhy{}\PYZhy{}\PYZhy{}\PYZhy{}\PYZhy{}\PYZhy{}\PYZhy{}\PYZhy{}\PYZhy{}\PYZhy{}\PYZhy{}\PYZhy{}\PYZhy{}\PYZhy{}\PYZhy{}\PYZhy{}\PYZhy{}\PYZhy{}\PYZhy{}\PYZhy{}\PYZhy{}\PYZhy{}\PYZhy{}\PYZhy{}\PYZhy{}\PYZhy{}\PYZhy{}\PYZhy{}}
    \PY{c+c1}{\PYZsh{} Input layer}
    \PY{n}{model}\PY{o}{.}\PY{n}{add}\PY{p}{(}\PY{n}{Dense}\PY{p}{(}\PY{n+nb}{int}\PY{p}{(}\PY{n}{my\PYZus{}layer\PYZus{}units} \PY{o}{*} \PY{l+m+mi}{1}\PY{p}{)}\PY{p}{,} \PY{n}{input\PYZus{}dim}\PY{o}{=}\PY{n}{input\PYZus{}dim}\PY{p}{,} \PY{n}{kernel\PYZus{}regularizer}\PY{o}{=}\PY{n}{my\PYZus{}regularizer}\PY{p}{,}
                    \PY{n}{kernel\PYZus{}initializer}\PY{o}{=}\PY{n}{my\PYZus{}kernel\PYZus{}initializer}\PY{p}{,} \PY{n}{activation}\PY{o}{=}\PY{n}{my\PYZus{}activation}\PY{p}{,}
                    \PY{n}{kernel\PYZus{}constraint}\PY{o}{=}\PY{n}{maxnorm}\PY{p}{(}\PY{n}{my\PYZus{}weight\PYZus{}constraint}\PY{p}{)}\PY{p}{)}\PY{p}{)}
    
    \PY{c+c1}{\PYZsh{}\PYZhy{}\PYZhy{}\PYZhy{}\PYZhy{}\PYZhy{}\PYZhy{}\PYZhy{}\PYZhy{}\PYZhy{}\PYZhy{}\PYZhy{}\PYZhy{}\PYZhy{}\PYZhy{}\PYZhy{}\PYZhy{}\PYZhy{}\PYZhy{}\PYZhy{}\PYZhy{}\PYZhy{}\PYZhy{}\PYZhy{}\PYZhy{}\PYZhy{}\PYZhy{}\PYZhy{}\PYZhy{}\PYZhy{}\PYZhy{}\PYZhy{}\PYZhy{}}
    \PY{c+c1}{\PYZsh{} Hidden layer 1         }
    \PY{n}{model}\PY{o}{.}\PY{n}{add}\PY{p}{(}\PY{n}{Dense}\PY{p}{(}\PY{n+nb}{int}\PY{p}{(}\PY{n}{my\PYZus{}layer\PYZus{}units} \PY{o}{*} \PY{l+m+mi}{1}\PY{p}{)}\PY{p}{,} \PY{n}{kernel\PYZus{}regularizer}\PY{o}{=}\PY{n}{my\PYZus{}regularizer}\PY{p}{,}
                    \PY{n}{kernel\PYZus{}initializer}\PY{o}{=}\PY{n}{my\PYZus{}kernel\PYZus{}initializer}\PY{p}{,} \PY{n}{activation}\PY{o}{=}\PY{n}{my\PYZus{}activation}\PY{p}{,}
                    \PY{n}{kernel\PYZus{}constraint}\PY{o}{=}\PY{n}{maxnorm}\PY{p}{(}\PY{n}{my\PYZus{}weight\PYZus{}constraint}\PY{p}{)}\PY{p}{)}\PY{p}{)}
    \PY{c+c1}{\PYZsh{}\PYZhy{}\PYZhy{}\PYZhy{}\PYZhy{}\PYZhy{}\PYZhy{}\PYZhy{}\PYZhy{}\PYZhy{}\PYZhy{}\PYZhy{}\PYZhy{}\PYZhy{}\PYZhy{}\PYZhy{}\PYZhy{}\PYZhy{}\PYZhy{}\PYZhy{}\PYZhy{}\PYZhy{}\PYZhy{}\PYZhy{}\PYZhy{}\PYZhy{}\PYZhy{}\PYZhy{}\PYZhy{}\PYZhy{}\PYZhy{}\PYZhy{}\PYZhy{}}
    \PY{c+c1}{\PYZsh{} Dropout und hidden layer 2}
    \PY{n}{model}\PY{o}{.}\PY{n}{add}\PY{p}{(}\PY{n}{Dropout}\PY{p}{(}\PY{n}{my\PYZus{}dropout}\PY{p}{)}\PY{p}{)} 
    \PY{n}{model}\PY{o}{.}\PY{n}{add}\PY{p}{(}\PY{n}{Dense}\PY{p}{(}\PY{n+nb}{int}\PY{p}{(}\PY{n}{my\PYZus{}layer\PYZus{}units} \PY{o}{*} \PY{l+m+mi}{1}\PY{p}{)}\PY{p}{,} \PY{n}{kernel\PYZus{}regularizer}\PY{o}{=}\PY{n}{my\PYZus{}regularizer}\PY{p}{,}
                    \PY{n}{kernel\PYZus{}initializer}\PY{o}{=}\PY{n}{my\PYZus{}kernel\PYZus{}initializer}\PY{p}{,} \PY{n}{activation}\PY{o}{=}\PY{n}{my\PYZus{}activation}\PY{p}{,}
                    \PY{n}{kernel\PYZus{}constraint}\PY{o}{=}\PY{n}{maxnorm}\PY{p}{(}\PY{n}{my\PYZus{}weight\PYZus{}constraint}\PY{p}{)}\PY{p}{)}\PY{p}{)}
    
    \PY{c+c1}{\PYZsh{}\PYZhy{}\PYZhy{}\PYZhy{}\PYZhy{}\PYZhy{}\PYZhy{}\PYZhy{}\PYZhy{}\PYZhy{}\PYZhy{}\PYZhy{}\PYZhy{}\PYZhy{}\PYZhy{}\PYZhy{}\PYZhy{}\PYZhy{}\PYZhy{}\PYZhy{}\PYZhy{}\PYZhy{}\PYZhy{}\PYZhy{}\PYZhy{}\PYZhy{}\PYZhy{}\PYZhy{}\PYZhy{}\PYZhy{}\PYZhy{}\PYZhy{}\PYZhy{}}
    \PY{c+c1}{\PYZsh{} Dropout und hidden layer 3        }
    \PY{n}{model}\PY{o}{.}\PY{n}{add}\PY{p}{(}\PY{n}{Dropout}\PY{p}{(}\PY{n}{my\PYZus{}dropout}\PY{p}{)}\PY{p}{)} 
    \PY{n}{model}\PY{o}{.}\PY{n}{add}\PY{p}{(}\PY{n}{Dense}\PY{p}{(}\PY{n+nb}{int}\PY{p}{(}\PY{n}{my\PYZus{}layer\PYZus{}units} \PY{o}{*} \PY{l+m+mi}{1}\PY{p}{)}\PY{p}{,} \PY{n}{kernel\PYZus{}regularizer}\PY{o}{=}\PY{n}{my\PYZus{}regularizer}\PY{p}{,}
                    \PY{n}{kernel\PYZus{}initializer}\PY{o}{=}\PY{n}{my\PYZus{}kernel\PYZus{}initializer}\PY{p}{,}  \PY{n}{activation}\PY{o}{=}\PY{n}{my\PYZus{}activation}\PY{p}{,}
                    \PY{n}{kernel\PYZus{}constraint}\PY{o}{=}\PY{n}{maxnorm}\PY{p}{(}\PY{n}{my\PYZus{}weight\PYZus{}constraint}\PY{p}{)}\PY{p}{)}\PY{p}{)}
    
    \PY{c+c1}{\PYZsh{}\PYZhy{}\PYZhy{}\PYZhy{}\PYZhy{}\PYZhy{}\PYZhy{}\PYZhy{}\PYZhy{}\PYZhy{}\PYZhy{}\PYZhy{}\PYZhy{}\PYZhy{}\PYZhy{}\PYZhy{}\PYZhy{}\PYZhy{}\PYZhy{}\PYZhy{}\PYZhy{}\PYZhy{}\PYZhy{}\PYZhy{}\PYZhy{}\PYZhy{}\PYZhy{}\PYZhy{}\PYZhy{}\PYZhy{}\PYZhy{}\PYZhy{}\PYZhy{}}
    \PY{c+c1}{\PYZsh{} Dropout und hidden layer 4      }
    \PY{n}{model}\PY{o}{.}\PY{n}{add}\PY{p}{(}\PY{n}{Dropout}\PY{p}{(}\PY{n}{my\PYZus{}dropout}\PY{p}{)}\PY{p}{)} 
    \PY{n}{model}\PY{o}{.}\PY{n}{add}\PY{p}{(}\PY{n}{Dense}\PY{p}{(}\PY{n+nb}{int}\PY{p}{(}\PY{n}{my\PYZus{}layer\PYZus{}units} \PY{o}{*} \PY{l+m+mi}{1}\PY{p}{)}\PY{p}{,} \PY{n}{kernel\PYZus{}regularizer}\PY{o}{=}\PY{n}{my\PYZus{}regularizer}\PY{p}{,}
                    \PY{n}{kernel\PYZus{}initializer}\PY{o}{=}\PY{n}{my\PYZus{}kernel\PYZus{}initializer}\PY{p}{,}  \PY{n}{activation}\PY{o}{=}\PY{n}{my\PYZus{}activation}\PY{p}{,}
                    \PY{n}{kernel\PYZus{}constraint}\PY{o}{=}\PY{n}{maxnorm}\PY{p}{(}\PY{n}{my\PYZus{}weight\PYZus{}constraint}\PY{p}{)}\PY{p}{)}\PY{p}{)}
    
    \PY{c+c1}{\PYZsh{}\PYZhy{}\PYZhy{}\PYZhy{}\PYZhy{}\PYZhy{}\PYZhy{}\PYZhy{}\PYZhy{}\PYZhy{}\PYZhy{}\PYZhy{}\PYZhy{}\PYZhy{}\PYZhy{}\PYZhy{}\PYZhy{}\PYZhy{}\PYZhy{}\PYZhy{}\PYZhy{}\PYZhy{}\PYZhy{}\PYZhy{}\PYZhy{}\PYZhy{}\PYZhy{}\PYZhy{}\PYZhy{}\PYZhy{}\PYZhy{}\PYZhy{}\PYZhy{}}
    \PY{c+c1}{\PYZsh{} Dropout und hidden layer 5    }
    \PY{n}{model}\PY{o}{.}\PY{n}{add}\PY{p}{(}\PY{n}{Dropout}\PY{p}{(}\PY{n}{my\PYZus{}dropout}\PY{p}{)}\PY{p}{)} 
    \PY{n}{model}\PY{o}{.}\PY{n}{add}\PY{p}{(}\PY{n}{Dense}\PY{p}{(}\PY{n+nb}{int}\PY{p}{(}\PY{n}{my\PYZus{}layer\PYZus{}units} \PY{o}{*} \PY{l+m+mi}{1}\PY{p}{)}\PY{p}{,} \PY{n}{kernel\PYZus{}regularizer}\PY{o}{=}\PY{n}{my\PYZus{}regularizer}\PY{p}{,}
                    \PY{n}{kernel\PYZus{}initializer}\PY{o}{=}\PY{n}{my\PYZus{}kernel\PYZus{}initializer}\PY{p}{,}  \PY{n}{activation}\PY{o}{=}\PY{n}{my\PYZus{}activation}\PY{p}{,}
                    \PY{n}{kernel\PYZus{}constraint}\PY{o}{=}\PY{n}{maxnorm}\PY{p}{(}\PY{n}{my\PYZus{}weight\PYZus{}constraint}\PY{p}{)}\PY{p}{)}\PY{p}{)}
    
    \PY{c+c1}{\PYZsh{}\PYZhy{}\PYZhy{}\PYZhy{}\PYZhy{}\PYZhy{}\PYZhy{}\PYZhy{}\PYZhy{}\PYZhy{}\PYZhy{}\PYZhy{}\PYZhy{}\PYZhy{}\PYZhy{}\PYZhy{}\PYZhy{}\PYZhy{}\PYZhy{}\PYZhy{}\PYZhy{}\PYZhy{}\PYZhy{}\PYZhy{}\PYZhy{}\PYZhy{}\PYZhy{}\PYZhy{}\PYZhy{}\PYZhy{}\PYZhy{}\PYZhy{}\PYZhy{}}
    \PY{c+c1}{\PYZsh{} Dropout und hidden layer 6        }
    \PY{n}{model}\PY{o}{.}\PY{n}{add}\PY{p}{(}\PY{n}{Dropout}\PY{p}{(}\PY{n}{my\PYZus{}dropout}\PY{p}{)}\PY{p}{)} 
    \PY{n}{model}\PY{o}{.}\PY{n}{add}\PY{p}{(}\PY{n}{Dense}\PY{p}{(}\PY{n+nb}{int}\PY{p}{(}\PY{n}{my\PYZus{}layer\PYZus{}units} \PY{o}{*} \PY{l+m+mi}{1}\PY{p}{)}\PY{p}{,} \PY{n}{kernel\PYZus{}regularizer}\PY{o}{=}\PY{n}{my\PYZus{}regularizer}\PY{p}{,}
                    \PY{n}{kernel\PYZus{}initializer}\PY{o}{=}\PY{n}{my\PYZus{}kernel\PYZus{}initializer}\PY{p}{,}  \PY{n}{activation}\PY{o}{=}\PY{n}{my\PYZus{}activation}\PY{p}{,}
                    \PY{n}{kernel\PYZus{}constraint}\PY{o}{=}\PY{n}{maxnorm}\PY{p}{(}\PY{n}{my\PYZus{}weight\PYZus{}constraint}\PY{p}{)}\PY{p}{)}\PY{p}{)}
    
    \PY{c+c1}{\PYZsh{}\PYZhy{}\PYZhy{}\PYZhy{}\PYZhy{}\PYZhy{}\PYZhy{}\PYZhy{}\PYZhy{}\PYZhy{}\PYZhy{}\PYZhy{}\PYZhy{}\PYZhy{}\PYZhy{}\PYZhy{}\PYZhy{}\PYZhy{}\PYZhy{}\PYZhy{}\PYZhy{}\PYZhy{}\PYZhy{}\PYZhy{}\PYZhy{}\PYZhy{}\PYZhy{}\PYZhy{}\PYZhy{}\PYZhy{}\PYZhy{}\PYZhy{}\PYZhy{}}
    \PY{c+c1}{\PYZsh{} Output layer}
    \PY{n}{model}\PY{o}{.}\PY{n}{add}\PY{p}{(}\PY{n}{Dense}\PY{p}{(}\PY{l+m+mi}{1}\PY{p}{,} \PY{n}{activation}\PY{o}{=}\PY{l+s+s1}{\PYZsq{}}\PY{l+s+s1}{linear}\PY{l+s+s1}{\PYZsq{}}\PY{p}{)}\PY{p}{)}
    \PY{c+c1}{\PYZsh{}\PYZhy{}\PYZhy{}\PYZhy{}\PYZhy{}\PYZhy{}\PYZhy{}\PYZhy{}\PYZhy{}\PYZhy{}\PYZhy{}\PYZhy{}\PYZhy{}\PYZhy{}\PYZhy{}\PYZhy{}\PYZhy{}\PYZhy{}\PYZhy{}\PYZhy{}\PYZhy{}\PYZhy{}\PYZhy{}\PYZhy{}\PYZhy{}\PYZhy{}\PYZhy{}\PYZhy{}\PYZhy{}\PYZhy{}\PYZhy{}\PYZhy{}\PYZhy{}}
    \PY{c+c1}{\PYZsh{} Compile model and return    }
    \PY{n}{model}\PY{o}{.}\PY{n}{compile}\PY{p}{(}\PY{n}{optimizer}\PY{o}{=}\PY{n}{my\PYZus{}optimizer}\PY{p}{,} \PY{n}{loss} \PY{o}{=} \PY{l+s+s1}{\PYZsq{}}\PY{l+s+s1}{mean\PYZus{}squared\PYZus{}error}\PY{l+s+s1}{\PYZsq{}}\PY{p}{)}    
    \PY{k}{return} \PY{n}{model}
\end{Verbatim}
\end{tcolorbox}

    \hypertarget{parameter-fuxfcr-die-kreuzvaliderung-definieren}{%
\subsubsection{Parameter für die Kreuzvaliderung
definieren}\label{parameter-fuxfcr-die-kreuzvaliderung-definieren}}

    \begin{tcolorbox}[breakable, size=fbox, boxrule=1pt, pad at break*=1mm,colback=cellbackground, colframe=cellborder]
\prompt{In}{incolor}{27}{\boxspacing}
\begin{Verbatim}[commandchars=\\\{\}]
\PY{c+c1}{\PYZsh{}\PYZhy{}\PYZhy{}\PYZhy{}\PYZhy{}\PYZhy{}\PYZhy{}\PYZhy{}\PYZhy{}\PYZhy{}\PYZhy{}\PYZhy{}\PYZhy{}\PYZhy{}\PYZhy{}\PYZhy{}\PYZhy{}\PYZhy{}\PYZhy{}\PYZhy{}\PYZhy{}\PYZhy{}\PYZhy{}\PYZhy{}\PYZhy{}\PYZhy{}\PYZhy{}\PYZhy{}\PYZhy{}\PYZhy{}\PYZhy{}\PYZhy{}\PYZhy{}}
\PY{c+c1}{\PYZsh{} Listen mit zu testenden Parametern}
\PY{n}{batch\PYZus{}size\PYZus{}list} \PY{o}{=} \PY{p}{[}\PY{l+m+mi}{15}\PY{p}{]}\PY{c+c1}{\PYZsh{}, 5, 25, 30, 50]}
\PY{n}{epochs\PYZus{}list} \PY{o}{=} \PY{p}{[}\PY{l+m+mi}{75}\PY{p}{]}\PY{c+c1}{\PYZsh{}, 25, 50, 150, 250]\PYZsh{}, 500, 1000]}
\PY{n}{learn\PYZus{}rate\PYZus{}list} \PY{o}{=} \PY{p}{[}\PY{l+m+mf}{0.0001}\PY{p}{]}\PY{c+c1}{\PYZsh{}, 0.001, 0.01]}
\PY{n}{dropout\PYZus{}list} \PY{o}{=} \PY{p}{[}\PY{l+m+mf}{0.0}\PY{p}{]}\PY{c+c1}{\PYZsh{}, 0.1, 0.3, 0.5]}
\PY{n}{weight\PYZus{}constraint} \PY{o}{=} \PY{p}{[}\PY{k+kc}{None}\PY{p}{]}\PY{c+c1}{\PYZsh{}, 1, 2, 3]}
\PY{n}{layer\PYZus{}unit\PYZus{}list} \PY{o}{=} \PY{p}{[}\PY{l+m+mi}{140}\PY{p}{]}\PY{c+c1}{\PYZsh{}, 20, 50, 80, 120, 180]\PYZsh{}, 120, 160]}
\PY{n}{kernel\PYZus{}initializer\PYZus{}list} \PY{o}{=} \PY{p}{[}\PY{l+s+s1}{\PYZsq{}}\PY{l+s+s1}{uniform}\PY{l+s+s1}{\PYZsq{}}\PY{p}{]}\PY{c+c1}{\PYZsh{}, \PYZsq{}normal\PYZsq{}, \PYZsq{}lecun\PYZus{}uniform\PYZsq{}, \PYZsq{}uniform\PYZsq{}, \PYZsq{}zero\PYZsq{}, \PYZsq{}glorot\PYZus{}normal\PYZsq{}, \PYZsq{}glorot\PYZus{}uniform\PYZsq{}, \PYZsq{}he\PYZus{}normal\PYZsq{}, \PYZsq{}he\PYZus{}uniform\PYZsq{}]}
\PY{n}{regularizer\PYZus{}list} \PY{o}{=} \PY{p}{[}\PY{k+kc}{None}\PY{p}{]}\PY{c+c1}{\PYZsh{}, regularizers.l2(0.001), regularizers.l2(0.01)]}
\PY{n}{activation\PYZus{}list} \PY{o}{=} \PY{p}{[}\PY{l+s+s1}{\PYZsq{}}\PY{l+s+s1}{relu}\PY{l+s+s1}{\PYZsq{}}\PY{p}{]}\PY{c+c1}{\PYZsh{}, \PYZsq{}tanh\PYZsq{}, \PYZsq{}sigmoid\PYZsq{}]\PYZsh{}, \PYZsq{}hard\PYZus{}sigmoid\PYZsq{}, \PYZsq{}linear\PYZsq{}, \PYZsq{}softmax\PYZsq{}, \PYZsq{}softplus\PYZsq{}, \PYZsq{}softsign\PYZsq{}] }

\PY{c+c1}{\PYZsh{}\PYZhy{}\PYZhy{}\PYZhy{}\PYZhy{}\PYZhy{}\PYZhy{}\PYZhy{}\PYZhy{}\PYZhy{}\PYZhy{}\PYZhy{}\PYZhy{}\PYZhy{}\PYZhy{}\PYZhy{}\PYZhy{}\PYZhy{}\PYZhy{}\PYZhy{}\PYZhy{}\PYZhy{}\PYZhy{}\PYZhy{}\PYZhy{}\PYZhy{}\PYZhy{}\PYZhy{}\PYZhy{}\PYZhy{}\PYZhy{}\PYZhy{}\PYZhy{}}
\PY{c+c1}{\PYZsh{} Erstellung Dictonary mit parametern}
\PY{n}{param\PYZus{}grid} \PY{o}{=} \PY{p}{\PYZob{}}\PY{l+s+s1}{\PYZsq{}}\PY{l+s+s1}{batch\PYZus{}size}\PY{l+s+s1}{\PYZsq{}}\PY{p}{:} \PY{n}{batch\PYZus{}size\PYZus{}list}\PY{p}{,}
              \PY{l+s+s1}{\PYZsq{}}\PY{l+s+s1}{epochs}\PY{l+s+s1}{\PYZsq{}}\PY{p}{:} \PY{n}{epochs\PYZus{}list}\PY{p}{,}
              \PY{l+s+s1}{\PYZsq{}}\PY{l+s+s1}{my\PYZus{}layer\PYZus{}units}\PY{l+s+s1}{\PYZsq{}}\PY{p}{:} \PY{n}{layer\PYZus{}unit\PYZus{}list}\PY{p}{,}
              \PY{l+s+s1}{\PYZsq{}}\PY{l+s+s1}{my\PYZus{}learn\PYZus{}rate}\PY{l+s+s1}{\PYZsq{}}\PY{p}{:} \PY{n}{learn\PYZus{}rate\PYZus{}list}\PY{p}{,}
              \PY{l+s+s1}{\PYZsq{}}\PY{l+s+s1}{my\PYZus{}dropout}\PY{l+s+s1}{\PYZsq{}}\PY{p}{:} \PY{n}{dropout\PYZus{}list}\PY{p}{,}
              \PY{l+s+s1}{\PYZsq{}}\PY{l+s+s1}{my\PYZus{}regularizer}\PY{l+s+s1}{\PYZsq{}}\PY{p}{:} \PY{n}{regularizer\PYZus{}list}\PY{p}{,}
              \PY{l+s+s1}{\PYZsq{}}\PY{l+s+s1}{my\PYZus{}activation}\PY{l+s+s1}{\PYZsq{}}\PY{p}{:} \PY{n}{activation\PYZus{}list}\PY{p}{,}
              \PY{l+s+s1}{\PYZsq{}}\PY{l+s+s1}{my\PYZus{}kernel\PYZus{}initializer}\PY{l+s+s1}{\PYZsq{}}\PY{p}{:} \PY{n}{kernel\PYZus{}initializer\PYZus{}list}\PY{p}{,}
              \PY{l+s+s1}{\PYZsq{}}\PY{l+s+s1}{my\PYZus{}weight\PYZus{}constraint}\PY{l+s+s1}{\PYZsq{}}\PY{p}{:} \PY{n}{weight\PYZus{}constraint}\PY{p}{\PYZcb{}}
\end{Verbatim}
\end{tcolorbox}

    \hypertarget{modell-finalisieren-und-kreuzvaliderung-starten}{%
\subsubsection{Modell finalisieren und Kreuzvaliderung
starten}\label{modell-finalisieren-und-kreuzvaliderung-starten}}

    \begin{tcolorbox}[breakable, size=fbox, boxrule=1pt, pad at break*=1mm,colback=cellbackground, colframe=cellborder]
\prompt{In}{incolor}{28}{\boxspacing}
\begin{Verbatim}[commandchars=\\\{\}]
\PY{l+s+sd}{\PYZdq{}\PYZdq{}\PYZdq{}}
\PY{l+s+sd}{\PYZsh{}\PYZhy{}\PYZhy{}\PYZhy{}\PYZhy{}\PYZhy{}\PYZhy{}\PYZhy{}\PYZhy{}\PYZhy{}\PYZhy{}\PYZhy{}\PYZhy{}\PYZhy{}\PYZhy{}\PYZhy{}\PYZhy{}\PYZhy{}\PYZhy{}\PYZhy{}\PYZhy{}\PYZhy{}\PYZhy{}\PYZhy{}\PYZhy{}\PYZhy{}\PYZhy{}\PYZhy{}\PYZhy{}\PYZhy{}\PYZhy{}\PYZhy{}\PYZhy{}}
\PY{l+s+sd}{\PYZsh{} Modell finalisieren}
\PY{l+s+sd}{model = KerasRegressor(build\PYZus{}fn=create\PYZus{}model, verbose=0)}
\PY{l+s+sd}{grid = GridSearchCV(estimator=model, param\PYZus{}grid=param\PYZus{}grid, cv=3, verbose=2, n\PYZus{}jobs=10)}

\PY{l+s+sd}{\PYZsh{}\PYZhy{}\PYZhy{}\PYZhy{}\PYZhy{}\PYZhy{}\PYZhy{}\PYZhy{}\PYZhy{}\PYZhy{}\PYZhy{}\PYZhy{}\PYZhy{}\PYZhy{}\PYZhy{}\PYZhy{}\PYZhy{}\PYZhy{}\PYZhy{}\PYZhy{}\PYZhy{}\PYZhy{}\PYZhy{}\PYZhy{}\PYZhy{}\PYZhy{}\PYZhy{}\PYZhy{}\PYZhy{}\PYZhy{}\PYZhy{}\PYZhy{}\PYZhy{}}
\PY{l+s+sd}{\PYZsh{} Kreuzvaliderung starten}
\PY{l+s+sd}{grid\PYZus{}result = grid.fit(X, y)}
\PY{l+s+sd}{\PYZdq{}\PYZdq{}\PYZdq{}}
\end{Verbatim}
\end{tcolorbox}

            \begin{tcolorbox}[breakable, size=fbox, boxrule=.5pt, pad at break*=1mm, opacityfill=0]
\prompt{Out}{outcolor}{28}{\boxspacing}
\begin{Verbatim}[commandchars=\\\{\}]
'\textbackslash{}n\#--------------------------------\textbackslash{}n\# Modell finalisieren\textbackslash{}nmodel =
KerasRegressor(build\_fn=create\_model, verbose=0)\textbackslash{}ngrid =
GridSearchCV(estimator=model, param\_grid=param\_grid, cv=3, verbose=2,
n\_jobs=10)\textbackslash{}n\textbackslash{}n\#--------------------------------\textbackslash{}n\# Kreuzvaliderung
starten\textbackslash{}ngrid\_result = grid.fit(X, y)\textbackslash{}n'
\end{Verbatim}
\end{tcolorbox}
        
    \hypertarget{ausgabe-der-besten-parameterkombination}{%
\subsubsection{Ausgabe der besten
Parameterkombination}\label{ausgabe-der-besten-parameterkombination}}

    \begin{tcolorbox}[breakable, size=fbox, boxrule=1pt, pad at break*=1mm,colback=cellbackground, colframe=cellborder]
\prompt{In}{incolor}{ }{\boxspacing}
\begin{Verbatim}[commandchars=\\\{\}]
\PY{l+s+sd}{\PYZdq{}\PYZdq{}\PYZdq{}}
\PY{l+s+sd}{print(\PYZsq{}MSE for best found hyperparameter combitnation: \PYZob{}\PYZcb{} with params: \PYZob{}\PYZcb{}\PYZsq{}.format(grid\PYZus{}result.best\PYZus{}score\PYZus{}, grid\PYZus{}result.best\PYZus{}params\PYZus{}))}
\PY{l+s+sd}{\PYZdq{}\PYZdq{}\PYZdq{}}
\end{Verbatim}
\end{tcolorbox}

    \hypertarget{alternatives-modell-ohne-kreuzvaliderung}{%
\subsection{Alternatives Modell ohne
Kreuzvaliderung}\label{alternatives-modell-ohne-kreuzvaliderung}}

    \begin{tcolorbox}[breakable, size=fbox, boxrule=1pt, pad at break*=1mm,colback=cellbackground, colframe=cellborder]
\prompt{In}{incolor}{29}{\boxspacing}
\begin{Verbatim}[commandchars=\\\{\}]
\PY{n}{model} \PY{o}{=} \PY{n}{create\PYZus{}model}\PY{p}{(}\PY{n}{my\PYZus{}layer\PYZus{}units}\PY{o}{=}\PY{l+m+mi}{140}\PY{p}{,} \PY{n}{my\PYZus{}learn\PYZus{}rate}\PY{o}{=}\PY{l+m+mf}{0.0001}\PY{p}{,} \PY{n}{my\PYZus{}dropout}\PY{o}{=}\PY{l+m+mf}{0.0}\PY{p}{,}
                     \PY{n}{my\PYZus{}weight\PYZus{}constraint}\PY{o}{=}\PY{k+kc}{None}\PY{p}{,} \PY{n}{my\PYZus{}regularizer}\PY{o}{=}\PY{k+kc}{None}\PY{p}{,} \PY{n}{my\PYZus{}kernel\PYZus{}initializer}\PY{o}{=}\PY{l+s+s1}{\PYZsq{}}\PY{l+s+s1}{normal}\PY{l+s+s1}{\PYZsq{}}\PY{p}{,}
                     \PY{n}{my\PYZus{}activation}\PY{o}{=}\PY{l+s+s1}{\PYZsq{}}\PY{l+s+s1}{relu}\PY{l+s+s1}{\PYZsq{}}\PY{p}{,} \PY{n}{input\PYZus{}dim}\PY{o}{=}\PY{n}{X\PYZus{}train}\PY{o}{.}\PY{n}{shape}\PY{p}{[}\PY{l+m+mi}{1}\PY{p}{]}\PY{p}{)}

\PY{n}{model}\PY{o}{.}\PY{n}{fit}\PY{p}{(}\PY{n}{x}\PY{o}{=}\PY{n}{X}\PY{p}{,} \PY{n}{y}\PY{o}{=}\PY{n}{y}\PY{p}{,} \PY{n}{batch\PYZus{}size}\PY{o}{=}\PY{l+m+mi}{15}\PY{p}{,} \PY{n}{verbose}\PY{o}{=}\PY{l+m+mi}{2}\PY{p}{,} \PY{n}{epochs}\PY{o}{=}\PY{l+m+mi}{40}\PY{p}{)}
\end{Verbatim}
\end{tcolorbox}

    \begin{Verbatim}[commandchars=\\\{\}]
Epoch 1/40
 - 0s - loss: 1.0356
Epoch 2/40
 - 0s - loss: 1.0095
Epoch 3/40
 - 0s - loss: 0.8904
Epoch 4/40
 - 0s - loss: 0.5008
Epoch 5/40
 - 0s - loss: 0.2658
Epoch 6/40
 - 0s - loss: 0.2420
Epoch 7/40
 - 0s - loss: 0.1796
Epoch 8/40
 - 0s - loss: 0.2195
Epoch 9/40
 - 0s - loss: 0.1991
Epoch 10/40
 - 0s - loss: 0.1796
Epoch 11/40
 - 0s - loss: 0.1835
Epoch 12/40
 - 0s - loss: 0.1765
Epoch 13/40
 - 0s - loss: 0.1394
Epoch 14/40
 - 0s - loss: 0.1782
Epoch 15/40
 - 0s - loss: 0.1634
Epoch 16/40
 - 0s - loss: 0.1678
Epoch 17/40
 - 0s - loss: 0.1695
Epoch 18/40
 - 0s - loss: 0.1325
Epoch 19/40
 - 0s - loss: 0.1478
Epoch 20/40
 - 0s - loss: 0.1306
Epoch 21/40
 - 0s - loss: 0.1473
Epoch 22/40
 - 0s - loss: 0.1408
Epoch 23/40
 - 0s - loss: 0.1225
Epoch 24/40
 - 0s - loss: 0.1304
Epoch 25/40
 - 0s - loss: 0.1468
Epoch 26/40
 - 0s - loss: 0.1367
Epoch 27/40
 - 0s - loss: 0.1299
Epoch 28/40
 - 0s - loss: 0.1090
Epoch 29/40
 - 0s - loss: 0.1355
Epoch 30/40
 - 0s - loss: 0.1343
Epoch 31/40
 - 0s - loss: 0.1388
Epoch 32/40
 - 0s - loss: 0.1145
Epoch 33/40
 - 0s - loss: 0.1249
Epoch 34/40
 - 0s - loss: 0.1206
Epoch 35/40
 - 0s - loss: 0.1393
Epoch 36/40
 - 0s - loss: 0.1006
Epoch 37/40
 - 0s - loss: 0.1161
Epoch 38/40
 - 0s - loss: 0.1458
Epoch 39/40
 - 0s - loss: 0.1320
Epoch 40/40
 - 0s - loss: 0.1377
    \end{Verbatim}

            \begin{tcolorbox}[breakable, size=fbox, boxrule=.5pt, pad at break*=1mm, opacityfill=0]
\prompt{Out}{outcolor}{29}{\boxspacing}
\begin{Verbatim}[commandchars=\\\{\}]
<keras.callbacks.callbacks.History at 0x25d6f5afa88>
\end{Verbatim}
\end{tcolorbox}
        
    \hypertarget{prognose}{%
\subsubsection{Prognose}\label{prognose}}

\hypertarget{auf-den-testdaten}{%
\paragraph{Auf den Testdaten}\label{auf-den-testdaten}}

    \begin{tcolorbox}[breakable, size=fbox, boxrule=1pt, pad at break*=1mm,colback=cellbackground, colframe=cellborder]
\prompt{In}{incolor}{30}{\boxspacing}
\begin{Verbatim}[commandchars=\\\{\}]
\PY{n}{pred} \PY{o}{=} \PY{n}{model}\PY{o}{.}\PY{n}{predict}\PY{p}{(}\PY{n}{X}\PY{p}{)}
\PY{n}{pred} \PY{o}{=} \PY{n}{y\PYZus{}scalar}\PY{o}{.}\PY{n}{inverse\PYZus{}transform}\PY{p}{(}\PY{n}{pred}\PY{o}{.}\PY{n}{reshape}\PY{p}{(}\PY{o}{\PYZhy{}}\PY{l+m+mi}{1}\PY{p}{,} \PY{l+m+mi}{1}\PY{p}{)}\PY{p}{)}
\PY{n}{pred}\PY{p}{[}\PY{p}{:}\PY{l+m+mi}{5}\PY{p}{]}
\end{Verbatim}
\end{tcolorbox}

            \begin{tcolorbox}[breakable, size=fbox, boxrule=.5pt, pad at break*=1mm, opacityfill=0]
\prompt{Out}{outcolor}{30}{\boxspacing}
\begin{Verbatim}[commandchars=\\\{\}]
array([[108414.016],
       [136084.05 ],
       [133209.62 ],
       [247302.9  ],
       [624315.94 ]], dtype=float32)
\end{Verbatim}
\end{tcolorbox}
        
    \begin{tcolorbox}[breakable, size=fbox, boxrule=1pt, pad at break*=1mm,colback=cellbackground, colframe=cellborder]
\prompt{In}{incolor}{31}{\boxspacing}
\begin{Verbatim}[commandchars=\\\{\}]
\PY{n}{test} \PY{o}{=} \PY{n}{y\PYZus{}scalar}\PY{o}{.}\PY{n}{inverse\PYZus{}transform}\PY{p}{(}\PY{n}{y}\PY{p}{)}\PY{p}{[}\PY{p}{:}\PY{l+m+mi}{5}\PY{p}{]}
\PY{n}{test}
\end{Verbatim}
\end{tcolorbox}

            \begin{tcolorbox}[breakable, size=fbox, boxrule=.5pt, pad at break*=1mm, opacityfill=0]
\prompt{Out}{outcolor}{31}{\boxspacing}
\begin{Verbatim}[commandchars=\\\{\}]
array([[137500.],
       [132000.],
       [147000.],
       [245000.],
       [745000.]])
\end{Verbatim}
\end{tcolorbox}
        
    \hypertarget{auf-den-validierungsdaten}{%
\paragraph{Auf den Validierungsdaten}\label{auf-den-validierungsdaten}}

    \begin{tcolorbox}[breakable, size=fbox, boxrule=1pt, pad at break*=1mm,colback=cellbackground, colframe=cellborder]
\prompt{In}{incolor}{32}{\boxspacing}
\begin{Verbatim}[commandchars=\\\{\}]
\PY{n}{pred} \PY{o}{=} \PY{n}{model}\PY{o}{.}\PY{n}{predict}\PY{p}{(}\PY{n}{X\PYZus{}validierung}\PY{p}{)}
\PY{n}{pred} \PY{o}{=} \PY{n}{y\PYZus{}scalar}\PY{o}{.}\PY{n}{inverse\PYZus{}transform}\PY{p}{(}\PY{n}{pred}\PY{o}{.}\PY{n}{reshape}\PY{p}{(}\PY{o}{\PYZhy{}}\PY{l+m+mi}{1}\PY{p}{,} \PY{l+m+mi}{1}\PY{p}{)}\PY{p}{)}
\PY{n}{pred}\PY{p}{[}\PY{p}{:}\PY{l+m+mi}{5}\PY{p}{]}
\end{Verbatim}
\end{tcolorbox}

            \begin{tcolorbox}[breakable, size=fbox, boxrule=.5pt, pad at break*=1mm, opacityfill=0]
\prompt{Out}{outcolor}{32}{\boxspacing}
\begin{Verbatim}[commandchars=\\\{\}]
array([[298377.06],
       [129082.86],
       [119308.51],
       [152271.42],
       [181709.47]], dtype=float32)
\end{Verbatim}
\end{tcolorbox}
        
    \begin{tcolorbox}[breakable, size=fbox, boxrule=1pt, pad at break*=1mm,colback=cellbackground, colframe=cellborder]
\prompt{In}{incolor}{33}{\boxspacing}
\begin{Verbatim}[commandchars=\\\{\}]
\PY{n}{test} \PY{o}{=} \PY{n}{y\PYZus{}scalar}\PY{o}{.}\PY{n}{inverse\PYZus{}transform}\PY{p}{(}\PY{n}{y\PYZus{}validierung}\PY{p}{)}\PY{p}{[}\PY{p}{:}\PY{l+m+mi}{5}\PY{p}{]}
\PY{n}{test}
\end{Verbatim}
\end{tcolorbox}

            \begin{tcolorbox}[breakable, size=fbox, boxrule=.5pt, pad at break*=1mm, opacityfill=0]
\prompt{Out}{outcolor}{33}{\boxspacing}
\begin{Verbatim}[commandchars=\\\{\}]
array([[310000.],
       [131000.],
       [110000.],
       [155000.],
       [174000.]])
\end{Verbatim}
\end{tcolorbox}
        
    \hypertarget{testdaten-fit-machen-und-submit-durchfuxfchren}{%
\paragraph{Testdaten fit machen und Submit
durchführen}\label{testdaten-fit-machen-und-submit-durchfuxfchren}}

    \begin{tcolorbox}[breakable, size=fbox, boxrule=1pt, pad at break*=1mm,colback=cellbackground, colframe=cellborder]
\prompt{In}{incolor}{34}{\boxspacing}
\begin{Verbatim}[commandchars=\\\{\}]
\PY{c+c1}{\PYZsh{}\PYZhy{}\PYZhy{}\PYZhy{}\PYZhy{}\PYZhy{}\PYZhy{}\PYZhy{}\PYZhy{}\PYZhy{}\PYZhy{}\PYZhy{}\PYZhy{}\PYZhy{}\PYZhy{}\PYZhy{}\PYZhy{}\PYZhy{}\PYZhy{}\PYZhy{}\PYZhy{}\PYZhy{}\PYZhy{}\PYZhy{}\PYZhy{}\PYZhy{}\PYZhy{}\PYZhy{}\PYZhy{}\PYZhy{}\PYZhy{}\PYZhy{}\PYZhy{}}
\PY{c+c1}{\PYZsh{} Fit machen}
\PY{n}{X} \PY{o}{=} \PY{n}{make\PYZus{}test\PYZus{}data\PYZus{}fit}\PY{p}{(}\PY{n}{test\PYZus{}data}\PY{p}{,} \PY{n}{my\PYZus{}scaler}\PY{o}{=}\PY{n}{RobustScaler}\PY{p}{)}
\PY{c+c1}{\PYZsh{}\PYZhy{}\PYZhy{}\PYZhy{}\PYZhy{}\PYZhy{}\PYZhy{}\PYZhy{}\PYZhy{}\PYZhy{}\PYZhy{}\PYZhy{}\PYZhy{}\PYZhy{}\PYZhy{}\PYZhy{}\PYZhy{}\PYZhy{}\PYZhy{}\PYZhy{}\PYZhy{}\PYZhy{}\PYZhy{}\PYZhy{}\PYZhy{}\PYZhy{}\PYZhy{}\PYZhy{}\PYZhy{}\PYZhy{}\PYZhy{}\PYZhy{}\PYZhy{}}
\PY{c+c1}{\PYZsh{} Submitten}
\PY{n}{submission} \PY{o}{=} \PY{n}{submit}\PY{p}{(}\PY{n}{model}\PY{p}{,} \PY{n}{X}\PY{p}{,} \PY{n}{y\PYZus{}scalar}\PY{p}{,} \PY{n}{save}\PY{o}{=}\PY{k+kc}{True}\PY{p}{,} \PY{n}{manu\PYZus{}name}\PY{o}{=}\PY{k+kc}{True}\PY{p}{)}
\PY{n}{submission}\PY{o}{.}\PY{n}{head}\PY{p}{(}\PY{p}{)}
\end{Verbatim}
\end{tcolorbox}

            \begin{tcolorbox}[breakable, size=fbox, boxrule=.5pt, pad at break*=1mm, opacityfill=0]
\prompt{Out}{outcolor}{34}{\boxspacing}
\begin{Verbatim}[commandchars=\\\{\}]
    Id     SalePrice
0  478  405431.31250
1  311  174138.71875
2  798  103350.93750
3  795  197373.25000
4  247  140921.96875
\end{Verbatim}
\end{tcolorbox}
        
    \hypertarget{betrachtung-der-verteilung-auf-testdatensatz}{%
\paragraph{Betrachtung der Verteilung auf
Testdatensatz}\label{betrachtung-der-verteilung-auf-testdatensatz}}

    \begin{tcolorbox}[breakable, size=fbox, boxrule=1pt, pad at break*=1mm,colback=cellbackground, colframe=cellborder]
\prompt{In}{incolor}{35}{\boxspacing}
\begin{Verbatim}[commandchars=\\\{\}]
\PY{n}{submission}\PY{o}{.}\PY{n}{hist}\PY{p}{(}\PY{n}{column}\PY{o}{=}\PY{p}{[}\PY{l+s+s1}{\PYZsq{}}\PY{l+s+s1}{SalePrice}\PY{l+s+s1}{\PYZsq{}}\PY{p}{]}\PY{p}{)}  
\end{Verbatim}
\end{tcolorbox}

            \begin{tcolorbox}[breakable, size=fbox, boxrule=.5pt, pad at break*=1mm, opacityfill=0]
\prompt{Out}{outcolor}{35}{\boxspacing}
\begin{Verbatim}[commandchars=\\\{\}]
array([[<matplotlib.axes.\_subplots.AxesSubplot object at 0x0000025D7011F488>]],
      dtype=object)
\end{Verbatim}
\end{tcolorbox}
        
    \begin{center}
    \adjustimage{max size={0.9\linewidth}{0.9\paperheight}}{output_60_1.png}
    \end{center}
    { \hspace*{\fill} \\}
    
    \begin{tcolorbox}[breakable, size=fbox, boxrule=1pt, pad at break*=1mm,colback=cellbackground, colframe=cellborder]
\prompt{In}{incolor}{36}{\boxspacing}
\begin{Verbatim}[commandchars=\\\{\}]
\PY{n}{submission}\PY{p}{[}\PY{l+s+s1}{\PYZsq{}}\PY{l+s+s1}{SalePrice}\PY{l+s+s1}{\PYZsq{}}\PY{p}{]}\PY{o}{.}\PY{n}{describe}\PY{p}{(}\PY{p}{)}
\end{Verbatim}
\end{tcolorbox}

            \begin{tcolorbox}[breakable, size=fbox, boxrule=.5pt, pad at break*=1mm, opacityfill=0]
\prompt{Out}{outcolor}{36}{\boxspacing}
\begin{Verbatim}[commandchars=\\\{\}]
count      1000.000000
mean     181470.609375
std       75372.414062
min       27863.453125
25\%      131303.710938
50\%      156153.726562
75\%      211499.222656
max      838150.437500
Name: SalePrice, dtype: float64
\end{Verbatim}
\end{tcolorbox}
        
    \hypertarget{verteilung-traingsdatensatz-zum-vergleich}{%
\subsubsection{Verteilung Traingsdatensatz zum
Vergleich}\label{verteilung-traingsdatensatz-zum-vergleich}}

    \begin{tcolorbox}[breakable, size=fbox, boxrule=1pt, pad at break*=1mm,colback=cellbackground, colframe=cellborder]
\prompt{In}{incolor}{37}{\boxspacing}
\begin{Verbatim}[commandchars=\\\{\}]
\PY{n}{train\PYZus{}data}\PY{o}{.}\PY{n}{hist}\PY{p}{(}\PY{n}{column}\PY{o}{=}\PY{p}{[}\PY{l+s+s1}{\PYZsq{}}\PY{l+s+s1}{SalePrice}\PY{l+s+s1}{\PYZsq{}}\PY{p}{]}\PY{p}{)}  
\end{Verbatim}
\end{tcolorbox}

            \begin{tcolorbox}[breakable, size=fbox, boxrule=.5pt, pad at break*=1mm, opacityfill=0]
\prompt{Out}{outcolor}{37}{\boxspacing}
\begin{Verbatim}[commandchars=\\\{\}]
array([[<matplotlib.axes.\_subplots.AxesSubplot object at 0x0000025D70340EC8>]],
      dtype=object)
\end{Verbatim}
\end{tcolorbox}
        
    \begin{center}
    \adjustimage{max size={0.9\linewidth}{0.9\paperheight}}{output_63_1.png}
    \end{center}
    { \hspace*{\fill} \\}
    
    \begin{tcolorbox}[breakable, size=fbox, boxrule=1pt, pad at break*=1mm,colback=cellbackground, colframe=cellborder]
\prompt{In}{incolor}{38}{\boxspacing}
\begin{Verbatim}[commandchars=\\\{\}]
\PY{n}{train\PYZus{}data}\PY{p}{[}\PY{l+s+s1}{\PYZsq{}}\PY{l+s+s1}{SalePrice}\PY{l+s+s1}{\PYZsq{}}\PY{p}{]}\PY{o}{.}\PY{n}{describe}\PY{p}{(}\PY{p}{)}
\end{Verbatim}
\end{tcolorbox}

            \begin{tcolorbox}[breakable, size=fbox, boxrule=.5pt, pad at break*=1mm, opacityfill=0]
\prompt{Out}{outcolor}{38}{\boxspacing}
\begin{Verbatim}[commandchars=\\\{\}]
count       460.000000
mean     179771.752174
std       77419.758683
min       40000.000000
25\%      130187.500000
50\%      162950.000000
75\%      207750.000000
max      745000.000000
Name: SalePrice, dtype: float64
\end{Verbatim}
\end{tcolorbox}
        

    % Add a bibliography block to the postdoc
    
    
    
\end{document}
